\documentclass[algorithmlist, figurelist,tablelist, nomlist, engineering, AutoFakeBold]{config/seuthesiY}
\usepackage{amsmath}
\usepackage{enumitem}
\usepackage{graphicx}
\usepackage{booktabs}
\usepackage{multirow}
\usepackage{listings}
\usepackage{xcolor}
\usepackage{enumitem}

\lstset{
    basicstyle=\ttfamily,        % 代码字体
    keywordstyle=\bfseries,      % 关键词加粗
    columns=fullflexible,        % 适应文本宽度
    frame=single,                % 加框
    breaklines=true,             % 自动换行
    postbreak=\hbox{$\hookrightarrow$}, % 换行符
    xleftmargin=10pt,            % 左边距
}

%===============================================================================
% 采用的编译方式为  XELATEX -→  BIBER -→    XELATEX -→    XELATEX
% 为了加快字体的缓存效率 需要在命令行运行 fc-cache
% 对于学位论文需要标注【硕博】
%===============================================================================

% %biblatex宏包的参考文献数据源加载方式
\addbibresource[location=local]{config/seuthesiY.bib}

\begin{document}
%===============================================================================
\categorynumber{000} % 分类采用《中国图书资料分类法》
\UDC{000}            %《国际十进分类法UDC》的类号
\secretlevel{公开}    %学位论文密级分为"公开"、"内部"、"秘密"和"机密"四种
\studentid{222171}   %学号要完整,前面的零不能省略。
\title{积木世界VQA中的}{积木世界VQA中的}{空间推理问答技术研究与实现}{空间推理问答技术研究与实现}{Research and Implementation of Spatial Reasoning Questioning Answering Techniques }{in the Block World VQA}
\author{贾梁}{Jia Liang}
\advisor{张志政}{}{Zhang Zhizheng}{}
% \coadvisor{张志政}{副教授}{Zhang Zhizheng}{Associate Prof.} % 没有% 可以不填
\degreetype{工程硕士}{Master of Engineering} % 详细学位名称
\thesisform{应用研究} % 包括应用研究、调研报告、规划、产品开发、案例分析、项目管理、文学艺术作品、其它。非专业型硕士可忽略
\major{电子信息}
\submajor{计算机技术}
\defenddate{2025年5月30日}
\authorizedate{2025年6月20日}
\committeechair{翟玉庆}
\reviewer{倪庆剑}{张祥}
\department{东南大学计算机科学与工程学院}{School of Computer Science and Engineering}
\makebigcover
\makecover
\begin{abstract}{回答集编程,视觉问答,空间推理,神经符号方法}
积木世界是人工智能研究、教学、实验、评估的重要场景,能够模拟实际应用中物体间的空间关系。
积木世界VQA要求视觉语言模型(Visual Language Model, VLM)具备空间推理能力,但已有研究表明,当面对部分可见场景时,
其回答准确率显著下降。目前,通过结合神经网络和符号推理形成的神经符号模型,利用深度学习将视觉信息转化为逻辑符号,
再利用符号推理进行问题求解问题,比单纯依赖深度学习的视觉语言模型在空间推理任务中有更优异的表现。
由于回答集程序(Answer Set Program, ASP)是一种具备非单调推理和高效推理机的符号推理方法,
神经符号模型中采用ASP构建的神经符号VQA框架被寄予厚望。然而,现有框架设计中ASP规则的扩展仍依赖人工,
难以有效应对部分可见场景中基于空间推理的VQA任务。
    
针对上述问题,本文从构建积木世界部分可见场景中基于空间推理的VQA数据集、
设计规则自动补充的神经符号VQA方法、设计实现VQA课堂演示原型系统三个方面开展工作,具体如下:

\begin{enumerate}[nosep]
\item 部分可见积木世界场景空间推理VQA数据集(Partial Observation VQA Dataset, POVQA\-D)构建。
CLEVR是积木世界的经典数据集,然而CLEVR中问题涉及的物体属性和物体间空间关系在图像中均完全可见,
通过引入部分可见性与环境约束,构建了覆盖部分场景可见问题的VQA数据集POVQAD。
POVQAD要求模型能够利用背景知识和部分可见场景中的信息进行推理
,比原数据集更能有效考察模型在部分可见积木世界场景中回答空间推理问题的能力。
\item 规则自动补充的神经符号VQA框架(Rule Complement Neuro-Symbolic Pipeline, RCNSP)设计。
RCNSP借鉴现有神经符号VQA框架的工作,对LLM进行提示词优化以提高ASP规则生成、规则修正的语法及语义准确率,
并新增规则蒸馏模块以实现ASP规则自动拓展。
实验表明,在DeepSeek、LLaMA3、ChatGPT-4o三种大语言模型上,使用RCNSP比直接向VLM提问的准确率平均提升16.5\%,
比现有的神经符号VQA框架的准确率平均提升8.9\%,
表明通过规则蒸馏能有效提升神经符号方法在部分可见积木世界场景下空间推理问答的准确率。
\item 设计实现了一个积木世界VQA原型系统。在RCNSP框架和POVQAD数据集基础上,设计实现了一个
积木世界VQA原型系统。
该系统模拟了在自动规划课程的授课场景下,由教师向系统提出积木世界的空间推理问题,系统进行解答并展示推理的中间步骤和逻辑链条,
并支持自定义生成复杂度不同的演示数据集,
为教师向学生展示智能体如何理解外部环境信息并进行规划提供了便利。
初步测试表明该系统在CPU为Intel Core i9-12900K,内存128G,显卡为3张RTX 3090并联的硬件环境下,并发量为38,
90\%响应时间为6.9秒,能够满足自动规划课程教学场景下的用户需要。
\end{enumerate}
\end{abstract}

\begin{englishabstract}{Answer Set Programming, Visual Question Answering, Spatial Reasoning, Neuro-symbolic Method}
The Block World is a fundamental scenario for research, education, experimentation, and evaluation in artificial intelligence, capable of simulating spatial relationships among objects in real-world applications. Block World Visual Question Answering (VQA) tasks require Visual Language Models (VLMs) to possess spatial reasoning capabilities. However, existing studies have shown that the accuracy of VLMs significantly drops in partially observable environments. Neuro-symbolic models, which integrate neural networks and symbolic reasoning, convert visual information into logical symbols using deep learning and solve problems through symbolic inference. These models have demonstrated superior performance in spatial reasoning tasks compared to purely neural-based VLMs.

Answer Set Programming (ASP), a symbolic reasoning paradigm with non-monotonic reasoning and efficient solvers, has been widely adopted in neuro-symbolic VQA frameworks. However, current frameworks still rely heavily on manual rule construction, making it difficult to handle spatial reasoning tasks in partially observable scenarios effectively.

To address this issue, this work focuses on three main aspects: constructing a spatial reasoning VQA dataset under partial observability in Block World, designing a neuro-symbolic VQA method with automatic rule augmentation, and developing a classroom demonstration prototype system for VQA. Specifically:

\begin{enumerate}[nosep]
\item \textbf{Construction of the Partial Observation VQA Dataset (POVQAD):}  
CLEVR is a classical dataset for spatial reasoning, but all object attributes and spatial relations in its scenes are fully visible. We introduce partial observability and environmental constraints to construct POVQAD, a dataset that challenges models to reason with background knowledge and limited visual input. Compared to existing datasets, POVQAD better evaluates a model's capability in answering spatial reasoning questions under partial observability in the Block World.

\item \textbf{Design of the Rule Complement Neuro-Symbolic Pipeline (RCNSP):}  
RCNSP builds upon existing neuro-symbolic frameworks and enhances large language models (LLMs) with prompt optimization to improve the syntactic and semantic accuracy of ASP rule generation and correction. Furthermore, it introduces a rule distillation module to automatically expand ASP rules. Experiments with DeepSeek, LLaMA3, and ChatGPT-4o demonstrate that RCNSP improves the average accuracy by 16.5\% over direct VLM questioning and by 8.9\% over existing neuro-symbolic VQA frameworks. These results confirm that rule distillation significantly enhances the spatial reasoning accuracy in partially observable Block World scenarios.

\item \textbf{Implementation of a Block World VQA Prototype System:}  
Based on the RCNSP framework and the POVQAD dataset, a prototype system for Block World VQA is implemented. This system simulates classroom teaching of automated planning, where instructors pose spatial reasoning questions to the system. The system provides answers along with intermediate reasoning steps and logical chains, helping students understand how intelligent agents perceive and plan in their environments. Preliminary tests show that on a system with an Intel Core i9-12900K CPU, 128GB RAM, and three RTX 3090 GPUs, the system supports a concurrency of 38 with a 90\% response time of 6.9 seconds, meeting the performance requirements of educational scenarios in automated planning.
\end{enumerate}
\end{englishabstract}

\setnomname{术语与符号约定}
\tableofcontents
\listofothers
%===============================================================================


\mainmatter

\chapter{绪论}
\section{研究背景}
随着计算机视觉(Computer Vision)
和自然语言处理(Natural Language Processing)技术的迅猛发展,
跨模态智能理解成为人工智能研究的重要方向之一。
其中,视觉问答(Visual Question Answering, VQA)\cite{goyal2017making}任务因其广泛的应用前景和挑战性,
受到了学术界和工业界的广泛关注。

视觉问答是一种多模态任务,它要求计算机能够基于给定的图像内容,
理解并回答关于该图像的自然语言问题。例如,给定一幅包含动物的图片,
系统需要能够回答“这只动物是什么颜色?”或“图片中有几只猫?”等问题。
这一任务的核心在于多模态信息的深度融合,即如何在视觉特征和语言信息之间建立有效的联系。

视觉问答在多个实际应用场景中具有重要价值。
例如,在辅助盲人阅读图像信息、自主机器人理解环境、医疗影像分析、教育和娱乐等方面,
VQA 技术都可以提供智能化的交互方式,提高系统的可用性和便利性。
然而,由于数据的不确定性、语言表达的多样性、以及多模态信息融合的复杂性,
VQA 仍然面临诸多挑战,例如开放域问题的泛化能力、推理能力的提升、以及对长尾问题的有效应对等。

随着现实场景愈发复杂多样,VQA系统需要解决的问题也愈发困难,尤其是涉及物体间相对位置、形状、
大小等空间关系的推理任务。空间推理指的是理解物体在某个二维或者三维的场景中的位置关系,并能够根据
该场景的信息,以及本身所有的关于空间的常识知识,回答涉及物体排列、方向、距离等方面的问题。
例如,在问答任务中,问题可能为“杯子在桌子的哪一侧?”或者“哪个物体位于另一个物体的上方?”这些问题
要求VQA系统能够理解图像中物体的空间关系,进行推理,并给出准确的答案。

相比传统的VQA任务,空间推理更具挑战性。为了正确回答涉及空间推理的问题,VQA系统需要能够理解
图像中的空间布局,并且在多步骤推理中保持逻辑一致性。然而,现有的大语言模型在空间推理方面仍然存在
显著局限。虽然已有的模型在目标检测方面表现十分出色,但它们在处理更复杂的空间关系时,尤其是
在缺乏常识性空间理解的情况下,容易出现推理错误。因此,如何提升大语言模型在空间推理方面的能力,
成为当前研究的重要课题。

为了解决VQA系统在空间推理方面的问题,研究者们提出了一些新的方法。近年来,神经符号方法(Neuro-Symbolic Methods)得到了广泛关注。
神经符号方法使用深度学习技术进行感知,为输入的图像和问题分别生成符号表示,
再使用符号系统进行推理求解,可以有效提升VQA系统在空间推理方面的能力。回答集编程(Answer Set Programming,ASP)是一种声明式编程范式,可用于解决复杂的
人工智能问题。ASP起源于对逻辑编程、非单调推理和知识表示的研究。ASP因其具有表达
性声明性的语言和以Clingo为代表的一些高效实现而流行起来。ASP已经在学术界和
工业界得到广泛应用,并被证明在人工智能的几个知识密集型应用中能够有效解决问题,
如调度、产品配置、机器人、劳动力管理和决策支持等。
ASP能够以简洁和直观的方式来表示知识,允许用户相对容易地去表示复杂问题,
而且ASP具有分单调推理的特性,允许不完整信息的表示和默认推理。
另外,ASP对知识的表达能力,使得其支持集成各种类型的知识,包括规则、约束和偏好,有助于灵活解决问题。
ASP的这些特性使得它作为符号系统的杰出代表,被广泛应用于神经符号方法中。

基于上述背景,本文重点关注将神经符号方法在视觉问答中的应用,分析研究现有已使用神经符号方法的VQA系统在空间推理方面的优势和不足,
,为进一步提升大语言模型在空间推理方面的能力开展研究。

\section{相关研究现状}
基于以上背景,本节主要从视觉问答的发展与现状、大语言模型在空间推理中的局限性、神经符号方法在视觉问答中的应用、视觉问答中复杂空间推理的难点等方面进行综述。

\subsection{视觉问答的发展与现状}
VQA任务最早由Antol\cite{Antol2015VQA}等人在2015年提出,标志着VQA研究早期阶段的开始。这一阶段的VQA系统采用比较简单的架构,主要包括视觉编码器、语言编码器和融合模块。
Ren\cite{ren2015exploring}等人将采用预训练的卷积神经网络(Convolutional Neural Network, CNN)作为视觉编码器提取图像特征,采用循环神经网络(Recurrent Neural Network, RNN)作为语言编码器或
提取问题特征。Malinowski\cite{malinowski2015neural}等人在视觉编码器上同样采用CNN,在语言编码器上则是采用长短期记忆(Long Short-Term Memory , LSTM)来生成问题的答案。
融合模块通常采用简单的拼接或注意力机制融合视觉和语言特征。这些系统在一些简单的VQA数据集上取得了不错的性能,但在处理复杂的空间推理问题时表现较差。

随着计算机视觉和自然语言处理技术的发展,VQA系统逐步引入了更高级的特征提取与融合方法。
例如,Yao\cite{lu2019look}等人将区域提议网络(RPN)引入到VQA模型中,以增强对图像中物体的感知能力。
注意力机制也逐渐引入VQA模型中,例如Xu\cite{xu2016stacked}等人提出堆叠注意力网络(SANs),
通过多层次注意力机制,逐步聚焦于图像和问题的关键部分,从而提升了VQA模型的推理能力。

在2019年前后,得益于基于Transformer的预训练语言模型的兴起,VQA研究取得了显著进展。如BERT、ViLBERT和LXMERT等模型通过联合训练大规模的视觉和语言数据,
学得如何在同一嵌入空间中表示视觉和文本信息,极大程度上提升了VQA的准确性和鲁棒性。
例如,Tan\cite{Tan2019LXMERT}等人提出了LXMERT模型,将预训练的BERT模型与视觉编码器相结合,实现了对图像和问题的联合编码。

如今,VQA的研究已经进入了全新阶段。随着大规模视觉语言模型(如GLIP、GPT-4等)的应用,VQA模型已经能够处理更加复杂的任务,不仅处理特定问题时效果出色,而且可以进行
跨模态推理,通过多种信息源之间的互动生成更加丰富和精准的答案。OpenAI的研究团队\cite{radford2021learning}在CLIP模型中,应用了自监督学习的方法
,利用海量的图文对数据进行训练,实现了视觉和语言的统一表示,且训练过程无需人工标注,充分体现了自监督学习的优势。Salesforce的研究团队\cite{li2022blip}在BLIP模型中,
通过生成式预训练任务,提升了视觉语言模型在理解和生成方面的性能。

\subsection{大语言模型在空间推理中的局限性}
当前,大语言模型在空间推理方面的能力仍存在显著局限。Cohn\cite{cohn2023evaluation}等人在 RCC-8 框架下对 GPT-4 在定性空间推理任务中的表现进行了评估,发现尽管 GPT-4 能够理解一些简单的空间关系,但在处理复杂的空间关系时,往往无法准确应用 RCC-8 的规则,导致推理精度较低。此外,GPT-4 在不同推理步骤之间表现出明显的不一致性,且其推理过程缺乏明确的策略,往往依赖于经验或直觉做出决策,但这些决策往往缺乏可解释性,难以追踪其推理路径。
Bang\cite{bang2023multitask}等人也指出,GPT 在空间推理任务中面临的挑战尤为突出,特别是在涉及多个空间区域、动态变化的场景或高维空间结构时,模型倾向于依赖已知的简单规则进行推理,但在面对复杂或不常见的空间配置时,难以有效处理。此外,GPT 在基于空间关系进行推理时,往往会出现错误,无法从准确的空间关系中得出合乎逻辑的结论。特别是在涉及常识性空间理解的任务中,GPT 的表现较差,难以理解如物体重叠、邻接、相交等基本的空间常识性假设。

为了解决这一问题,研究者们提出了一些新的方法,如基于神经符号方法的空间推理框架。这些方法通过将大语言模型与符号推理方法相结合,实现了对复杂空间推理问题的有效建模。
例如,Wang \cite{ishay2023leveraging}等人提出了一种基于大语言模型和 Answer Set Programming(ASP)的神经符号框架,用于解决复杂空间推理问题。
该方法通过将大语言模型与 ASP 求解器相结合,实现了对复杂空间推理问题的高效建模。此外,研究者们还提出了一些其他的神经符号方法,如基于知识图谱的推理方法、基于逻辑规划的推理方法等,以增强大语言模型在空间推理方面的能力。
\subsection{ASP在空间推理中的应用}
ASP作为一种形式化的知识表示和推理方法,已经在空间推理中得到了广泛的应用。ASP具有表达能力强、推理效率高、易于理解和调试等优点,适用于解决复杂的空间推理问题。有一些学者在这一方面做了一些研究。
例如,Wałęga\cite{walega2015aspmtqs}等人提出ASPMT(QS),将非单调空间推理与基于理论的答案集编程相结合,整合了定性和定量的空间信息,
解决了传统空间推理方法在处理复杂空间变化和组合约束时的局限性。
Baryannis\cite{Baryannis2018Trajectory}等人3将轨迹建模为由不重叠区域构成的序列,
并提出了多种ASP编码方案(包括专门优化的编码和通用编码)以利用合成表来保证约束的一致性,展示了ASP在空间推理中的应用潜力。
这些研究表明,ASP在空间推理中具有很大的潜力,可以有效解决复杂的空间推理问题。

\subsection{神经符号方法在空间推理中的应用}
尽管研究者将ASP成功引入空间推理问题,取得了一些研究成果,但ASP在感知能力方面仍然存在一些不足,如对图像和自然语言的理解能力较弱,难以处理复杂的视觉问答问题。
神经符号方法的出现,为解决这一问题提供了新的思路。
神经符号方法引入深度学习技术,为ASP求解器提供了更加丰富的输入信息,从而提升了空间推理的准确性和效率。
Tejas\cite{Gokhale2020CausalVQA}等人“逻辑透镜”(Lens of Logic, LOL)模型,采用了神经符号方法,利用了问题注意力
和逻辑注意力机制,以识别和理解问题汇总的逻辑连接词,并且引入了Fréchet兼容性损失(Fréchet-Compatibility Loss),
确保组件问题的答案与组合问题的答案在推理过程中保持一致性。
Thomas\cite{eiter2022neuro}等人提出了一种结合神经网络和符号推理的方法,该方法能够有效地进行空间推理和逻辑推理,从而在复杂地视觉问答任务中提高准确性,展示了
符号推理与深度学习结合的潜力,推动了神经符号方法在视觉问答中的应用。
Pan\cite{pan2023logic}等人对神经符号方法进行了改进,引入了一个自我优化模块,利用符号求解器的错误提示信息来修正符号表示,从而提高推理的准确性和可靠性。

\subsection{视觉问答中复杂空间推理的难点}
空间推理作为视觉问答系统所需的重点核心能力,涉及对图像中物体间拓扑关系、方位、尺寸等多种空间信息的理解、建模和推理。然而,
现有方法在空间推理中仍面临诸多挑战。主要表现在以下几个方面:

\subsubsection{复杂空间关系的建模与表示}
空间推理需要处理多层次关系,如拓扑关系(包含、相邻)、方位(左/右、前/后)、动态轨迹(移动路径)等。
传统方法依赖预定义逻辑规则(如RCC-8拓扑模型),但难以适应开放域场景的多样性\cite{li2021algorithm}。
例如,自然语言中的“靠近”这个词汇,缺乏量化阈值,难以定量界定两个物体之间的距离在什么场景下为“靠近”这一关系,导致符号逻辑无法精确映射\cite{shrestha2019answer}。
另外,基于注意力机制的模型虽然能定位目标区域,但难以捕捉长距离或者隐含的空间关联,进而导致错误定位。

\subsubsection{多模态对齐与语义鸿沟}
视觉与文本模态的语义对齐,是VQA模型能够顺利进行空间推理的基础。但是存在跨模态特征映射的困难以及稀疏空间关系的问题。
目前,有一些研究者试图解决这些问题。在跨模态特征映射这一问题上,Costanzino\cite{Costanzino2024MultimodalIA}等人提出了一种使用轻量级多层感知器(MLPs)的方法,训练两个映射函数,基于正常样本预测一个模态的特征,通过比较实际特征和预测特征的不一致性
来检测工业场景中的异常。在解决稀疏空间关系的问题方面,Zhang\cite{wu2024minds}等人提出了VoT模型,通过VoT提示来提升LLMs的空间推理能力,通过生成视觉表示增强了模型对。
这些方案在一定程度上解决了多模态对齐和语义鸿沟问题,但仍存在一定局限性。具体而言,多数模型通过全局或局部注意力加权融合多模态特征,但未显式建模空间关系的层次性。
另外,一些模型通过子任务分解实现推理,但对空间逻辑的组合泛化能力有限。

\subsubsection{动态场景与实时推理的挑战}
动态场景(如移动物体的避障路径规划)要求模型实时更新空间状态,但现有方法存在增量式推理不足和数据驱动的固有局限。具体而言,
现有方法往往依赖于离线训练的模型,无法实时更新空间状态,导致推理结果复杂空间推理任务滞后,无法满足实时推理的需求。例如,符号推理(如ASP)需重新求解完整的逻辑程序,导致延迟较高\cite{}。
另外像一些基于监督学习的模型,依赖于静态数据集(如CLEVR),无法适应动态环境中的连续变化。最新的一些研究成果,如微软提出的MVoT框架通过生成可视化中间推理步骤,
实现了结合文本与图像信息背景下,对空间关系表示的动态调整,且在复杂场景中比传统思维链(Chain of Thought, CoT)的稳健性提升20\%。

\subsubsection{数据集偏见与评估瓶颈}
现有VQA数据集(如CLEVR、VQAv2)存在显著偏差,如问题答案分布不均、问题类型单一等,导致模型在特定问题上表现优异,但在真实场景下泛化能力不足。
Shrestha\cite{shrestha2019answer}等人指出,像CLEVR这一类合成的数据集,虽然能够测试多步逻辑,但其几何简单性无法反映真实世界的复杂性,并且在问题设计上
也隐含对特定答案的倾向性,如“左侧”常与特定物体绑定,进而导致模型依赖表面统计规律,而并非真实推理。目前,已有一些研究者使用零样本学习的方法,通过引入
从未见过的问题-答案组合,测试模型的泛化能力。例如,Yang\cite{yang-etal-2022-zero}等人提出了一种零样本学习的方法,通过引入从未见过的问题-答案组合,测试模型的泛化能力。
此外,新的一些VQA数据集如VSR,通过控制答案分布来减少语言先验,以测试模型的纯视觉推理能力。

\subsubsection{可解释性与鲁棒性不足}
空间推理需要透明化的推理过程,以支持安全验证和决策解释。然而,现有的方法存在黑箱和对抗脆弱性等问题,无法提供可解释性保障。
目前,有一些研究者试图在这一方面进行改进,如通过可视化路径的方式,提高模型的可解释性。
例如,Li\cite{li2025imagine}等人提出了多模态思维可视化框架(MVoT),旨在通过生成推理轨迹的图像可视化,增强多模态大语言模型(MLLMs)在复杂空间推理任务中的表现。
该方法通过在自回归 MLLMs 中引入标记差异损失,显著提高了视觉连贯性和保真度。实验结果表明,MVoT在多个任务中表现出的性能出色,尤其是在思维链(Chain of Thought,CoT)
表现很差的场景中,展现出了显著的改进。Shah\cite{shah2019cycle}等人提出了一种基于循环一致性的训练框架,旨在增强VQA模型对语言变化的鲁棒性。
该方法通过双向训练和循环一致性,提高了模型对语言变化的适应性,从而提高了模型的鲁棒性和泛化能力。

\section{研究目标与内容}
本课题的主要研究目标是研究并设计一种新的神经符号框架,通过实验,证明该框架能够有效提升大语言模型在复杂空间推理问题上的性能。
最终将该神经符号框架接入视觉问答系统。

本文的主要研究内容包括以下三个方面:

\begin{enumerate}[label=(\arabic*),itemsep=0pt,parsep=0pt]
    \item 在参考借鉴CLEVR数据集的基础上,在图像中的物体交互、图像遮挡和部分可见性、图像噪声干扰等方面进行扩展,
构建一个新的视觉问答数据集,以更好地评估视觉问答模型在解决复杂空间推理问题上的能力
    \item 研究融合大语言模型及ASP的神经符号流水线,并使用DSPy来进行实现。通过DSPy,将ASP求解器与大语言模型实现系统集成。
在本文构建的视觉问答数据集上,进行对比实验验证。此外,为了评估,还进行了消融实验。
    \item 以本文中所提出的神经符号框架为核心,设计并实现一个视觉问答系统。
\end{enumerate}

\section{研究方法与技术路线}
本文针对以上研究目标和研究内容,综合多种方法进行研究。本文的研究主要涉及三个重点内容:
\begin{enumerate}[label=(\arabic*),itemsep=0pt,parsep=0pt]
    \item 对于视觉问答数据集的构建,首先,用案例分析法研究CLEVR数据集,分析其特点和不足,为新数据集的构建提供理论依据和设计方向。
然后采用文献研究法,调研现有的视觉问答数据集,学习它们在数据集构造过程中的方法和策略。
    \item 对于面向空间推理领域的神经符号框架的研究,首先,分析现有神经符号方法在空间推理方面的优势和不足。
其次,进行模型构建。最后,通过实验验证,进行对比实验和消融实验,评估设计的神经符号框架的性能。
    \item 对于视觉问答系统的设计与实现,首先,进行需求分析,明确系统的功能和性能要求。其次,设计视觉问答系统的架构和模块划分,确定各模块实现所需的技术方案。
再次,基于前文的研究成果,实现视觉问答系统的各个模块,并进行系统集成和测试。最后,通过实验验证,评估视觉问答系统的性能和可用性。
\end{enumerate}

具体的技术路线如图\ref{roadmap}所示。

\begin{figure}
    \centering
    \includegraphics[width=\textwidth]{process.png}
    \caption{技术路线图\label{roadmap}}
\end{figure}

\section{论文结构}
本文共分为六个章节,各章节的主要内容具体如下:

第一章为绪论,总体介绍本文的研究背景及意义、相关研究现状与不足、研究目标
与研究内容、研究方法与技术路线及本文的结构安排。

第二章为背景知识,对本文涉及到的主要技术进行介绍,具体包括ASP程序语法及ASP求解器、
GLIP以及DSPy。

第三章为数据集构建。对本文所用的视觉问答数据集进行详细介绍,包括数据集的构造目的、数据集的研究方向、数据集的设计流程以及对数据集质量的验证。

第四章为神经符号框架的研究设计。对本文设计的神经符号框架进行详细介绍,包括流水线总体架构、视觉场景理解、语义解析、知识蒸馏、迭代反馈和规则修正、ASP推理等模块的设计。

第五章为实验及结果分析。通过一系列实验,验证本文所提出的神经符号框架对复杂空间推理任务的提升效果,以及其在不同大语言模型架构上的泛化能力。

第六章中对本文工作加以总结,分析本文的创新点和不足之处,并对未来的研究方向进行展望。


\newtheorem{definition}{定义}[section]
\newtheorem{example}{例}[section]
\chapter{背景知识}
为深入研究积木世界VQA中的空间推理问题,理解和掌握相关基础技术至关重要。本章将围绕本研究所依赖的核心技术
和数据集进行介绍,旨在为后续正式开展研究提供理论支撑与技术基础。
首先介绍 CLEVR 数据集,它是空间推理领域的经典数据集,
具有明确的物体属性、空间关系标签和程序化生成机制,其设计理念与积木世界高度契合,
是构建符合本研究的部分可见VQA数据集的重要参考依据。
其次,介绍回答集编程(Answer Set Programming, ASP),
它是一种基于非单调逻辑的知识表示与推理方法,具备高度表达能力和高效求解性能,
广泛应用于神经符号系统。
接着,介绍视觉语言预训练模型 GLIP,它能够实现图文联合理解和目标定位,为视觉信息的结构化提供强有力支持。
随后,将介绍近年来发展迅速的大语言模型(Large Language Models, LLMs),
特别是在规则生成与补全任务中的应用价值。
最后,介绍 DSPy 框架,其作为一种以程序化方式组织和优化大语言模型行为的开发工具,
在本研究中被用于规则蒸馏、迭代反馈与逻辑修正等需要对LLM进行微调的环节,
从而简化神经符号VQA框架的开发流程。
\section{CLEVR数据集}
CLEVR(Compositional Language and Elementary Visual Reasoning)数据集是由 Johnson 等人于 2017 年
提出的一个用于评估视觉问答模型综合推理能力的合成数据集\cite{johnson2017clevr},
其设计初衷在于提供一个结构清晰、控制变量明确、便于自动标注的实验环境,
以有效分析VQA系统在多步推理、空间关系理解、属性比较与条件组合等方面的能力。
CLEVR 中的图像由多个具有不同几何形状、颜色、大小和材质的三维物体构成,生成方式基于程序控制的场景构建器,
从而保证了图像的可解释性与数据标签的完备性。问题与答案对则通过模板化方法自动生成,
每个问题背后对应一段形式化的程序,用于精确刻画其推理路径与中间步骤。

从视觉场景的构造逻辑来看,CLEVR 的设计理念与传统人工智能研究中的“积木世界”(\-Blocks World)高度一致,
二者均基于三维空间中多个物体的组合关系,并围绕空间布局、属性识别与对象间关系开展推理任务。
在 CLEVR 中,诸如“球体位于立方体的左侧”“红色物体后面有一个金属物体”这类问题与积木世界中的经典空间关系判断任务
在本质上具有相同的语义结构,因此,CLEVR 可被视为积木世界的一种程序化扩展与抽象建模,
其在空间推理表达的广度和精度方面提供了良好的基础支撑。由于其合成属性的可控性与逻辑严谨性,
CLEVR 被广泛用于神经符号推理模型的验证实验,成为空间推理型 VQA 任务的基准数据集之一。

然而,CLEVR数据集存在一些局限,具体表现在以下三方面:
\begin{enumerate}[itemsep=0pt,parsep=0pt]
\item CLEVR中的场景完全可见。CLEVR数据集的图像中的场景都是完整的,即该图像对应的问题的全部所需信息均可从图像中获得,不需要其它外部的额外知识,
进而模型无需执行涉及隐藏信息的推理。场景完全可见的样例见图\ref{observable},其对应的自然语言问题是:“What color is the cube to the right of the small red metal cube?”。
由于该问题所涉及的所有物体在图\ref{observable}中完全可见,并且所有物体的各项属性(颜色、形状、大小、材质)非常清晰。
此外,问题中涉及的参照对象(小红色金属立方体)和目标对象(其右边的立方体)都在图中可以明确识别。
然而,在现实生活中部分可见性处处可见,例如扫地机器人在执行任务过程中的视角极其有限,
可能当前视野的图像中缺少判断前方是否存在障碍物的信息,此时需要一些关于该场景的额外知识,例如障碍物的图形学特征
等常识,辅助进行推理。
\item CLEVR缺乏显式表达背景知识或先验约束的机制。尽管CLEVR数据集旨在测试组合推理能力和基本视觉推理能力,
但其仅提供图像及图像中的对象属性,并不要求模型在回答问题时使用当前场景之外的、与当前场景相关的背景知识或逻辑约束。
在模型对CLEVR中的问题进行推理的过程中,模型主要依据对图像的直接理解和内部逻辑进行推理。仍以图\ref{observable}为例,
由于所有信息完全可见,故在该问题的推理过程中,模型不需要猜测、补全或者使用外部知识,
显然,真正的人工智能应该能够模仿人类的思考过程,根据问题所处的场景,灵活选取已有的相关知识,结合图像场景
的约束和问题的要求,进行推理。
\item CLEVR中所有问题通常设计为唯一答案,未考虑部分可见场景这种不确定性条件之下的多解性问题。
在部分可见场景中,由于不确定性的存在,可能会有多个满足条件的物体或关系,导致问题的答案不唯一。
\end{enumerate}
以上三点局限导致模型在该数据集上获得的推理性能难以泛化至更具挑战性的实际应用场景,
尤其是在积木世界这类典型结构化环境中,部分可见性对问题求解结果具有显著影响。因此,
尽管 CLEVR 为空间推理的研究奠定了坚实基础,但仍需进一步构建具有部分可见性的VQA数据集,
以更真实地模拟积木世界中存在的不完全视觉信息问题。
\begin{figure}[h]
    \centering
    \includegraphics[scale=0.2]{figures/observable.png}
    \caption{CLEVR数据集中场景完全可见的样例}
    \label{observable}
\end{figure}
基于上述考虑,本文在 CLEVR 的设计框架基础上,提出并构建了适用于部分可见积木世界场景的 VQA 数据集 POVQAD,
为数据集提供了部分可见性,并以此为实验平台,
进一步探索结合神经网络与符号推理的神经符号VQA框架。这一研究思路不仅继承了 CLEVR 的可控性与程序化优势,
同时将研究重心扩展至更贴近实际认知过程的复杂场景,对提升视觉问答系统在真实环境中的空间推理能力具有重要意义。
\section{回答集编程}
% \numberwithin{equation}{section}
\subsection{语法}
项(terms)是ASP程序中最基本的元素,项可以是常量、变量或者函数。常量以符号常量或者数字表示(如4、a)。变量以字符串来表示,
要求首字母必须大写(如Dog、Person)。函数由函数符号与若干参数构成,例如$f(t_1,...,t_n),(n \geq 0)$。各参数
$t_i,(i=1,2,...,n)$也是项。若项中不存在函数符号与变量,则称该项为实例化项,否则称该项为非实例化项。

原子(atom)由谓词和项共同组成,例如$q(t_1,...,t_n),n \geq 0$。其中,$q$为$n$元谓词的标识符,$t_1,...,t_n$为项,
$n$为0时,谓词之后的括号可以省略。原子用于表示不同项之间的关系,例如:$dog(animal)$、$family(jensen,lucy)$、
$tomorrow\_sunny$,分别表示狗是动物、jensen和lucy是家人以及明天晴天。如果原子中的每一项都是实例化的,则该原子是实例化
原子,否则,该原子是非实例化的原子。例如,$dog(animal)$是实例化原子,$family(jensen,lucy)$是非实例化原子。
原子$q(t_1,...,t_n)$的强否定形式为$\urcorner q(t_1,...,t_)n$,其中符号$\urcorner$是经典逻辑中的否定,
$\urcorner q(t_1,...,t_n)$表示各项$t_i(1 \leq i \leq n)$。不符合
$q$所描述的关系,$\urcorner \urcorner a = a$,与经典逻辑中的排中律相同。文字(literal)可以是原子
$q(t_1,...,t_n)$,也可以是原子的强否定形式$\urcorner q(t_1,...,t_n)$。若文字对应的原子是实例化的,则称
该文字是实例化的。

\begin{definition}[规则] 规则$r$是具有如下形式的式子:
    \begin{equation}
        l_0 \thinspace or ... \thinspace or \thinspace l_m \leftarrow l_{m+1},...,l_n \thinspace not \thinspace l_{n+1},...,\thinspace not \thinspace l_p.
    \end{equation}
\end{definition}

其中$p \ge n \ge m \geq 0, l_i$代表文字,$not$为新的逻辑连接符,通常称作缺省否定符(default negation)或者
失败即否定(Negation As Failure, NAF),又称为若否定,$not \thinspace l_i$被读作“没有理由相信$l_i$为真”,但是这并不表明
$l_i$为假。对ASP程序而言,$not not a$不与a等价。规则r读作“如果相信$l_{m+1},...,l_n$为真,并且没有理由相信
$l_{m+1},...,l_p$为真,则相信$l_0 \thinspace or \thinspace ... \thinspace l_m$为真”。将$not l_i$称为缺省
文字(default literal),文字与缺省文字共同组成拓展文字(extended literal)。

规则$r$左侧的文字称为头部,可以表示为$head(r) = \{ l_0,...,l_m \}$。规则$r$右侧的文字称为体部,可以表示为
$body(r) = \{ l_{m+1},...l_n,not \thinspace l_{n+1},..., not \thinspace l_p \}$,规则的体部可以分为正
体部与负体部,正体部表示为$body^+(r) = \{ l_{m+1},...,l_n \}$,负体部表示为$body^-(r)=\{ l_{n+1},...,l_p \}$
。因此,规则$r$可以表示为:
\begin{equation}
    head(r) \thinspace \leftarrow \thinspace body^+(r), \thinspace not body^-(r) \label{con:body}
\end{equation}
一个ASP程序是有限条\eqref{con:body}所示的规则组成的集合。根据规则各部分所满足的相关条
件,可以分别定义事实、约束、正规规则、缺省规则和严格规则,如定义\eqref{def:fact}-定义\eqref{def:definite_rule}所
示。

\begin{definition}[事实]
当规则$r$满足$head(r)\neq \emptyset, body(r) = \emptyset$时,被称为事实(fact)。\label{def:fact}
\end{definition}
\begin{definition}[约束]
当规则$r$满足$head(r)= \emptyset, body(r) \neq \emptyset$时,被称为约束(constraint)。
\end{definition}
\begin{definition}[正则规则]
    当规则$r$满足$|head(r)|=1$时,被称为正则规则(normal rule)。
\end{definition}
\begin{definition}[缺省规则]
    当规则$r$满足$body(r)^-\neq \emptyset$时,被称为缺省规则(default rule)。
\end{definition}
\begin{definition}[严格规则]
    当规则$r$满足$body(r)^- = \emptyset$时,被称为严格规则(definite rule)。\label{def:definite_rule}
\end{definition}
根据上述各类规则的定义,本文进一步定义简单逻辑程序、缺省逻辑程序和正规逻辑程序如下。
\begin{definition}[简单逻辑程序]
    若程序$P$由有限条严格规则组成,则$P$被称为简单逻辑程序。
\end{definition}
\begin{definition}[拓展逻辑程序]
    若程序$P$由有限条缺省规则组成,则$P$被称为拓展逻辑程序。
\end{definition}
\begin{definition}[正则逻辑程序]
    若程序$P$除约束外,由有限条正则规则组成,则$P$被称为正则逻辑程序(Normal Logic Program, NLP)。
\end{definition}
若不做特别说明,本文考虑的程序均为不含约束的正规逻辑程序,即程序中的每条规则头部有且仅有一个文字。
\subsection{语义}
\subsubsection{文字的可满足性语义解释和程序的模型}
作为基础理论,首先给出 Herbrand 域、Herbrand 基的定义,分别如定义2.10和2.11所
示,进一步定义规则与程序的实例化,最终在实例化程序下讨论文字的可满足性语义解
释和可满足性。
\begin{definition}[Herbrand基]
    $P$是一个 ASP 程序,将程序$P$中出现的常量和函数形成的
    所有不含变量的项的集合称为 Herbrand 域,使用$\mathcal{U}_P$表示。
\end{definition}
\begin{definition}[Herbrand域]
    $P$是一个 ASP 程序,出现在程序$P$的规则中的谓词符号与
$\mathcal{U}_P$中的项组成的所有可能的不含变量的原子集合称为程序$P$的 Herbrand 基,使用$\mathcal{A}_P$
表示,为了保持简洁,通常简写为$\mathcal{A}$。
\end{definition}
\begin{definition}[规则的实例化]
    给定$P$中的一个规则$r$,使用$ground(r)$表示通过使用
$\mathcal{A}_P$中的常量对应地替换$r$中变量而获得的规则集合,这一映射被称为规则的实例化。
\end{definition}
\begin{definition}[程序的实例化]
    程序$P$由一系列规则的集合组成,程序$P$的实例化$ground(P)$表示为式\eqref{equation:ground}:
    \begin{equation}
    ground(P)=\bigcup_{r \in P}ground(r) \label{equation:ground}
    \end{equation}
\end{definition}
\begin{definition}[文字的可满足性语义解释]
    定义文字的可满足性语义解释$I=\langle I^+,I^- \rangle$,其中$I^+\cup I^-\subseteq \mathcal{A}_P$
    ($P$是实例化程序),$I^+\cap I^-=\emptyset$。$I^+$表示为已知为真的文字集合,$I^-$表示已知为假
    的文字集合。当$I^+\cap I^-=\mathcal{A}_P$时,$I$被称作完全可满足性语义解释(complete interpretation)。
    若I不是完全可满足性语义解释,则$I$中存在真值未确定(undefined)的文字。
\end{definition}
\begin{definition}[程序的模型]
    假设$P$是一个实例化的回答集程序,$I=\langle I^+,I^- \rangle$是一个可满足性语义解释。若文字$l \in I^+$,
    则称I满足$l$,记作$I\models l$;对缺省文字$not l$,若$l \in I^-$,则称I满足$not l$,记作$I\models not l$。
    对于文字集合$S$,若$\veebar l \in S,I \models l$,则称$I$满足集合$S$,记作$I \models S$,当一个规则
    $r$满足$I \nvDash body(r)$或$I \models head(r)$时,$I$满足规则$r$。当$I$满足程序$P$的所有规则时,称
    $I$是$P$的一个模型。
\end{definition}
\subsubsection{回答集语义}
\begin{definition}[一致性文字集合]
    给定一个文字集合$S$,若$S$中不同时包含$l$和$\urcorner l$,则集合S是一致性文字集合,其中,$l \in S$是集合
    S中的任意文字。
\end{definition}
\begin{definition}[可满足性,Satisfiability]
    给定一致性文字集合$S$,文字$lit$与规则 $r$,若
$lit \in S$,则文字$ lit $满足集合 $S$。若 $head(r)\cap S \neq \emptyset$,则 $S $满足头部。若$ body^+(r) \subseteq 
S, S ∩ body^−(r) = \emptyset$,则集合$ S $满足体部。若 $S $满足规则 $r$ 的头部或者 $S $不满足规则$ r$ 的
体部,则$ S$满足规则 $r$,记作$ S \models r$。给定一个 ASP 程序 $P$,若 $\veebar r \in P, S \models r$,则 $S$ 满足
程序 $P$。
\end{definition}
\begin{definition}[规则的适用与阻塞]
    给定一致性文字集合 $S$ 与规则 $r$,$S$ 满足 $body(r)$,
则称规则 $r $在集合$S $下是适用的(applicable),否则称规则$ r $在集合$ S$ 下被阻塞(block)。
\end{definition}
\begin{example}
    给定ASP程序$P_3$,该程序包含两条规则:
    \begin{align*}
        &r_1: p \leftarrow q, s. \\ 
        &r_2: s \leftarrow t.
    \end{align*}
\end{example}
下面通过分析说明集合$S=\{ p, q\}$对程序$P$的可满足性。
\begin{enumerate}[itemsep=0pt,parsep=0pt]
    \item 因为$p\in S$,所以文字$p$与文字$q$均满足$S$;
    \item 因为$S \cap head(r_1) = \{ p \} \neq \emptyset$,所以$S$满足$head(r_1), S \models r_1$;
    \item 因为$body^+(r_2) \subsetneq S$,所以$S$不满足$body(r_2)$,所以$S \models r_2$;
    \item 因为$S$满足程序$P$中的每一条规则,所以$S$满足程序$P$。
\end{enumerate}
\begin{definition}[Gelfond-Lifshitz规则(GL规约)]
    假设程序 $P$ 是给定的实例化程序,$S$
是一致性文字集合,$l$ 是文字,$S \subseteq  Lit(P), l \in Lit(P)$, $P $关于 $S$ 的$ GL$ 规约结果$ P^S$定义为式\eqref{equation:GL}:
\begin{equation}
    P^S = \{ head(r) \leftarrow body^+(r) \mid r \in P, body^-(r) \cap S = \emptyset \} \label{equation:GL}
\end{equation}
\end{definition}
\begin{definition}[简单逻辑程序回答集\cite{}]
    假设$P$是简单逻辑程序,程序$P$的回答集是$S \subseteq Lit(P)$ 满足:

    \begin{enumerate}[itemsep=0pt,parsep=0pt]
        \item 对于程序$P$中的每一条规则$l_0 \leftarrow l_1,...,l_m$,如果$l_1,...,l_m \in S$,那么$l_0 \in S$。
        \item $S \subseteq Lit(P)$且不存在$ S$ 的任何子集也满足程序 $P$ 的每一个规则,即 $S $是满足程序$ P$的最小集合;
        \item 如果 $S$ 包含互补的文字,那么 $S = Lit(P)$;
    \end{enumerate}
\end{definition}
\begin{definition}[拓展逻辑程序回答集]
    假设$P$是包含缺省否定的扩展逻辑程序,$S$是实例化文字集合,如果$S$是$P^S$的回答集,则$S$是$P$的回答集。如果程序$P$没有回答集或
者只有一个包含互补文字的回答集$Lit(P)$,那么程序$P$是不一致程序,否则,程序$P$是一致性程序。

可以看出,使用 GL 规约,可以将扩展逻辑程序程序转换为简单逻辑程序,从而得到扩展逻辑程序的回答集。下面通过一个例子说明。
\begin{example}[GL规约与问答集]
    给定一个实例化的ASP程序$P_4$:
    \begin{align*}
        &r_1: p(a) \leftarrow not p(b) \\
        &r_2: p(b) \leftarrow not p(a) \\
        &r_3: \leftarrow p(b).
    \end{align*}
    程序$P_4$可能的回答集有四个:$S_1 = \emptyset, S_2 = \{ p(a) \}, S_3 = \{ p(b) \}, S_4 = 
    \{ p(a), p(b) \}$, 根据回答集的定义依次验证:
    \begin{enumerate}[itemsep=0pt,parsep=0pt]
        \item $P^{S_1} = \{ p(a) \leftarrow .\} \cup \{ p(b) \leftarrow . \} \cup \{ \leftarrow p(b) .\}$,事实$(p(b) \leftarrow .)$
    与约束$(\leftarrow p(b).)$同时在$P^{S_1}$中出现,因此$S_1$不是$P^{S_1}$的回答集,所以$S_1$不是$P$的回答集。
        \item $P^{S_2} = \{ p(a) \leftarrow .\} \cup \{ \leftarrow p(b) .\}$,而$S_2 \models ((p(a) \leftarrow .))$且
    $S_2 \models (\leftarrow p(b).)$,因此$S_2 \models P^{S_2}$,且不存在$S_2^{'}\subset S_2, S_2^{'} \models P^{S_2}$,所以$S_2$是$P^{S_2}$的回答集;

        \item $P^{S_3} = \{ p(b) \leftarrow . \} \cup \{ \leftarrow p(b) .\}$,事实$(p(b) \leftarrow .)$
    与约束$(\leftarrow p(b).)$同时在$P^{S_3}$中出现,因此$S_3$不是$P^{S_3}$的回答集,所以$S_3$不是$P$的回答集;
        \item $P^{S_4} = \{ \leftarrow p(b). \}$,可知$P^{S_4}$的回答集为$\emptyset$,因此$S_4$不是
    $P^{S_4}$的回答集,所以$S_4$不是$P$的回答集。
    \end{enumerate}
\end{example}
\end{definition}

\subsubsection{well-founded语义}
上世纪 90 年代,良基语义(well-founded semantics)的概念由 Van Gelder A 提出,
对于任意的回答集程序,都存在对应的良基模型(well-founded models),并且在多项式
时间内可以计算出这个模型\cite{van1991well}。该语义模型是本文对 NAF 文字进行解释和对不一致程
序原因进行分类的重要基础,下面介绍这一模型的定义及计算方法。

\begin{definition}[立即结论]
$P$是一个 ASP 程序,$S\subseteq \mathcal{A}_P,V\subseteq \mathcal{A}_P$,则集合$S$关于$P$和$V$的
立即结论(immediate consequence)记作 $\mathcal{T}_{P,V}(S)$,定义如式\eqref{equation:immediate_consequence}所示:
\begin{equation}
    T_{P,V}(S) = \{ a | \exists r \in P, head(r) = a, body^+(r) \subseteq S, body^-(r)\cap V = \emptyset \} \label{equation:immediate_consequence}
\end{equation}
\end{definition}
考察上式可以发现,当集合$V$确定时,关于集合$S$的函数$T_{P,V}(S)$是单调的。因此,当集合$V$固定时,该函数
存在最小不动点(Least Fixed Point, LFP),使用$lfp(.)$表示。

\begin{definition}[良基模型]
    $P$是一个 ASP 程序,$P^+$是程序$P$中的全部严格规则的集合,序列$(K_i,U_i)_{i \geq 0}$定义如下:
    \begin{align*}
        &K_0 = \mathit{lfp}(T_{P^+}) & K_i &= \mathit{lfp}(T_P, U_{i-1}) \\
        &U_0 = \mathit{lfp}(T_{P, K_0}) & U_i &= \mathit{lfp}(T_{P, K_i})
    \end{align*}
\end{definition}
若$\langle K_j,U_j \rangle = \langle K_{j+1},U_{j+1} \rangle$,且不存在$0 \leq k < j$,使得
$\langle K_k, U_k = \langle K_{k+1},U_{k+1} \rangle$(j,k是正整数),则$P$的良基模型$WF_P=\langle W^+,W^- \rangle$,
满足$W^+=K_j$,$W^-=\mathcal{A}\backslash U_j$
\begin{example}[良基模型]
    程序$P_5$包含以下四条规则:
    \begin{align*}
        &r_1:q \leftarrow , not p. &r_2: p \leftarrow, not q. \\
        &r_3:a \leftarrow b. &r_4: b \leftarrow. 
    \end{align*}
\end{example}
首先得到$P_5^+=\{ a \leftarrow b \} \cup \{b \leftarrow . \}$,而$lfp(T_{P_5^+}) = {a, b}$,因此
$K_0 = {a,b}$;进一步计算得到$U_0 = lfp(T_{P_5^+, K_0}) = \{ a,b,p,q \}$,$K_1 = lfp(T_{P_5^+, U_0}) = 
\{ a, b \} = K_0$,$U_1 = lfp(T_{P_5^+,K_1}) = \{ a,b,p,q \} = U_0$。因此$\langle K_0, U_0 \rangle = 
\langle K_1, U_1 \rangle$且不存在$0 \leq k < 1$,$\langle K_k, U_k \rangle = \langle K_{k+1}, U_{k+1} \rangle$。
因此程序$P_5^+$的良基模型为$WF_{P_5^+} = \langle \{a, b\}, \{ \} \rangle$,$p$和$q$在良基模型中属于
未定义($undefined$)。
\subsection{ASP求解器}
ASP求解器是实现ASP语义推理的核心工具,其工作原理是:通过搜索逻辑程序的最小模型(即回答集)来解决复杂的组合优化问题。

ASP求解器的工作流程可以划分为两个阶段:基础化(Grounding)和求解(Solving)。
基础化是将一阶逻辑程序实例化为命题逻辑形式,而求解则是基于CDCL算法搜索满足约束的模型。

截至目前,已经有众多ASP求解器。学术界和工业界所用的主流的求解器,可分为两类:单阶段求解器(如Clasp、DLV)、
多阶段求解器(如Clingo)。两类求解器的主要区别在于,单阶段求解器独立地进行基础化与求解,而多阶段求解器
则是集成了基础化与求解的交互式系统。

表\ref{tab:solver_comparison}对目前的主流ASP求解器的适用场景及核心优势进行了对比。因为 Clingo 求解器
具有免费开源、使用简单、运行高效的特点,本文在求解 ASP程序时,使用 Clingo 进行相关实验。
\begin{table}[h]
    \centering
    \renewcommand{\arraystretch}{1.2}
    \begin{tabular}{l l l l}
        \toprule
        \textbf{求解器} & \textbf{核心技术} & \textbf{适用场景} & \textbf{核心优势} \\
        \midrule
        \multirow{2}{*}{Clasp} & 多线程CDCL & 大规模组合优化 & 支持并行搜索 \\
                               & 情性子句学习 & 工业级应用 & 性能竞赛冠军 \\
        \midrule
        \multirow{2}{*}{Clingo} & 增量式基础化 & 动态规则生成 & 集成Python API \\
                                & 多阶段编程 & 交互式推理 & 支持在线更新程序 \\
        \midrule
        \multirow{2}{*}{DLV} & 数据库优化 & 知识表示型问题 & 高效处理复杂规则 \\
                             & 存在线量化推理 & 语义Web & 支持非确定域 \\
        \midrule
        \multirow{2}{*}{WASP} & 分层弱约束优化 & 偏好推理 & 加权约束高效处理 \\
                              & 冲突驱动剪枝技术 & 多目标优化 & \\
        \midrule
        \multirow{2}{*}{Smodels} & 稳定模型算法 & 教学与研究 & 算法透明易扩展 \\
                                 & 部分求值 & 基础模型验证 & \\
        \midrule
        LPARSE & 基础化预处理 & 规则实例优化 & 与Smodels配合使用 \\
        \midrule
        \multirow{2}{*}{IDP} & 基于模型的推理 & 复杂知识库管理 & 支持类型系统 \\
                             & 扩展一阶逻辑 & & 高阶推理 \\
        \midrule
        Alpha & 并行求解引擎 & 超大规模问题 & GPU加速搜索 \\
        \bottomrule
    \end{tabular}
    \caption{扩展后的系统对比}
    \label{tab:solver_comparison}
\end{table}
\section{GLIP}
在积木世界VQA任务中,系统需要准确识别图像中的各类几何体、理解其属性(如颜色、材质、形状)、
定位其空间位置,并将这些视觉事实转化为结构化的符号信息,供符号推理模块进行逻辑推演。
因此,目标检测模块不仅需要具备高精度的检测能力,还应能支持丰富灵活的语言描述,以适应多样化的问答形式。
在本研究构建的神经符号VQA系统中,GLIP(Grounded Language-Image Pretraining)被选为核心的视觉解析组件,
其通用性与语言驱动检测能力为后续的空间关系建模提供了关键支持。

GLIP 是一种基于图文对齐预训练的目标检测框架,旨在统一视觉识别与自然语言理解任务。
与传统检测模型如 Faster R-CNN、YOLO 系列或 DETR 不同,这些模型依赖于封闭类别集合进行训练和推理,
难以适配于用户自定义或开放式的语言查询。而 GLIP 的创新点在于引入 语言提示(language prompts) 
作为查询机制,允许用户以任意自然语言短语描述待检测目标,
从而实现开放词汇(open-vocabulary)目标检测\cite{li2022grounded}。
GLIP 的这一特性尤为适合积木世界VQA场景中涉及“最右边的紫色圆柱体”“靠近红球体的黄色立方体”等组合性语言
表达的识别需求。

GLIP的架构如图\ref{GLIP_architecture}所示,从架构上看,GLIP 借鉴了 DETR 的端到端检测思想,整体由以下几个模块构成:
\begin{enumerate}[itemsep=0pt,parsep=0pt]
\item 视觉编码器:通常为ViT(Vision Transformer)或ResNet,将图像编码为空间维度保留的特征张量;
\item 文本编码器:采用 BERT 或类似的 Transformer 模型,将输入语言提示转换为语义嵌入;
\item 跨模态交互模块:引入 Cross-Attention 机制,实现图文特征的融合对齐;
\item 检测头(Detection Head):利用融合后的特征对每个语言提示生成目标框位置与匹配分数。
\end{enumerate}
\begin{figure}[h]
    \centering
    \includegraphics[width=\textwidth]{GLIP_architecture.jpg}
    \caption{GLIP结构示意图\label{GLIP_architecture}}
\end{figure}

该模型在训练阶段采用了包含图像-文本对的大规模预训练语料(例如 Visual Genome 和 Objects365 等),
使得其具备强大的跨模态语义匹配能力。实验结果表明,GLIP 在开放词汇目标检测任务中的平均精度(mAP)
显著优于传统检测框架,并在下游视觉问答、图文匹配等任务中展现出出色的迁移性能\cite{li2022grounded}。

在本研究中,GLIP 被用于将积木世界图像中的物体及其属性结构化为三元组形式,
如 object(sphere, red, leftmost),这些三元组作为输入供 ASP 推理
模块进行空间关系推理。与基于固定类标的目标检测方法相比,GLIP 的语言可控性显著降低了对类别词表的依赖,
使得系统具备更强的扩展性与灵活性,尤其适应于问答任务中动态变化的问题语境。

综上,GLIP 凭借其开放词汇检测能力、图文联合建模机制以及良好的通用性,
成为本研究神经符号VQA框架中的关键视觉组件。它不仅提升了积木世界图像结构化建模的准确性,
还显著增强了系统对复杂语言问句的适应能力,为构建高效、可解释的空间推理VQA系统提供了坚实基础。
\section{大语言模型}
在本文提出的神经符号VQA框架中,大语言模型(Large Language Models, LLMs)在多个关键模块中发挥了核心作用。
我们主要采用了 ChatGPT、LLaMA 和 DeepSeek 三种主流模型。
它们各自具备不同的特点和优势,以下将对这三种主流模型展开介绍。
\subsection{GPT}
目前,GPT系列最新的模型为GPT-4o,其在架构方面采用了高达1.8T参数的多模态Transformer,引入联合注意力机制,支持文本-图像-语音跨模态联合推理。
此外,GPT采用了动态窗口扩展技术,长文本生成连贯性指标提升37\%。

训练策略上,采用了强化学习对齐(RLHF)​,通过3阶段训练(SFT → RM → PPO),有害内容生成率降至0.3\%。
并且采用了​数据蒸馏的方案,其使用合成数据占比达45\%,有效减少了网络噪声数据污染。

GPT目前在​多模态理解方面表现优异,图像描述任务BLEU-4达0.74,视频内容摘要准确率91\%。
然而GPT训练成本高昂,价格较高,每百万token费用7.5美元,是DeepSeek的53倍。
\subsection{LLaMA}
LLaMA是由Meta开源的LLM,其在架构上采用​分组查询注意力(GQA),​键值头数量压缩至查询头的1/4,70B模型推理显存需求降至48GB。
此外,还采用了​RoPE位置编码,动态旋转嵌入技术提升长文本连贯性,32k上下文下困惑度降低12\%。

训练策略上,LLaMA​采用了课程学习策略,分阶段增加数据复杂度,代码生成任务准确率提升19\%。
此外,LLaMA还支持​量化部署,4bit量化版可在消费级显卡(RTX 4090)运行,大大降低了模型的部署应用门槛。

​LLaMA-3整合CLIP视觉编码器,使其在多模态领域具有较强的应用能力,图像问答VQA准确率82\%。
\subsection{DeepSeek}
在架构层面上,DeepSeek采用了混合专家架构(MoE)​、​多头潜在注意力(MLA)以及动态路由策略,以更好地对训练过程进行优化,降低
对硬件的需求。其采用细粒度专家分割(256个路由专家+1个共享专家),激活参数仅占全量参数的1/10,推理速度提升3倍。
此外,DeepSeek还通过低秩键值压缩技术,将KV缓存需求降低60\%,支持64k token长上下文处理(约3-4万字)。基于负载均衡优化算法,专家利用率提升30\%,在代码生成任务中HumanEval得分达53.8\%。

在训练优化上,DeepSeek使用了FP8混合精度训练的方案,结合BF16与FP8量化,单卡H800训练吞吐量提升2.1倍,175B模型训练成本降至OpenAI的1/10。
此外,也采用了​强化学习优先(RL-First)​的策略,DeepSeek-R1通过纯强化学习训练实现复杂推理,数学证明任务准确率比GPT-4o高15.4\%。

相比目前其它的LLM,DeepSeek对硬件的需求更小,应用场景拓展到了移动端本地部署。
此外,DeepSeek的高推理速度也使得其被应用于实时风险分析领域,某银行系统响应延迟压缩至200ms。
\subsection{LLM总结}
目前,LLM的产业化应用已渗透至许多行业,在金融、教育、法律等领域大显身手。

头部投行部署GPT-4o模型实现高频交易策略优化,实时解析财经新闻、财报电话会议录音,构建情感指数(Sentiment Index),预测标普500指数波动率的误差率仅2.3\%;
​通过强化学习动态调整投资组合,在2024年美股波动期间最大回撤控制在5\%以内,优于人工策略的12\%;
LLM自动解析《巴塞尔协议III》等法规文件,识别交易记录中的异常模式,某欧洲银行借此将反洗钱审查效率提升40倍。

在线教育平台“学而思”集成LLaMA-3模型开发智能辅导系统,根据学生答题数据(如错误知识点关联图)动态生成习题,某初中数学实验班平均成绩提升23\%;
采用多维度评估框架(逻辑性、语法、创新性),与教师评分的一致性系数(Cohen's Kappa)达0.81;
7×24小时解答学科问题,在江苏省高三模拟考中,系统对物理压轴题的解题正确率超过95\%。

某顶级律所联合DeepSeek-R1模型构建法律智能平台,扫描百万字合同文本,自动标记非常规条款(如对赌协议中的隐藏风险),准确率98.7\%;
基于历史案例数据库,对知识产权纠纷案的胜诉概率预测与法院判决吻合度达89\%;将法律意见书起草时间从40小时压缩至2小时,且符合《民法典》最新司法解释要求。

尽管LLM取得突破性进展,仍需解决以下问题。
\begin{enumerate}[itemsep=0pt,parsep=0pt]
\item ​高质量语料短缺与数据多样性不足的问题。训练GPT-4和Gemini Ultra需要4-8万亿单词,而人类高质量文本数据可能在2026年前耗尽。中文语料尤其匮乏,全球大模型训练集中文占比仅1.3\%,且国内缺乏垂直领域专用数据集。此外,共享语料库存在大量噪声,例如社交媒体内容碎片化、标注不一致等问题,直接影响模型性能。例如,化工行业安全检查单需通过LLM重新筛选和优化低相关性条款。
\item 模型扩展定律(Scaling Laws)的失效。OpenAI新一代模型Orion在训练进度20\%时达到GPT-4水平,但后续投入更多资源后性能提升不足5\%,显示传统“堆参数”模式已遇瓶颈。此外,GPT-4单次推理成本达数万次浮点运算,日均调用百万次的企业算力成本超数万美元,而模型能力提升与成本增长不成正比。
\item 幻觉(Hallucination)与可靠性缺陷问题。LLM在缺乏数据支撑时可能编造虚假信息,例如谷歌Bard错误描述天文照片来源。金融、医疗领域因此面临业务风险,某法律分析系统需人工审核降低误判率。
\item 推理效率与速率限制。GPT-4推理需40GB显存,平均响应延迟超3秒,远超用户容忍阈值。异步函数调用技术(如AsyncLM)通过并行化处理,可将任务加速5.4倍,缓解了这一问题,但仍需硬件优化。
\end{enumerate}
\section{DSPy}
为了实现大语言模型与符号推理模块之间的高效协同,
近年来出现了一种新型编程范式——声明式语言模型编程(Declarative LLM Programming)。
DSPy(Declarative Self-improving Python)正是该范式的代表性框架。
它通过模块化、声明式的方式,将语言模型调用与复杂任务逻辑解耦,极大地简化了神经符号系统的构建流程。

DSPy 将语言模型视为可组合的模块(Modules),如 Prompt, ChainOfThought, ReAct, Retrieve, Select 等,
并支持在这些模块中嵌入目标任务的“意图说明”(specifications)。它最突出的特点包括:
\begin{enumerate}[itemsep=0pt,parsep=0pt]
\item \textbf{声明式建模}:用户只需定义“输入-输出要求”,而不需要手动编写 Prompt 或控制模型细节。
\item \textbf{程序级优化}:框架通过训练、调优或样例注入自动优化模块表现,具备一定的自我改进(Self-improvement)能力。
\item \textbf{LLM-agnostic 架构}:DSPy 可无缝对接 ChatGPT、LLaMA、DeepSeek 等多种 LLM,提升了系统的灵活性与可移植性。
\end{enumerate}

DSPy 所具备的可组合性、可解释性和与 LLM 的深度集成能力,
使其成为构建“可控”“结构化”“可调试”的神经符号框架的理想桥梁。
与传统的 Prompt 工程相比,DSPy 更适合构建语义导向、规则驱动的VQA系统,
特别是在空间推理任务中,其对规则生成、语义一致性与模型泛化能力提供了有力支持。
\section{本章小结}
首先介绍 CLEVR 数据集,该数据集与积木世界高度契合,是构建符合本研究的部分可见VQA数据集的重要参考依据。
其次,介绍回答集编程(Answer Set Programming, ASP),围绕其语法、语义以及不同类型的ASP求解器展开了介绍。
接着,介绍本文所用的视觉语言预训练模型 GLIP,它能够实现图文联合理解和目标定位,为视觉信息的结构化提供强有力支持。
随后,将介绍近年来发展迅速的大语言模型(Large Language Models, LLMs),并对当前LLM发展需要解决的问题和LLM目前已经应用的领域展开介绍。
最后,介绍 DSPy 框架,其作为一种以程序化方式组织和优化大语言模型行为的开发工具,在本研究中被用于规则蒸馏、迭代反馈与逻辑修正等需要对LLM进行微调的环节,
从而简化神经符号VQA框架的开发流程。本章介绍的众多技术手段,为后续进一步开展研究提供了重要保障。

\chapter{数据集}
对模型的空间推理能力进行充分评估,高质量的数据集必不可少。本章将详细介绍如何构造用于空间推理SRASP数据集,并对数据集的质量进行测试。

\section{设计目标}
尽管CLEVR数据集至今仍被广泛应用于多模态模型的空间推理能力研究,但其存在着一些较为明显的问题,其中比较凸出的是,CLEVR数据集中的场景是完全可观察的,模型可以直接从图像中获取所有必要信息。现实世界中,有很多信息并不能直接从图像中获取。人类基于图像对问题进行回答时,除了直接从图像中观察到的信息,往往也需要使用已经积累的先验知识。本文在此对空间推理的问题进行如下定义:如果回答该问题所需要的全部知识均已包含在图像中,即回答问题不需要使用先验知识,那么称该问题为完全可观察问题。同理,如果图像中并不包括回答该问题所需的全部知识,也即回答该问题需要使用已经积累的先验知识,那么称该问题为不完全可观察问题。

为了解决这一问题,本章基于CLEVR数据集,构建一个名为SRASP的数据集。该数据集的主要设计目标聚焦于以下几个方面:
\begin{enumerate}[label=(\arabic*),itemsep=0pt,parsep=0pt]
\item 考察模型对不完全可观察问题的解答能力。通过引入部分可观察性,模拟现实世界中物体被遮挡或隐藏的场景,要求模型在不完整的视觉信息下进行推理,考察模型真正解决问题的能力。
\item 考察模型对推理密集型问题的解答能力。现实世界的同一场景中,往往存在多个物体,物体之间的关系多样,解决空间推理问题需要多个步骤,逐步解决。通过增大解决问题所需的推理跳数,可以确保模型无法通过简单的模式匹配或者直接观察得出答案,进而真正考察模型的思考和推理能力。
\item 考察模型的知识整合能力。正如在数学证明中定理之间相互印证产生新的定理一样,回答问题所需要的知识,往往也是需要将现有知识进行结合产生新知识,才能最终解决问题。在SRASP数据集中,为每个场景提供一组逻辑约束作为背景知识,模型需将这些先验知识与观察到的视觉信息相结合,以生成正确的答案。
\end{enumerate}

\section{物体基本属性与约束}
与CLEVR数据集中的几何体一样,SRASP数据集的图像中的每个物体,均有形状、尺寸、材质、颜色四种属性。每种属性的可能取值如下:
\begin{enumerate}[label=(\arabic*),itemsep=0pt,parsep=0pt]
    \item 形状:圆锥体、球体、圆柱体和立方体。
    \item 尺寸:小、中、大。
    \item 材质:橡胶、金属。
    \item 颜色:红色、蓝色、绿色、黄色、灰色、棕色、紫色、青色。
\end{enumerate}

除上述四种基本属性之外,图像中的物体也有“所在区域”这一属性,可取值为0、1、2、3。由于所有图像都被划分成
4个区域,故图像中物体也会处在某一个区域之中。为了研究问题方便,本文规定每个物体只能在图像的一个区域中,不能
同时跨多个区域。物体包含这一属性能够为指定约束提供便利。在本数据集中,约束是一组规则的集合,对场景生成
与推理而言密不可分,其主要作用包括:(1)限制物体的属性组合,如对颜色、形状等进行限制;
(2)定义物体之间的关联关系;(3)支持对遮挡物体的推理,当物体被遮挡时,通过约束可以缩小潜在答案的范围。

基于对物体的应用范围,约束可以划分为以下三类:
\begin{enumerate}[itemsep=0pt,parsep=0pt]
    \item 区域约束:仅作用于特定区域的局部规则。例如,“区域1中所有物体形状必须为立方体”。
    \item 跨区域约束:涉及多个区域的全局规则。例如,“区域1和区域2中同颜色物体的总数不超过2个”。
    \item 全局约束:适用于整个场景的通用规则。例如,“所有物体必须属于至少一个区域”或“不允许存在完全相同的两个物体属性组合”。
\end{enumerate}

约束通过使用ASP来进行表示。例如:
\begin{lstlisting}
    :- object(X), at(X, 0), not hasProperty(X, shape, cube), not hasProperty(X, shape, cylinder).
\end{lstlisting}
表示如果X在区域0,那么它的形状必须是立方体或圆柱体。

\section{构造流程}
在确定图像中物体的基本属性之后,本文进一步确定了构建数据集的如下步骤流程:
\begin{enumerate}[itemsep=0pt,parsep=0pt]
\item 生成一组由约束定义的环境,记作$Environment_i$。
\item 生成一个完整的场景图$Complete_i$。该场景图完全符合上一步生成的环境$Environment_i$的要求。
\item 通过从完整场景图$Complete_i$中删除一个物体$Obj_i$,来生成一个部分场景图$Partial_i$。
\item 生成一个对于部分场景图$Partial_i$,对物体$Obj_i$的有关情况进行提问的问题$Q_i$。
\end{enumerate}

\section{环境表示定义}
SRASP中的环境是由一组约束来定义的。与场景相比,环境是一个更抽象一层的概念。针对某个特定的环境,
可以生成以其为模板的场景。可以理解为,环境是场景的抽象,场景是环境的实例。
每个约束决定了特定环境下物体的属性限制。本文首先设计了10个约束模板,所有模板使用ASP来进行表示。
每个环境最多通过20个不同的实例化后的模板来创建,也即每个环境中最多包含20个不同的约束。
部分约束模板的ASP编码表示以及对应表示含义见表\ref{tab:asp_templates}。最终,一共生成了50个
环境,数据集中的每个场景都是由其中一个环境进行实例化后生成的。环境的具体示例见附录\ref{appendix:environment}。

\begin{table}[!h]
    \centering
    \renewcommand{\arraystretch}{1.0}
    \begin{tabular}{|p{3cm}|p{12cm}|}
        \hline
        \textbf{模板} & \textbf{描述} \\
        \hline
        \textbf{模板1(取值约束)} & 
        \texttt{:- object(X), at(X, R), not hasProperty(X, P1, V1).} \\ 
        & 解释: 对区域R中的所有物体,它们P1属性的取值均为V1。 \\ 
        & 具体实现: :- object(X), at(X, 0), not hasProperty(X, color, red). \\
        \hline
        
        \textbf{模板2(否定约束)} & 
        \texttt{:- object(X), at(X, R), hasProperty(X, P1, V1).} \\ 
        & 解释:对区域R中的所有物体,它们的P1属性的取值,均不能为V1。 \\ 
        & 具体实现::- object(X), at(X, 0), hasProperty(X, material, metal). \\
        \hline
        
        \textbf{模板3(恰有N个约束)} & 
        \texttt{:- \#count\{X: hasProperty(X, P1, V1), object(X), at(X, R)\} != N.} \\ 
        & \textbf{解释}:在区域R中,恰好有N个物体的P1属性的取值为V1。 \\ 
        & 具体实现::- \#count\{X: hasProperty(X, size, small), object(X), at(X, R')\} != 2. \\
        \hline
        
        \textbf{模板4(至少有N个约束)} & 
        \texttt{:- \#count\{X1, X2: sameProperty(X1, X2, P1), object(X1), object(X2), at(X1, R1), at(X2, R2)\} < N.} \\ 
        & 解释:在区域R1和区域R2中,至少有N对物体,它们的P1属性的取值都是V1。 \\ 
        & 具体实现::- \#count\{X1, X2: sameProperty(X1, X2, shape), object(X1), object(X2), at(X1, 1), at(X2, 2)\} < 1. \\
        \hline
        
        \textbf{模板5(或约束)} & 
        \texttt{:- object(X), at(X, R), not hasProperty(X, P1, V1), not hasProperty(X, P1, V2).} \\ 
        & 解释:区域 R中的所有对象都具有属性 P1 的 V1 值或属性 P2 的 V2 值。 \\ 
        & 具体实现::- object(X), at(X, 1), not hasProperty(X, color, yellow), not hasProperty(X, color, blue). \\
        \hline
    \end{tabular}
    \caption{部分约束模板示例}
    \label{tab:asp_templates}
\end{table}
\section{场景表示定义}
场景是环境的实例。
SRASP数据集以场景图的形式表示场景,其节点表示使用其属性进行注释的对象,边表示对象之间的空间关系(前、后、左、右)。
在SRASP中,除了场景图表示之外,也用ASP对场景进行表示。
以下展示图\ref{}中部分场景的ASP​表示:
\begin{lstlisting}
%场景中的物体
object(0). object(1). object(2). object(3).

%物体的属性
at(0, 2).
hasProperty(0, color, green).
hasProperty(0, size, large).
hasProperty(0, material, rubber).
hasProperty(0, shape, cylinder).
....

%物体间的空间关系
front(1, 0). right(1, 0). ...
\end{lstlisting}

其中涉及到的谓词的功能如下:谓词\texttt{object}用于定义不同的物体(所有物体的名称用0,1等数字来表示)。
谓词\texttt{hasProperty(Object, Attribute, Value)}用于将对象的名为Attribute的属性的值设置为Value。
对象之间的空间关系用谓词\texttt{left}、\texttt{right}、\texttt{front}、\texttt{behind}来表示,例如
\texttt{left(A, B)}表示B位于A的左侧。
\section{图像生成}
图像生成基于前文中定义的场景图。场景图是一种对场景中物体、属性及其空间关系的结构化描述,而环境则由一系列约束条件所决定,
这些约束可能包括空间分布、物体间的相对关系以及特定属性的限制。
因此,场景图的生成问题可以归结为一个复杂的推理问题:在给定的环境约束(基于ASP的规则)以及场景中预期的物体数量$n$这两个前提下,完成以下任务:
\begin{enumerate}[itemsep=0pt,parsep=0pt]
\item \textbf{区域划分}:将每个物体合理分配到预定义的四个区域之一,确保满足诸如区域容纳量、相邻关系及物体间可能的干扰等约束;
\item \textbf{属性赋值}:为每个物体赋予颜色、尺寸、形状和材质等属性,其取值必须与环境中规定的约束条件一致。例如,某些区域可能只允许出现特定颜色或尺寸范围的物体;
\item \textbf{关系一致性}:在属性分配过程中,还需要确保各物体之间的关系(如邻近、对称或排斥关系)符合逻辑规则,从而保证场景图整体的合理性和一致性。
\end{enumerate}

为了解决上述推理问题,本文采用了ASP的方法。ASP作为一种声明性逻辑编程范式,特别适合解决复杂约束和组合优化问题。
在本系统中,ASP 求解器的工作流程主要包括以下几个步骤:
\begin{enumerate}[itemsep=0pt,parsep=0pt]
\item \textbf{约束建模}:将场景的环境约束以及物体属性赋值规则形式化为 ASP 规则。此步骤需要充分利用逻辑公式来描述物体的空间位置、属性取值范围以及各类关系约束;
\item \textbf{回答集计算}:在输入了物体数量$n$和所有相关约束之后,ASP 求解器会计算出满足所有约束条件的解集。每一个答案集代表一种物体属性及区域分配的合理配置,即一种可能的场景图;
\item \textbf{解集筛选与随机采样}:由于满足所有约束的配置方案可能数量巨大,为了使后续图像生成过程具有一定的随机性与代表性,系统从所有可能的解集中随机采样一百万个场景图。这一随机采样策略既保证了生成场景的多样性,也为后续的图像生成提供了充分的候选数据。
\end{enumerate}

在获得充分的场景图后,系统利用 Blender3 进行图像渲染。
Blender3 是一款高效且功能丰富的三维渲染软件,它能够基于场景图的结构信息生成逼真的图像。
渲染过程包括以下几个关键步骤:(1)场景搭建,根据场景图中各物体的位置信息及属性参数,在 Blender3 中自动搭建三维场景;
(2)光照与材质设置,对各物体的材质、光照、纹理等进行设置,确保渲染出的图像在视觉上具有真实感;(3)
图像生成,利用 Blender3 的渲染引擎,将搭建好的场景生成最终图像。

图\ref{pipeline_for_generating_environment}直观展示了从环境约束到场景图构建,再到图像生成的整个流水线过程,为后续工作的深入探讨提供了坚实的理论与实践基础。
\begin{figure}
    \centering
    \includegraphics[width=\textwidth]{pipeline_for_generating_environment.png}
    \caption{生成环境以及该环境中的完整场景的流水线}
    \label{pipeline_for_generating_environment}
\end{figure}
\section{问题表示定义}
在SRASP数据集中,每个问题均围绕部分场景中缺失物体的四个关键属性之一:颜色、大小、形状和材质。
问题的设计目标在于通过推理补全场景信息,从而考察系统在不完整场景下对物体属性关系的理解能力。
为此,本文将自然语言描述的问题转化为基于ASP的形式化表示,
使得问题的求解过程可以通过逻辑推理得到明确答案。

数据集中的每个问题最初以自然语言的形式提出,例如:“与中等大小的红色物体的材质相同的,另一个圆柱体的颜色是什么?”
为了使问题具有可操作性,本文设计了对应的ASP编码,如下所示:
\begin{lstlisting}
    query(Q) :- hasProperty(X, color, Q),
    hasProperty(X, shape, cylinder),

    hasProperty(Y, size, medium),
    hasProperty(Y, color, red),
    same_material(Y, X),
    X != Y.
\end{lstlisting}
在该编码中,\texttt{query(Q)}表示需要推导出满足条件的物体X的颜色Q。
此外,通过多个\texttt{hasProperty}和\texttt{same\_material}条件,明确限定了参与推理的物体之间的属性关系,并利用 X != Y 排除自反情况。
这种表示方式不仅使问题的语义精确、结构清晰,而且便于通过 ASP 求解器进行自动求解,从而为数据集中的问题构建提供统一、标准的表达形式。

在问题生成过程中,必须对每个问题的答案范围进行严格控制。对于涉及属性$A$(其中
$A \in \{ color, size, material, shape\}$)的问题,其可能的解集$S$的大小满足$1 \leq |S| \leq |A|$。
其中,$|A|$表示属性$A$所有可能取值的数目。例如,对于尺寸属性,若其可能的取值集合
$\{ large, medium, small\}$,则$|size| = 3$。

如果某个问题生成的解集数量恰好为$|A|$,例如问题答案为“尺寸可以为 large、medium 或 small”,则该问题在特定场景下并未起到区分或推断作用,因此被判定为无效。这一设计思路确保了数据集中每个问题在解答上都具有针对性和挑战性,从而避免生成普遍适用的、无区分价值的答案。
\section{问题生成}
在SRASP数据集中,每个问题均围绕部分场景中缺失物体的某一关键属性展开,如颜色、大小、形状或材质。
为此,本文设计了一套模板,用以指导自然语言问题的生成。以下为问题模板样例:
\begin{lstlisting}
What shape is the < Z2 > (size) < C2 > (color) < M2 > (material) [that is] 
< R > (relation) the < Z > (size) < C > (color) < M > (material) < S > (shape) ?
\end{lstlisting}
其中,<Z2>、<C2>、<M2> 表示待查询对象的已知属性(例如尺寸、颜色、材质),由随机策略从完整场景图中选取;
<R> 为空间关系(如left、right、front、behind),其取值既满足随机性,又依赖于完整场景中物体间的真实空间分布;
<Z>、<C>、<M>、<S> 则代表参考对象的属性,通过对完整场景图中与查询对象具有特定空间关系的候选对象进行筛选而确定。
这种模板化设计不仅使自然语言问题的结构化描述成为可能,而且便于后续转换为ASP的形式化表示,从而实现问题求解的自动化。

问题模板的实例化过程基于与图像对应的完整场景图。具体步骤包括:
\begin{enumerate}[itemsep=0pt,parsep=0pt]
\item 场景图构建与部分场景生成。利用完整场景图构建方法,将真实场景中的物体、属性以及空间关系进行抽象建模;从完整场景中随机移除一个物体,以构造部分场景图,此移除的对象即为“查询对象”,其缺失的属性将作为问题求解目标。
\item 已知属性的随机选取。对于查询对象,模板中的已知属性(例如<Z2>、<C2>、<M2>)由随机采样策略确定,确保不同问题之间在属性分布上具有较好的随机性和代表性;同时,选取的属性应满足数据集整体的“问题类型平衡”要求,避免某一属性出现频率过高或过低。
\item 空间关系的确定。对于模板中表示空间关系的部分(<R>),取值虽然随机,但参考对象的选择依赖于完整场景图中的物体空间布局。具体而言,从与查询对象存在 <R> 关系的物体中进行候选对象的筛选,从而确保问题中提及的空间关系具有实际语义意义。
\end{enumerate}

为了使问题具有可操作性和求解性,所有生成的问题均转化为ASP形式。转换过程中包括以下步骤:
\begin{enumerate}
\item 规则构建。将自然语言问题中的各项约束(属性约束、空间关系约束、对象排他性约束等)以ASP规则的形式表达;
\item 约束整合。同时将部分场景图与环境约束作为求解器的输入,确保问题求解过程在完整逻辑下进行;
\item 解集判定。由ASP求解器计算出问题的所有可能解。针对每个属性$A \in \{ color, size, material, shape\}$
,若生成的解集$S$满足$1 \leq |S| \leq |A|$,则问题被认为具有合理的答案区分性;若解集规模等于$|A|$
说明答案涵盖了属性的全部取值,缺乏针对性,此时问题将被判定为无效并从数据集中剔除。
\end{enumerate}

图\ref{pipeline_for_generating_partial}展示了从完整场景图构建、部分场景生成、问题模板实例化、ASP表示转换,到最终问题求解的流水线过程。
该流程通过随机采样,使生成的问题在属性组合和空间关系上呈现多样性;通过对解集规模的判定机制,
有效过滤掉普遍适用或无区分意义的问题,确保每个问题都能够准确反映场景中物体属性的关系;
使用模板化设计和ASP表示为后续数据集扩充与新问题类型的加入提供了灵活性和统一标准。
\begin{figure}
    \centering
    \includegraphics[width=\textwidth]{pipeline_for_generating_partial.png}
    \caption{生成部分场景和问题,并进行标记的流程}
    \label{pipeline_for_generating_partial}
\end{figure}
\section{数据集分析}
\subsection{统计分析}
问题模板的统计分布及问题数量在5到9之间的查询属性分布统计见图\ref{fig:template_statistics}。
本数据集是基于CLEVR数据集进行生成的,在问题模板方面,采用了CLEVR数据集中的六种问题模板。此外,也展示
了特定类型的问题在不同场景物体数量下的分布情况。
\begin{figure}
    \includegraphics[width=\textwidth]{figures/template_combined-crop.pdf}
    \caption{问题模板统计及问题数量在5到9之间的查询属性分布统计}
    \label{fig:template_statistics}
\end{figure}

问题分布的统计图见图\ref{fig:question_statistics}。从统计图中可得知,有关颜色和形状的问题在SRASP数据集中
占比最高,分别是39\%和37.6\%,关于大小和材质的问题则相对较少,分别只占到了13.1\%和10.3\%。
此外,统计图中也展示了不同类型问题的答案集分布。在生成数据集的过程中尽量实现均衡分布,避免多数问题
指向相同答案集的情况。例如,当问题涉及物体尺寸时,其潜在解可能为\{大、中\}、\{大、小\}、\{小、中\}、\{大\}、\{中\}或\{小\}。
\begin{figure}
    \includegraphics[width=\textwidth]{figures/combined_statistics-crop.pdf}
    \caption{问题分布统计图}
    \label{fig:question_statistics}
\end{figure}

答案的分布统计。
\subsection{质量分析}
\subsubsection{质量保证}
为了保证数据集的质量,采用双盲审核机制,邀请两名评审员各自独立对数据集进行评审,验证每个问题
的答案、推理解答所需步数及问题是否可观察。在评审过程中,如果两名评审员同时判定该问题出现错误,那么
将该问题剔除。双盲评审的结果见表\ref{tab:kappa},其中展示了评审员1和评审员2分别与初始标注的一致性,以及两名评审员
之间的一致性。通过Fleiss's Kappa衡量的评审员间的一致性超过0.8,证明了数据集的可靠性。
\begin{table}[h]
    \centering
    \renewcommand{\arraystretch}{0.8}
    \begin{tabular}{lccc}
    \toprule
     & \makecell{答案是否正确} & \makecell{推理所需步数} & \makecell{问题是否可观察}\\
    \midrule
    初始标注与评审员1 & 81.2 & 84.4 & 89.6 \\
    初始标注与评审员2 & 84.2 & 85.6 & 85.4 \\
    评审员1与评审员2 & 80.1 & 83.8 & 86.9 \\
    \midrule
    平均值 & 81.8 & 84.6 & 87.3 \\
    \bottomrule
    \end{tabular}
    \label{tab:kappa}
    \caption{评审员1、2和初始标注之间的标注者间一致性}
\end{table}
\subsubsection{难度保证}
本文通过记录2位评审员在SRASP数据集上回答问题时的正确率、以及所需检索信息的次数、回答问题所需要的跳数,来对
数据集的难度进行判断,并于其它现有的VQA数据集进行比较。实验结果见表\ref{tab:human_performance},其中明显可以看出,SRASP回答问题所需
的跳数,明显大于现有数据集VQAv2,另外SRASP所需的检索信息的次数也更多一些,这些都证明了回答SRASP数据集
所需的外部知识更多,考虑次数更多,数据集难度更大。另外,人类在本文
构造的SRASP数据集上的准确率最低,进一步侧面印证了本文构造数据集的挑战性。
\begin{table}[h]
    \centering
    \renewcommand{\arraystretch}{0.8}
    \begin{tabular}{lccc}
    \toprule
     & \makecell{回答问题正确率} & \makecell{推理所需跳数} & \makecell{检索信息次数}\\
    \midrule
    SRASP & 81.2 & 84.4 & 89.6 \\
    CLEVR & 84.2 & 85.6 & 85.4 \\
    GQAv2 & 80.1 & 83.8 & 86.9 \\
    \midrule
    平均值 & 81.8 & 84.6 & 87.3 \\
    \bottomrule
    \end{tabular}
    \label{tab:human_performance}
    \caption{人类评审员在不同VQA数据集上回答问题的表现}
\end{table}
\section{本章小结}
本章介绍了SRASP数据集的构造过程,重点描述了如何基于完整场景图生成部分场景图,
并通过 ASP 实例化问题模板以确保问题的多样性和合理性。

首先,本章阐述了对象移除的原则,即如何选择查询对象以及如何保证其属性在问题类型上的均衡性。
随后,详细说明了查询模板的填充策略,包括查询属性的选取、参考对象的确定以及空间关系的合理性,
以确保生成的问题符合实际场景。最后,本章介绍了 ASP 求解器在问题生成过程中的作用,
即利用 ASP 规则对部分场景进行推理,以确定查询属性的可能取值范围,从而生成符合逻辑约束的高质量视觉问答数据。

通过上述方法,SRASP数据集不仅保证了问题的可解释性,还增强了对复杂空间关系的推理能力,
为视觉问答任务提供了更具挑战性的数据支持。

\chapter{实验及结果分析}
\section{实验设计}
\section{结果分析}
\section{本章小结}

\chapter{问答系统的构建}
\section{回答集编程}
\subsection{语法}
项(terms)是ASP程序中最基本的元素,原子
\subsection{语义}
\subsection{求解器}
\section{GLIP}
\section{大语言模型}

\chapter{结论与展望}
\section{回答集编程}
\subsection{语法}
项(terms)是ASP程序中最基本的元素,原子
\subsection{语义}
\subsection{求解器}
\section{GLIP}
\section{大语言模型}

\backmatter
%打印参考文献表
\thesisbib

\appendix

\end{document}
