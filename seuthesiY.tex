\documentclass[algorithmlist, figurelist,tablelist, nomlist, engineering, AutoFakeBold]{config/seuthesiY}
\usepackage{amsmath}
\usepackage{enumitem}
\usepackage{graphicx}
\usepackage{booktabs}
\usepackage{multirow}
\usepackage{listings}
\usepackage{xcolor}
\usepackage{enumitem}
\usepackage{array}
\usepackage{graphicx}
\usepackage{makecell}

\lstset{
    basicstyle=\ttfamily,        % 代码字体
    keywordstyle=\bfseries,      % 关键词加粗
    columns=fullflexible,        % 适应文本宽度
    frame=single,                % 加框
    breaklines=true,             % 自动换行
    postbreak=\hbox{$\hookrightarrow$}, % 换行符
    xleftmargin=10pt,            % 左边距
}

%===============================================================================
% 采用的编译方式为  XELATEX -→  BIBER -→    XELATEX -→    XELATEX
% 为了加快字体的缓存效率 需要在命令行运行 fc-cache
% 对于学位论文需要标注【硕博】
%===============================================================================

% %biblatex宏包的参考文献数据源加载方式
\addbibresource[location=local]{config/seuthesiY.bib}

\begin{document}
%===============================================================================
\categorynumber{000} % 分类采用《中国图书资料分类法》
\UDC{000}            %《国际十进分类法UDC》的类号
\secretlevel{公开}    %学位论文密级分为"公开"、"内部"、"秘密"和"机密"四种
\studentid{222171}   %学号要完整,前面的零不能省略。
\title{积木世界VQA中的}{积木世界VQA中的}{空间推理问答技术研究与实现}{空间推理问答技术研究与实现}{Research and Implementation of Spatial Reasoning Questioning Answering Techniques }{in the Block World VQA}
\author{贾梁}{Jia Liang}
\advisor{张志政}{}{Zhang Zhizheng}{}
% \coadvisor{张志政}{副教授}{Zhang Zhizheng}{Associate Prof.} % 没有% 可以不填
\degreetype{工程硕士}{Master of Engineering} % 详细学位名称
\thesisform{应用研究} % 包括应用研究、调研报告、规划、产品开发、案例分析、项目管理、文学艺术作品、其它。非专业型硕士可忽略
\major{电子信息}
\submajor{计算机技术}
\defenddate{2025年5月30日}
\authorizedate{2025年6月20日}
\committeechair{翟玉庆}
\reviewer{倪庆剑}{张祥}
\department{东南大学计算机科学与工程学院}{School of Computer Science and Engineering}
\makebigcover
\makecover
\begin{abstract}{回答集编程,视觉问答,空间推理,神经符号方法}
积木世界是人工智能研究、教学、实验、评估的重要场景,能够模拟实际应用中物体间的空间关系。
积木世界VQA要求视觉语言模型(Visual Language Model, VLM)具备空间推理能力,但已有研究表明,当面对部分可见场景时,
其回答准确率显著下降。目前,通过结合神经网络和符号推理形成的神经符号模型,利用深度学习将视觉信息转化为逻辑符号,
再利用符号推理进行问题求解问题,比单纯依赖深度学习的视觉语言模型在空间推理任务中有更优异的表现。
由于回答集程序(Answer Set Program, ASP)是一种具备非单调推理和高效推理机的符号推理方法,
神经符号模型中采用ASP构建的神经符号VQA框架被寄予厚望。然而,现有框架设计中ASP规则的扩展仍依赖人工,
难以有效应对部分可见场景中基于空间推理的VQA任务。
    
针对上述问题,本文从构建积木世界部分可见场景中基于空间推理的VQA数据集、
设计规则自动补充的神经符号VQA方法、设计实现VQA课堂演示原型系统三个方面开展工作,具体如下:

\begin{enumerate}[nosep]
\item 部分可见积木世界场景空间推理VQA数据集(Partial Observation VQA Dataset, POVQA\-D)构建。
CLEVR是积木世界的经典数据集,然而CLEVR中问题涉及的物体属性和物体间空间关系在图像中均完全可见,
通过引入部分可见性与环境约束,构建了覆盖部分场景可见问题的VQA数据集POVQAD。
POVQAD要求模型能够利用背景知识和部分可见场景中的信息进行推理
,比原数据集更能有效考察模型在部分可见积木世界场景中回答空间推理问题的能力。
\item 规则自动补充的神经符号VQA框架(Rule Complement Neuro-Symbolic Pipeline, RCNSP)设计。
RCNSP借鉴现有神经符号VQA框架的工作,对LLM进行提示词优化以提高ASP规则生成、规则修正的语法及语义准确率,
并新增规则蒸馏模块以实现ASP规则自动拓展。
实验表明,在DeepSeek、LLaMA3、ChatGPT-4o三种大语言模型上,使用RCNSP比直接向VLM提问的准确率平均提升16.5\%,
比现有的神经符号VQA框架的准确率平均提升8.9\%,
表明通过规则蒸馏能有效提升神经符号方法在部分可见积木世界场景下空间推理问答的准确率。
\item 设计实现了一个积木世界VQA原型系统。在RCNSP框架和POVQAD数据集基础上,设计实现了一个
积木世界VQA原型系统。
该系统模拟了在自动规划课程的授课场景下,由教师向系统提出积木世界的空间推理问题,系统进行解答并展示推理的中间步骤和逻辑链条,
并支持自定义生成复杂度不同的演示数据集,
为教师向学生展示智能体如何理解外部环境信息并进行规划提供了便利。
初步测试表明该系统在CPU为Intel Core i9-12900K,内存128G,显卡为3张RTX 3090并联的硬件环境下,并发量为38,
90\%响应时间为6.9秒,能够满足自动规划课程教学场景下的用户需要。
\end{enumerate}
\end{abstract}

\begin{englishabstract}{Answer Set Programming, Visual Question Answering, Spatial Reasoning, Neuro-symbolic Method}
The Block World is a fundamental scenario for research, education, experimentation, and evaluation in artificial intelligence, capable of simulating spatial relationships among objects in real-world applications. Block World Visual Question Answering (VQA) tasks require Visual Language Models (VLMs) to possess spatial reasoning capabilities. However, existing studies have shown that the accuracy of VLMs significantly drops in partially observable environments. Neuro-symbolic models, which integrate neural networks and symbolic reasoning, convert visual information into logical symbols using deep learning and solve problems through symbolic inference. These models have demonstrated superior performance in spatial reasoning tasks compared to purely neural-based VLMs.

Answer Set Programming (ASP), a symbolic reasoning paradigm with non-monotonic reasoning and efficient solvers, has been widely adopted in neuro-symbolic VQA frameworks. However, current frameworks still rely heavily on manual rule construction, making it difficult to handle spatial reasoning tasks in partially observable scenarios effectively.

To address this issue, this work focuses on three main aspects: constructing a spatial reasoning VQA dataset under partial observability in Block World, designing a neuro-symbolic VQA method with automatic rule augmentation, and developing a classroom demonstration prototype system for VQA. Specifically:

\begin{enumerate}[nosep]
\item \textbf{Construction of the Partial Observation VQA Dataset (POVQAD):}  
CLEVR is a classical dataset for spatial reasoning, but all object attributes and spatial relations in its scenes are fully visible. We introduce partial observability and environmental constraints to construct POVQAD, a dataset that challenges models to reason with background knowledge and limited visual input. Compared to existing datasets, POVQAD better evaluates a model's capability in answering spatial reasoning questions under partial observability in the Block World.

\item \textbf{Design of the Rule Complement Neuro-Symbolic Pipeline (RCNSP):}  
RCNSP builds upon existing neuro-symbolic frameworks and enhances large language models (LLMs) with prompt optimization to improve the syntactic and semantic accuracy of ASP rule generation and correction. Furthermore, it introduces a rule distillation module to automatically expand ASP rules. Experiments with DeepSeek, LLaMA3, and ChatGPT-4o demonstrate that RCNSP improves the average accuracy by 16.5\% over direct VLM questioning and by 8.9\% over existing neuro-symbolic VQA frameworks. These results confirm that rule distillation significantly enhances the spatial reasoning accuracy in partially observable Block World scenarios.

\item \textbf{Implementation of a Block World VQA Prototype System:}  
Based on the RCNSP framework and the POVQAD dataset, a prototype system for Block World VQA is implemented. This system simulates classroom teaching of automated planning, where instructors pose spatial reasoning questions to the system. The system provides answers along with intermediate reasoning steps and logical chains, helping students understand how intelligent agents perceive and plan in their environments. Preliminary tests show that on a system with an Intel Core i9-12900K CPU, 128GB RAM, and three RTX 3090 GPUs, the system supports a concurrency of 38 with a 90\% response time of 6.9 seconds, meeting the performance requirements of educational scenarios in automated planning.
\end{enumerate}
\end{englishabstract}

\setnomname{术语与符号约定}
\tableofcontents
\listofothers
%===============================================================================


\mainmatter

\chapter{绪论}
\section{研究背景}
随着计算机视觉(Computer Vision)
和自然语言处理(Natural Language Processing)技术的迅猛发展,
跨模态智能理解成为人工智能研究的重要方向之一。
其中,视觉问答(Visual Question Answering, VQA)\cite{goyal2017making}任务因其广泛的应用前景和挑战性,
受到了学术界和工业界的广泛关注。

视觉问答是一种多模态任务,它要求计算机能够基于给定的图像内容,
理解并回答关于该图像的自然语言问题。例如,给定一幅包含动物的图片,
系统需要能够回答“这只动物是什么颜色?”或“图片中有几只猫?”等问题。
这一任务的核心在于多模态信息的深度融合,即如何在视觉特征和语言信息之间建立有效的联系。

视觉问答在多个实际应用场景中具有重要价值。
例如,在辅助盲人阅读图像信息、自主机器人理解环境、医疗影像分析、教育和娱乐等方面,
VQA 技术都可以提供智能化的交互方式,提高系统的可用性和便利性。
然而,由于数据的不确定性、语言表达的多样性、以及多模态信息融合的复杂性,
VQA 仍然面临诸多挑战,例如开放域问题的泛化能力、推理能力的提升、以及对长尾问题的有效应对等。

随着现实场景愈发复杂多样,VQA系统需要解决的问题也愈发困难,尤其是涉及物体间相对位置、形状、
大小等空间关系的推理任务。空间推理指的是理解物体在某个二维或者三维的场景中的位置关系,并能够根据
该场景的信息,以及本身所有的关于空间的常识知识,回答涉及物体排列、方向、距离等方面的问题。
例如,在问答任务中,问题可能为“杯子在桌子的哪一侧?”或者“哪个物体位于另一个物体的上方?”这些问题
要求VQA系统能够理解图像中物体的空间关系,进行推理,并给出准确的答案。

相比传统的VQA任务,空间推理更具挑战性。为了正确回答涉及空间推理的问题,VQA系统需要能够理解
图像中的空间布局,并且在多步骤推理中保持逻辑一致性。然而,现有的大语言模型在空间推理方面仍然存在
显著局限。虽然已有的模型在目标检测方面表现十分出色,但它们在处理更复杂的空间关系时,尤其是
在缺乏常识性空间理解的情况下,容易出现推理错误。因此,如何提升大语言模型在空间推理方面的能力,
成为当前研究的重要课题。

为了解决VQA系统在空间推理方面的问题,研究者们提出了一些新的方法。近年来,神经符号方法(Neuro-Symbolic Methods)得到了广泛关注。
神经符号方法使用深度学习技术进行感知,为输入的图像和问题分别生成符号表示,
再使用符号系统进行推理求解,可以有效提升VQA系统在空间推理方面的能力。回答集编程(Answer Set Programming,ASP)是一种声明式编程范式,可用于解决复杂的
人工智能问题。ASP起源于对逻辑编程、非单调推理和知识表示的研究。ASP因其具有表达
性声明性的语言和以Clingo为代表的一些高效实现而流行起来。ASP已经在学术界和
工业界得到广泛应用,并被证明在人工智能的几个知识密集型应用中能够有效解决问题,
如调度、产品配置、机器人、劳动力管理和决策支持等。
ASP能够以简洁和直观的方式来表示知识,允许用户相对容易地去表示复杂问题,
而且ASP具有分单调推理的特性,允许不完整信息的表示和默认推理。
另外,ASP对知识的表达能力,使得其支持集成各种类型的知识,包括规则、约束和偏好,有助于灵活解决问题。
ASP的这些特性使得它作为符号系统的杰出代表,被广泛应用于神经符号方法中。

基于上述背景,本文重点关注将神经符号方法在视觉问答中的应用,分析研究现有已使用神经符号方法的VQA系统在空间推理方面的优势和不足,
,为进一步提升大语言模型在空间推理方面的能力开展研究。

\section{相关研究现状}
基于以上背景,本节主要从视觉问答的发展与现状、大语言模型在空间推理中的局限性、神经符号方法在视觉问答中的应用、视觉问答中复杂空间推理的难点等方面进行综述。

\subsection{视觉问答的发展与现状}
VQA任务最早由Antol\cite{Antol2015VQA}等人在2015年提出,标志着VQA研究早期阶段的开始。这一阶段的VQA系统采用比较简单的架构,主要包括视觉编码器、语言编码器和融合模块。
Ren\cite{ren2015exploring}等人将采用预训练的卷积神经网络(Convolutional Neural Network, CNN)作为视觉编码器提取图像特征,采用循环神经网络(Recurrent Neural Network, RNN)作为语言编码器或
提取问题特征。Malinowski\cite{malinowski2015neural}等人在视觉编码器上同样采用CNN,在语言编码器上则是采用长短期记忆(Long Short-Term Memory , LSTM)来生成问题的答案。
融合模块通常采用简单的拼接或注意力机制融合视觉和语言特征。这些系统在一些简单的VQA数据集上取得了不错的性能,但在处理复杂的空间推理问题时表现较差。

随着计算机视觉和自然语言处理技术的发展,VQA系统逐步引入了更高级的特征提取与融合方法。
例如,Yao\cite{lu2019look}等人将区域提议网络(RPN)引入到VQA模型中,以增强对图像中物体的感知能力。
注意力机制也逐渐引入VQA模型中,例如Xu\cite{xu2016stacked}等人提出堆叠注意力网络(SANs),
通过多层次注意力机制,逐步聚焦于图像和问题的关键部分,从而提升了VQA模型的推理能力。

在2019年前后,得益于基于Transformer的预训练语言模型的兴起,VQA研究取得了显著进展。如BERT、ViLBERT和LXMERT等模型通过联合训练大规模的视觉和语言数据,
学得如何在同一嵌入空间中表示视觉和文本信息,极大程度上提升了VQA的准确性和鲁棒性。
例如,Tan\cite{Tan2019LXMERT}等人提出了LXMERT模型,将预训练的BERT模型与视觉编码器相结合,实现了对图像和问题的联合编码。

如今,VQA的研究已经进入了全新阶段。随着大规模视觉语言模型(如GLIP、GPT-4等)的应用,VQA模型已经能够处理更加复杂的任务,不仅处理特定问题时效果出色,而且可以进行
跨模态推理,通过多种信息源之间的互动生成更加丰富和精准的答案。OpenAI的研究团队\cite{radford2021learning}在CLIP模型中,应用了自监督学习的方法
,利用海量的图文对数据进行训练,实现了视觉和语言的统一表示,且训练过程无需人工标注,充分体现了自监督学习的优势。Salesforce的研究团队\cite{li2022blip}在BLIP模型中,
通过生成式预训练任务,提升了视觉语言模型在理解和生成方面的性能。

\subsection{大语言模型在空间推理中的局限性}
当前,大语言模型在空间推理方面的能力仍存在显著局限。Cohn\cite{cohn2023evaluation}等人在 RCC-8 框架下对 GPT-4 在定性空间推理任务中的表现进行了评估,发现尽管 GPT-4 能够理解一些简单的空间关系,但在处理复杂的空间关系时,往往无法准确应用 RCC-8 的规则,导致推理精度较低。此外,GPT-4 在不同推理步骤之间表现出明显的不一致性,且其推理过程缺乏明确的策略,往往依赖于经验或直觉做出决策,但这些决策往往缺乏可解释性,难以追踪其推理路径。
Bang\cite{bang2023multitask}等人也指出,GPT 在空间推理任务中面临的挑战尤为突出,特别是在涉及多个空间区域、动态变化的场景或高维空间结构时,模型倾向于依赖已知的简单规则进行推理,但在面对复杂或不常见的空间配置时,难以有效处理。此外,GPT 在基于空间关系进行推理时,往往会出现错误,无法从准确的空间关系中得出合乎逻辑的结论。特别是在涉及常识性空间理解的任务中,GPT 的表现较差,难以理解如物体重叠、邻接、相交等基本的空间常识性假设。

为了解决这一问题,研究者们提出了一些新的方法,如基于神经符号方法的空间推理框架。这些方法通过将大语言模型与符号推理方法相结合,实现了对复杂空间推理问题的有效建模。
例如,Wang \cite{ishay2023leveraging}等人提出了一种基于大语言模型和 Answer Set Programming(ASP)的神经符号框架,用于解决复杂空间推理问题。
该方法通过将大语言模型与 ASP 求解器相结合,实现了对复杂空间推理问题的高效建模。此外,研究者们还提出了一些其他的神经符号方法,如基于知识图谱的推理方法、基于逻辑规划的推理方法等,以增强大语言模型在空间推理方面的能力。
\subsection{ASP在空间推理中的应用}
ASP作为一种形式化的知识表示和推理方法,已经在空间推理中得到了广泛的应用。ASP具有表达能力强、推理效率高、易于理解和调试等优点,适用于解决复杂的空间推理问题。有一些学者在这一方面做了一些研究。
例如,Wałęga\cite{walega2015aspmtqs}等人提出ASPMT(QS),将非单调空间推理与基于理论的答案集编程相结合,整合了定性和定量的空间信息,
解决了传统空间推理方法在处理复杂空间变化和组合约束时的局限性。
Baryannis\cite{Baryannis2018Trajectory}等人3将轨迹建模为由不重叠区域构成的序列,
并提出了多种ASP编码方案(包括专门优化的编码和通用编码)以利用合成表来保证约束的一致性,展示了ASP在空间推理中的应用潜力。
这些研究表明,ASP在空间推理中具有很大的潜力,可以有效解决复杂的空间推理问题。

\subsection{神经符号方法在空间推理中的应用}
尽管研究者将ASP成功引入空间推理问题,取得了一些研究成果,但ASP在感知能力方面仍然存在一些不足,如对图像和自然语言的理解能力较弱,难以处理复杂的视觉问答问题。
神经符号方法的出现,为解决这一问题提供了新的思路。
神经符号方法引入深度学习技术,为ASP求解器提供了更加丰富的输入信息,从而提升了空间推理的准确性和效率。
Tejas\cite{Gokhale2020CausalVQA}等人“逻辑透镜”(Lens of Logic, LOL)模型,采用了神经符号方法,利用了问题注意力
和逻辑注意力机制,以识别和理解问题汇总的逻辑连接词,并且引入了Fréchet兼容性损失(Fréchet-Compatibility Loss),
确保组件问题的答案与组合问题的答案在推理过程中保持一致性。
Thomas\cite{eiter2022neuro}等人提出了一种结合神经网络和符号推理的方法,该方法能够有效地进行空间推理和逻辑推理,从而在复杂地视觉问答任务中提高准确性,展示了
符号推理与深度学习结合的潜力,推动了神经符号方法在视觉问答中的应用。
Pan\cite{pan2023logic}等人对神经符号方法进行了改进,引入了一个自我优化模块,利用符号求解器的错误提示信息来修正符号表示,从而提高推理的准确性和可靠性。

\subsection{视觉问答中复杂空间推理的难点}
空间推理作为视觉问答系统所需的重点核心能力,涉及对图像中物体间拓扑关系、方位、尺寸等多种空间信息的理解、建模和推理。然而,
现有方法在空间推理中仍面临诸多挑战。主要表现在以下几个方面:

\subsubsection{复杂空间关系的建模与表示}
空间推理需要处理多层次关系,如拓扑关系(包含、相邻)、方位(左/右、前/后)、动态轨迹(移动路径)等。
传统方法依赖预定义逻辑规则(如RCC-8拓扑模型),但难以适应开放域场景的多样性\cite{li2021algorithm}。
例如,自然语言中的“靠近”这个词汇,缺乏量化阈值,难以定量界定两个物体之间的距离在什么场景下为“靠近”这一关系,导致符号逻辑无法精确映射\cite{shrestha2019answer}。
另外,基于注意力机制的模型虽然能定位目标区域,但难以捕捉长距离或者隐含的空间关联,进而导致错误定位。

\subsubsection{多模态对齐与语义鸿沟}
视觉与文本模态的语义对齐,是VQA模型能够顺利进行空间推理的基础。但是存在跨模态特征映射的困难以及稀疏空间关系的问题。
目前,有一些研究者试图解决这些问题。在跨模态特征映射这一问题上,Costanzino\cite{Costanzino2024MultimodalIA}等人提出了一种使用轻量级多层感知器(MLPs)的方法,训练两个映射函数,基于正常样本预测一个模态的特征,通过比较实际特征和预测特征的不一致性
来检测工业场景中的异常。在解决稀疏空间关系的问题方面,Zhang\cite{wu2024minds}等人提出了VoT模型,通过VoT提示来提升LLMs的空间推理能力,通过生成视觉表示增强了模型对。
这些方案在一定程度上解决了多模态对齐和语义鸿沟问题,但仍存在一定局限性。具体而言,多数模型通过全局或局部注意力加权融合多模态特征,但未显式建模空间关系的层次性。
另外,一些模型通过子任务分解实现推理,但对空间逻辑的组合泛化能力有限。

\subsubsection{动态场景与实时推理的挑战}
动态场景(如移动物体的避障路径规划)要求模型实时更新空间状态,但现有方法存在增量式推理不足和数据驱动的固有局限。具体而言,
现有方法往往依赖于离线训练的模型,无法实时更新空间状态,导致推理结果复杂空间推理任务滞后,无法满足实时推理的需求。例如,符号推理(如ASP)需重新求解完整的逻辑程序,导致延迟较高\cite{}。
另外像一些基于监督学习的模型,依赖于静态数据集(如CLEVR),无法适应动态环境中的连续变化。最新的一些研究成果,如微软提出的MVoT框架通过生成可视化中间推理步骤,
实现了结合文本与图像信息背景下,对空间关系表示的动态调整,且在复杂场景中比传统思维链(Chain of Thought, CoT)的稳健性提升20\%。

\subsubsection{数据集偏见与评估瓶颈}
现有VQA数据集(如CLEVR、VQAv2)存在显著偏差,如问题答案分布不均、问题类型单一等,导致模型在特定问题上表现优异,但在真实场景下泛化能力不足。
Shrestha\cite{shrestha2019answer}等人指出,像CLEVR这一类合成的数据集,虽然能够测试多步逻辑,但其几何简单性无法反映真实世界的复杂性,并且在问题设计上
也隐含对特定答案的倾向性,如“左侧”常与特定物体绑定,进而导致模型依赖表面统计规律,而并非真实推理。目前,已有一些研究者使用零样本学习的方法,通过引入
从未见过的问题-答案组合,测试模型的泛化能力。例如,Yang\cite{yang-etal-2022-zero}等人提出了一种零样本学习的方法,通过引入从未见过的问题-答案组合,测试模型的泛化能力。
此外,新的一些VQA数据集如VSR,通过控制答案分布来减少语言先验,以测试模型的纯视觉推理能力。

\subsubsection{可解释性与鲁棒性不足}
空间推理需要透明化的推理过程,以支持安全验证和决策解释。然而,现有的方法存在黑箱和对抗脆弱性等问题,无法提供可解释性保障。
目前,有一些研究者试图在这一方面进行改进,如通过可视化路径的方式,提高模型的可解释性。
例如,Li\cite{li2025imagine}等人提出了多模态思维可视化框架(MVoT),旨在通过生成推理轨迹的图像可视化,增强多模态大语言模型(MLLMs)在复杂空间推理任务中的表现。
该方法通过在自回归 MLLMs 中引入标记差异损失,显著提高了视觉连贯性和保真度。实验结果表明,MVoT在多个任务中表现出的性能出色,尤其是在思维链(Chain of Thought,CoT)
表现很差的场景中,展现出了显著的改进。Shah\cite{shah2019cycle}等人提出了一种基于循环一致性的训练框架,旨在增强VQA模型对语言变化的鲁棒性。
该方法通过双向训练和循环一致性,提高了模型对语言变化的适应性,从而提高了模型的鲁棒性和泛化能力。

\section{研究目标与内容}
本课题的主要研究目标是研究并设计一种新的神经符号框架,通过实验,证明该框架能够有效提升大语言模型在复杂空间推理问题上的性能。
最终将该神经符号框架接入视觉问答系统。

本文的主要研究内容包括以下三个方面:

\begin{enumerate}[label=(\arabic*),itemsep=0pt,parsep=0pt]
    \item 在参考借鉴CLEVR数据集的基础上,在图像中的物体交互、图像遮挡和部分可见性、图像噪声干扰等方面进行扩展,
构建一个新的视觉问答数据集,以更好地评估视觉问答模型在解决复杂空间推理问题上的能力
    \item 研究融合大语言模型及ASP的神经符号流水线,并使用DSPy来进行实现。通过DSPy,将ASP求解器与大语言模型实现系统集成。
在本文构建的视觉问答数据集上,进行对比实验验证。此外,为了评估,还进行了消融实验。
    \item 以本文中所提出的神经符号框架为核心,设计并实现一个视觉问答系统。
\end{enumerate}

\section{研究方法与技术路线}
本文针对以上研究目标和研究内容,综合多种方法进行研究。本文的研究主要涉及三个重点内容:
\begin{enumerate}[label=(\arabic*),itemsep=0pt,parsep=0pt]
    \item 对于视觉问答数据集的构建,首先,用案例分析法研究CLEVR数据集,分析其特点和不足,为新数据集的构建提供理论依据和设计方向。
然后采用文献研究法,调研现有的视觉问答数据集,学习它们在数据集构造过程中的方法和策略。
    \item 对于面向空间推理领域的神经符号框架的研究,首先,分析现有神经符号方法在空间推理方面的优势和不足。
其次,进行模型构建。最后,通过实验验证,进行对比实验和消融实验,评估设计的神经符号框架的性能。
    \item 对于视觉问答系统的设计与实现,首先,进行需求分析,明确系统的功能和性能要求。其次,设计视觉问答系统的架构和模块划分,确定各模块实现所需的技术方案。
再次,基于前文的研究成果,实现视觉问答系统的各个模块,并进行系统集成和测试。最后,通过实验验证,评估视觉问答系统的性能和可用性。
\end{enumerate}

具体的技术路线如图\ref{roadmap}所示。

\begin{figure}
    \centering
    \includegraphics[width=\textwidth]{process.png}
    \caption{技术路线图\label{roadmap}}
\end{figure}

\section{论文结构}
本文共分为六个章节,各章节的主要内容具体如下:

第一章为绪论,总体介绍本文的研究背景及意义、相关研究现状与不足、研究目标
与研究内容、研究方法与技术路线及本文的结构安排。

第二章为背景知识,对本文涉及到的主要技术进行介绍,具体包括ASP程序语法及ASP求解器、
GLIP以及DSPy。

第三章为数据集构建。对本文所用的视觉问答数据集进行详细介绍,包括数据集的构造目的、数据集的研究方向、数据集的设计流程以及对数据集质量的验证。

第四章为神经符号框架的研究设计。对本文设计的神经符号框架进行详细介绍,包括流水线总体架构、视觉场景理解、语义解析、知识蒸馏、迭代反馈和规则修正、ASP推理等模块的设计。

第五章为实验及结果分析。通过一系列实验,验证本文所提出的神经符号框架对复杂空间推理任务的提升效果,以及其在不同大语言模型架构上的泛化能力。

第六章中对本文工作加以总结,分析本文的创新点和不足之处,并对未来的研究方向进行展望。


\input{chapter/Background.tex}

\chapter{数据集}
对模型的空间推理能力进行充分评估,高质量的数据集必不可少。本章将详细介绍如何构造用于空间推理SRASP数据集,并对数据集的质量进行测试。

\section{设计目标}
尽管CLEVR数据集至今仍被广泛应用于多模态模型的空间推理能力研究,但其存在着一些较为明显的问题,其中比较凸出的是,CLEVR数据集中的场景是完全可观察的,模型可以直接从图像中获取所有必要信息。现实世界中,有很多信息并不能直接从图像中获取。人类基于图像对问题进行回答时,除了直接从图像中观察到的信息,往往也需要使用已经积累的先验知识。本文在此对空间推理的问题进行如下定义:如果回答该问题所需要的全部知识均已包含在图像中,即回答问题不需要使用先验知识,那么称该问题为完全可观察问题。同理,如果图像中并不包括回答该问题所需的全部知识,也即回答该问题需要使用已经积累的先验知识,那么称该问题为不完全可观察问题。

为了解决这一问题,本章基于CLEVR数据集,构建一个名为SRASP的数据集。该数据集的主要设计目标聚焦于以下几个方面:
\begin{enumerate}[label=(\arabic*),itemsep=0pt,parsep=0pt]
\item 考察模型对不完全可观察问题的解答能力。通过引入部分可观察性,模拟现实世界中物体被遮挡或隐藏的场景,要求模型在不完整的视觉信息下进行推理,考察模型真正解决问题的能力。
\item 考察模型对推理密集型问题的解答能力。现实世界的同一场景中,往往存在多个物体,物体之间的关系多样,解决空间推理问题需要多个步骤,逐步解决。通过增大解决问题所需的推理跳数,可以确保模型无法通过简单的模式匹配或者直接观察得出答案,进而真正考察模型的思考和推理能力。
\item 考察模型的知识整合能力。正如在数学证明中定理之间相互印证产生新的定理一样,回答问题所需要的知识,往往也是需要将现有知识进行结合产生新知识,才能最终解决问题。在SRASP数据集中,为每个场景提供一组逻辑约束作为背景知识,模型需将这些先验知识与观察到的视觉信息相结合,以生成正确的答案。
\end{enumerate}

\section{物体基本属性与约束}
与CLEVR数据集中的几何体一样,SRASP数据集的图像中的每个物体,均有形状、尺寸、材质、颜色四种属性。每种属性的可能取值如下:
\begin{enumerate}[label=(\arabic*),itemsep=0pt,parsep=0pt]
    \item 形状:圆锥体、球体、圆柱体和立方体。
    \item 尺寸:小、中、大。
    \item 材质:橡胶、金属。
    \item 颜色:红色、蓝色、绿色、黄色、灰色、棕色、紫色、青色。
\end{enumerate}

除上述四种基本属性之外,图像中的物体也有“所在区域”这一属性,可取值为0、1、2、3。由于所有图像都被划分成
4个区域,故图像中物体也会处在某一个区域之中。为了研究问题方便,本文规定每个物体只能在图像的一个区域中,不能
同时跨多个区域。物体包含这一属性能够为指定约束提供便利。在本数据集中,约束是一组规则的集合,对场景生成
与推理而言密不可分,其主要作用包括:(1)限制物体的属性组合,如对颜色、形状等进行限制;
(2)定义物体之间的关联关系;(3)支持对遮挡物体的推理,当物体被遮挡时,通过约束可以缩小潜在答案的范围。

基于对物体的应用范围,约束可以划分为以下三类:
\begin{enumerate}[itemsep=0pt,parsep=0pt]
    \item 区域约束:仅作用于特定区域的局部规则。例如,“区域1中所有物体形状必须为立方体”。
    \item 跨区域约束:涉及多个区域的全局规则。例如,“区域1和区域2中同颜色物体的总数不超过2个”。
    \item 全局约束:适用于整个场景的通用规则。例如,“所有物体必须属于至少一个区域”或“不允许存在完全相同的两个物体属性组合”。
\end{enumerate}

约束通过使用ASP来进行表示。例如:
\begin{lstlisting}
    :- object(X), at(X, 0), not hasProperty(X, shape, cube), not hasProperty(X, shape, cylinder).
\end{lstlisting}
表示如果X在区域0,那么它的形状必须是立方体或圆柱体。

\section{构造流程}
在确定图像中物体的基本属性之后,本文进一步确定了构建数据集的如下步骤流程:
\begin{enumerate}[itemsep=0pt,parsep=0pt]
\item 生成一组由约束定义的环境,记作$Environment_i$。
\item 生成一个完整的场景图$Complete_i$。该场景图完全符合上一步生成的环境$Environment_i$的要求。
\item 通过从完整场景图$Complete_i$中删除一个物体$Obj_i$,来生成一个部分场景图$Partial_i$。
\item 生成一个对于部分场景图$Partial_i$,对物体$Obj_i$的有关情况进行提问的问题$Q_i$。
\end{enumerate}

\section{环境表示定义}
SRASP中的环境是由一组约束来定义的。与场景相比,环境是一个更抽象一层的概念。针对某个特定的环境,
可以生成以其为模板的场景。可以理解为,环境是场景的抽象,场景是环境的实例。
每个约束决定了特定环境下物体的属性限制。本文首先设计了10个约束模板,所有模板使用ASP来进行表示。
每个环境最多通过20个不同的实例化后的模板来创建,也即每个环境中最多包含20个不同的约束。
部分约束模板的ASP编码表示以及对应表示含义见表\ref{tab:asp_templates}。最终,一共生成了50个
环境,数据集中的每个场景都是由其中一个环境进行实例化后生成的。环境的具体示例见附录\ref{appendix:environment}。

\begin{table}[!h]
    \centering
    \renewcommand{\arraystretch}{1.0}
    \begin{tabular}{|p{3cm}|p{12cm}|}
        \hline
        \textbf{模板} & \textbf{描述} \\
        \hline
        \textbf{模板1(取值约束)} & 
        \texttt{:- object(X), at(X, R), not hasProperty(X, P1, V1).} \\ 
        & 解释: 对区域R中的所有物体,它们P1属性的取值均为V1。 \\ 
        & 具体实现: :- object(X), at(X, 0), not hasProperty(X, color, red). \\
        \hline
        
        \textbf{模板2(否定约束)} & 
        \texttt{:- object(X), at(X, R), hasProperty(X, P1, V1).} \\ 
        & 解释:对区域R中的所有物体,它们的P1属性的取值,均不能为V1。 \\ 
        & 具体实现::- object(X), at(X, 0), hasProperty(X, material, metal). \\
        \hline
        
        \textbf{模板3(恰有N个约束)} & 
        \texttt{:- \#count\{X: hasProperty(X, P1, V1), object(X), at(X, R)\} != N.} \\ 
        & \textbf{解释}:在区域R中,恰好有N个物体的P1属性的取值为V1。 \\ 
        & 具体实现::- \#count\{X: hasProperty(X, size, small), object(X), at(X, R')\} != 2. \\
        \hline
        
        \textbf{模板4(至少有N个约束)} & 
        \texttt{:- \#count\{X1, X2: sameProperty(X1, X2, P1), object(X1), object(X2), at(X1, R1), at(X2, R2)\} < N.} \\ 
        & 解释:在区域R1和区域R2中,至少有N对物体,它们的P1属性的取值都是V1。 \\ 
        & 具体实现::- \#count\{X1, X2: sameProperty(X1, X2, shape), object(X1), object(X2), at(X1, 1), at(X2, 2)\} < 1. \\
        \hline
        
        \textbf{模板5(或约束)} & 
        \texttt{:- object(X), at(X, R), not hasProperty(X, P1, V1), not hasProperty(X, P1, V2).} \\ 
        & 解释:区域 R中的所有对象都具有属性 P1 的 V1 值或属性 P2 的 V2 值。 \\ 
        & 具体实现::- object(X), at(X, 1), not hasProperty(X, color, yellow), not hasProperty(X, color, blue). \\
        \hline
    \end{tabular}
    \caption{部分约束模板示例}
    \label{tab:asp_templates}
\end{table}
\section{场景表示定义}
场景是环境的实例。
SRASP数据集以场景图的形式表示场景,其节点表示使用其属性进行注释的对象,边表示对象之间的空间关系(前、后、左、右)。
在SRASP中,除了场景图表示之外,也用ASP对场景进行表示。
以下展示图\ref{}中部分场景的ASP​表示:
\begin{lstlisting}
%场景中的物体
object(0). object(1). object(2). object(3).

%物体的属性
at(0, 2).
hasProperty(0, color, green).
hasProperty(0, size, large).
hasProperty(0, material, rubber).
hasProperty(0, shape, cylinder).
....

%物体间的空间关系
front(1, 0). right(1, 0). ...
\end{lstlisting}

其中涉及到的谓词的功能如下:谓词\texttt{object}用于定义不同的物体(所有物体的名称用0,1等数字来表示)。
谓词\texttt{hasProperty(Object, Attribute, Value)}用于将对象的名为Attribute的属性的值设置为Value。
对象之间的空间关系用谓词\texttt{left}、\texttt{right}、\texttt{front}、\texttt{behind}来表示,例如
\texttt{left(A, B)}表示B位于A的左侧。
\section{图像生成}
图像生成基于前文中定义的场景图。场景图是一种对场景中物体、属性及其空间关系的结构化描述,而环境则由一系列约束条件所决定,
这些约束可能包括空间分布、物体间的相对关系以及特定属性的限制。
因此,场景图的生成问题可以归结为一个复杂的推理问题:在给定的环境约束(基于ASP的规则)以及场景中预期的物体数量$n$这两个前提下,完成以下任务:
\begin{enumerate}[itemsep=0pt,parsep=0pt]
\item \textbf{区域划分}:将每个物体合理分配到预定义的四个区域之一,确保满足诸如区域容纳量、相邻关系及物体间可能的干扰等约束;
\item \textbf{属性赋值}:为每个物体赋予颜色、尺寸、形状和材质等属性,其取值必须与环境中规定的约束条件一致。例如,某些区域可能只允许出现特定颜色或尺寸范围的物体;
\item \textbf{关系一致性}:在属性分配过程中,还需要确保各物体之间的关系(如邻近、对称或排斥关系)符合逻辑规则,从而保证场景图整体的合理性和一致性。
\end{enumerate}

为了解决上述推理问题,本文采用了ASP的方法。ASP作为一种声明性逻辑编程范式,特别适合解决复杂约束和组合优化问题。
在本系统中,ASP 求解器的工作流程主要包括以下几个步骤:
\begin{enumerate}[itemsep=0pt,parsep=0pt]
\item \textbf{约束建模}:将场景的环境约束以及物体属性赋值规则形式化为 ASP 规则。此步骤需要充分利用逻辑公式来描述物体的空间位置、属性取值范围以及各类关系约束;
\item \textbf{回答集计算}:在输入了物体数量$n$和所有相关约束之后,ASP 求解器会计算出满足所有约束条件的解集。每一个答案集代表一种物体属性及区域分配的合理配置,即一种可能的场景图;
\item \textbf{解集筛选与随机采样}:由于满足所有约束的配置方案可能数量巨大,为了使后续图像生成过程具有一定的随机性与代表性,系统从所有可能的解集中随机采样一百万个场景图。这一随机采样策略既保证了生成场景的多样性,也为后续的图像生成提供了充分的候选数据。
\end{enumerate}

在获得充分的场景图后,系统利用 Blender3 进行图像渲染。
Blender3 是一款高效且功能丰富的三维渲染软件,它能够基于场景图的结构信息生成逼真的图像。
渲染过程包括以下几个关键步骤:(1)场景搭建,根据场景图中各物体的位置信息及属性参数,在 Blender3 中自动搭建三维场景;
(2)光照与材质设置,对各物体的材质、光照、纹理等进行设置,确保渲染出的图像在视觉上具有真实感;(3)
图像生成,利用 Blender3 的渲染引擎,将搭建好的场景生成最终图像。

图\ref{pipeline_for_generating_environment}直观展示了从环境约束到场景图构建,再到图像生成的整个流水线过程,为后续工作的深入探讨提供了坚实的理论与实践基础。
\begin{figure}
    \centering
    \includegraphics[width=\textwidth]{pipeline_for_generating_environment.png}
    \caption{生成环境以及该环境中的完整场景的流水线}
    \label{pipeline_for_generating_environment}
\end{figure}
\section{问题表示定义}
在SRASP数据集中,每个问题均围绕部分场景中缺失物体的四个关键属性之一:颜色、大小、形状和材质。
问题的设计目标在于通过推理补全场景信息,从而考察系统在不完整场景下对物体属性关系的理解能力。
为此,本文将自然语言描述的问题转化为基于ASP的形式化表示,
使得问题的求解过程可以通过逻辑推理得到明确答案。

数据集中的每个问题最初以自然语言的形式提出,例如:“与中等大小的红色物体的材质相同的,另一个圆柱体的颜色是什么?”
为了使问题具有可操作性,本文设计了对应的ASP编码,如下所示:
\begin{lstlisting}
    query(Q) :- hasProperty(X, color, Q),
    hasProperty(X, shape, cylinder),

    hasProperty(Y, size, medium),
    hasProperty(Y, color, red),
    same_material(Y, X),
    X != Y.
\end{lstlisting}
在该编码中,\texttt{query(Q)}表示需要推导出满足条件的物体X的颜色Q。
此外,通过多个\texttt{hasProperty}和\texttt{same\_material}条件,明确限定了参与推理的物体之间的属性关系,并利用 X != Y 排除自反情况。
这种表示方式不仅使问题的语义精确、结构清晰,而且便于通过 ASP 求解器进行自动求解,从而为数据集中的问题构建提供统一、标准的表达形式。

在问题生成过程中,必须对每个问题的答案范围进行严格控制。对于涉及属性$A$(其中
$A \in \{ color, size, material, shape\}$)的问题,其可能的解集$S$的大小满足$1 \leq |S| \leq |A|$。
其中,$|A|$表示属性$A$所有可能取值的数目。例如,对于尺寸属性,若其可能的取值集合
$\{ large, medium, small\}$,则$|size| = 3$。

如果某个问题生成的解集数量恰好为$|A|$,例如问题答案为“尺寸可以为 large、medium 或 small”,则该问题在特定场景下并未起到区分或推断作用,因此被判定为无效。这一设计思路确保了数据集中每个问题在解答上都具有针对性和挑战性,从而避免生成普遍适用的、无区分价值的答案。
\section{问题生成}
在SRASP数据集中,每个问题均围绕部分场景中缺失物体的某一关键属性展开,如颜色、大小、形状或材质。
为此,本文设计了一套模板,用以指导自然语言问题的生成。以下为问题模板样例:
\begin{lstlisting}
What shape is the < Z2 > (size) < C2 > (color) < M2 > (material) [that is] 
< R > (relation) the < Z > (size) < C > (color) < M > (material) < S > (shape) ?
\end{lstlisting}
其中,<Z2>、<C2>、<M2> 表示待查询对象的已知属性(例如尺寸、颜色、材质),由随机策略从完整场景图中选取;
<R> 为空间关系(如left、right、front、behind),其取值既满足随机性,又依赖于完整场景中物体间的真实空间分布;
<Z>、<C>、<M>、<S> 则代表参考对象的属性,通过对完整场景图中与查询对象具有特定空间关系的候选对象进行筛选而确定。
这种模板化设计不仅使自然语言问题的结构化描述成为可能,而且便于后续转换为ASP的形式化表示,从而实现问题求解的自动化。

问题模板的实例化过程基于与图像对应的完整场景图。具体步骤包括:
\begin{enumerate}[itemsep=0pt,parsep=0pt]
\item 场景图构建与部分场景生成。利用完整场景图构建方法,将真实场景中的物体、属性以及空间关系进行抽象建模;从完整场景中随机移除一个物体,以构造部分场景图,此移除的对象即为“查询对象”,其缺失的属性将作为问题求解目标。
\item 已知属性的随机选取。对于查询对象,模板中的已知属性(例如<Z2>、<C2>、<M2>)由随机采样策略确定,确保不同问题之间在属性分布上具有较好的随机性和代表性;同时,选取的属性应满足数据集整体的“问题类型平衡”要求,避免某一属性出现频率过高或过低。
\item 空间关系的确定。对于模板中表示空间关系的部分(<R>),取值虽然随机,但参考对象的选择依赖于完整场景图中的物体空间布局。具体而言,从与查询对象存在 <R> 关系的物体中进行候选对象的筛选,从而确保问题中提及的空间关系具有实际语义意义。
\end{enumerate}

为了使问题具有可操作性和求解性,所有生成的问题均转化为ASP形式。转换过程中包括以下步骤:
\begin{enumerate}
\item 规则构建。将自然语言问题中的各项约束(属性约束、空间关系约束、对象排他性约束等)以ASP规则的形式表达;
\item 约束整合。同时将部分场景图与环境约束作为求解器的输入,确保问题求解过程在完整逻辑下进行;
\item 解集判定。由ASP求解器计算出问题的所有可能解。针对每个属性$A \in \{ color, size, material, shape\}$
,若生成的解集$S$满足$1 \leq |S| \leq |A|$,则问题被认为具有合理的答案区分性;若解集规模等于$|A|$
说明答案涵盖了属性的全部取值,缺乏针对性,此时问题将被判定为无效并从数据集中剔除。
\end{enumerate}

图\ref{pipeline_for_generating_partial}展示了从完整场景图构建、部分场景生成、问题模板实例化、ASP表示转换,到最终问题求解的流水线过程。
该流程通过随机采样,使生成的问题在属性组合和空间关系上呈现多样性;通过对解集规模的判定机制,
有效过滤掉普遍适用或无区分意义的问题,确保每个问题都能够准确反映场景中物体属性的关系;
使用模板化设计和ASP表示为后续数据集扩充与新问题类型的加入提供了灵活性和统一标准。
\begin{figure}
    \centering
    \includegraphics[width=\textwidth]{pipeline_for_generating_partial.png}
    \caption{生成部分场景和问题,并进行标记的流程}
    \label{pipeline_for_generating_partial}
\end{figure}
\section{数据集分析}
\subsection{统计分析}
问题模板的统计分布及问题数量在5到9之间的查询属性分布统计见图\ref{fig:template_statistics}。
本数据集是基于CLEVR数据集进行生成的,在问题模板方面,采用了CLEVR数据集中的六种问题模板。此外,也展示
了特定类型的问题在不同场景物体数量下的分布情况。
\begin{figure}
    \includegraphics[width=\textwidth]{figures/template_combined-crop.pdf}
    \caption{问题模板统计及问题数量在5到9之间的查询属性分布统计}
    \label{fig:template_statistics}
\end{figure}

问题分布的统计图见图\ref{fig:question_statistics}。从统计图中可得知,有关颜色和形状的问题在SRASP数据集中
占比最高,分别是39\%和37.6\%,关于大小和材质的问题则相对较少,分别只占到了13.1\%和10.3\%。
此外,统计图中也展示了不同类型问题的答案集分布。在生成数据集的过程中尽量实现均衡分布,避免多数问题
指向相同答案集的情况。例如,当问题涉及物体尺寸时,其潜在解可能为\{大、中\}、\{大、小\}、\{小、中\}、\{大\}、\{中\}或\{小\}。
\begin{figure}
    \includegraphics[width=\textwidth]{figures/combined_statistics-crop.pdf}
    \caption{问题分布统计图}
    \label{fig:question_statistics}
\end{figure}

答案的分布统计。
\subsection{质量分析}
\subsubsection{质量保证}
为了保证数据集的质量,采用双盲审核机制,邀请两名评审员各自独立对数据集进行评审,验证每个问题
的答案、推理解答所需步数及问题是否可观察。在评审过程中,如果两名评审员同时判定该问题出现错误,那么
将该问题剔除。双盲评审的结果见表\ref{tab:kappa},其中展示了评审员1和评审员2分别与初始标注的一致性,以及两名评审员
之间的一致性。通过Fleiss's Kappa衡量的评审员间的一致性超过0.8,证明了数据集的可靠性。
\begin{table}[h]
    \centering
    \renewcommand{\arraystretch}{0.8}
    \begin{tabular}{lccc}
    \toprule
     & \makecell{答案是否正确} & \makecell{推理所需步数} & \makecell{问题是否可观察}\\
    \midrule
    初始标注与评审员1 & 81.2 & 84.4 & 89.6 \\
    初始标注与评审员2 & 84.2 & 85.6 & 85.4 \\
    评审员1与评审员2 & 80.1 & 83.8 & 86.9 \\
    \midrule
    平均值 & 81.8 & 84.6 & 87.3 \\
    \bottomrule
    \end{tabular}
    \label{tab:kappa}
    \caption{评审员1、2和初始标注之间的标注者间一致性}
\end{table}
\subsubsection{难度保证}
本文通过记录2位评审员在SRASP数据集上回答问题时的正确率、以及所需检索信息的次数、回答问题所需要的跳数,来对
数据集的难度进行判断,并于其它现有的VQA数据集进行比较。实验结果见表\ref{tab:human_performance},其中明显可以看出,SRASP回答问题所需
的跳数,明显大于现有数据集VQAv2,另外SRASP所需的检索信息的次数也更多一些,这些都证明了回答SRASP数据集
所需的外部知识更多,考虑次数更多,数据集难度更大。另外,人类在本文
构造的SRASP数据集上的准确率最低,进一步侧面印证了本文构造数据集的挑战性。
\begin{table}[h]
    \centering
    \renewcommand{\arraystretch}{0.8}
    \begin{tabular}{lccc}
    \toprule
     & \makecell{回答问题正确率} & \makecell{推理所需跳数} & \makecell{检索信息次数}\\
    \midrule
    SRASP & 81.2 & 84.4 & 89.6 \\
    CLEVR & 84.2 & 85.6 & 85.4 \\
    GQAv2 & 80.1 & 83.8 & 86.9 \\
    \midrule
    平均值 & 81.8 & 84.6 & 87.3 \\
    \bottomrule
    \end{tabular}
    \label{tab:human_performance}
    \caption{人类评审员在不同VQA数据集上回答问题的表现}
\end{table}
\section{本章小结}
本章介绍了SRASP数据集的构造过程,重点描述了如何基于完整场景图生成部分场景图,
并通过 ASP 实例化问题模板以确保问题的多样性和合理性。

首先,本章阐述了对象移除的原则,即如何选择查询对象以及如何保证其属性在问题类型上的均衡性。
随后,详细说明了查询模板的填充策略,包括查询属性的选取、参考对象的确定以及空间关系的合理性,
以确保生成的问题符合实际场景。最后,本章介绍了 ASP 求解器在问题生成过程中的作用,
即利用 ASP 规则对部分场景进行推理,以确定查询属性的可能取值范围,从而生成符合逻辑约束的高质量视觉问答数据。

通过上述方法,SRASP数据集不仅保证了问题的可解释性,还增强了对复杂空间关系的推理能力,
为视觉问答任务提供了更具挑战性的数据支持。

\chapter{神经符号框架的架构设计与实现}
\section{引言}
本章对神经符号框架的架构进行详细的介绍,包括流水线总体架构、视觉场景理解、语义解析、知识蒸馏、迭代反馈和规则修正、ASP推理等模块的设计和实现。
框架的目标如下:
\begin{enumerate}[label=(\arabic*),itemsep=0pt,parsep=0pt]
\item 增强VQA系统在复杂空间关系推理方面的能力。
\item 使用Dspy来完成对LLM的提示、优化,以降低神经符号方法的开发难度,并增强其可扩展性。
\end{enumerate}
\section{框架总体架构}
框架的总体架构如图所示。

LLM在整个框架中的作用有以下两点:
\begin{enumerate}[label=(\arabic*),itemsep=0pt,parsep=0pt]
    \item 在语义解析模块中,对自然语言问题进行语义解析,将问题以ASP程序的形式表示出来。
    \item 在迭代反馈模块中,对ASP表示进行多次迭代优化,其中包括:
添加规则,对规则进行一致性检查以及整合错误信息的反馈。
\end{enumerate}

本文在语义解析模块和迭代反馈模块中,分别采用不同的LLM,各自进行微调,以更好地满足不同任务的需求。
\section{视觉场景理解}
视觉场景理解是整个流水线的重要任务之一,其目标是从输入图像中提取结构化的信息,包括物体的属性(形状、颜色、大小等)、位置以及物体之间的空间关系。
这些信息随后以ASP程序来表示,为使用ASP求解器进行推理提供基础。本文选择使用GLIP模型来完成目标检测任务,以实现高效且准确的目标检测和定位。
下面本文
\subsection{目标检测}
目标检测是视觉场景理解的最基础的任务,其目标是从图像中识别出所有相关物体,并标注其边界框、类别和属性。本文选择GLIP模型作为目标检测的实现工具,原因在于
其独特的语言-视觉预训练特性。基于大规模的图文对数据,GLIP进行大量预训练,能够根据自然语言描述(如“红色立方体”)直接定位图像中的对应物体。
这种能力特别适合VQA任务,使问题中的语言信息与图像内容高效对齐成为可能。

具体到实现中,本文首先对输入图像进行预处理,将其调整为GLIP模型的输入分辨率(例如,800×1333像素)。
随后,构造一组文本提示,以覆盖图像中可能出现的物体。考虑到本文第三章构造的数据集,本文使用以下短语集合:
\begin{enumerate}[label=(\arabic*),itemsep=0pt,parsep=0pt]
    \item “大物体”、“小物体”等,描述物体的大小。
    \item “红色立方体”、“蓝色球体”、“绿色圆柱体”等,涵盖所有颜色和形状的组合。
\end{enumerate}
GLIP接收这些文本提示和图像作为输入,输出每个物体的边界框及其对应的类别标签。例如,对于一张包含“红色立方体”和“蓝色球体”的图像,GLIP的输出可能如下:
\begin{enumerate}
    \item 物体1:类别=“红色立方体”,边界框=($x_1$, $y_1$, $x_2$, $y_2$)。
    \item 物体2:类别=“蓝色球体”,边界框=($x_3$, $y_3$, $x_4$, $y_4$)。
\end{enumerate}

为了进一步提取物体的属性(如颜色、形状和大小),本文对GLIP的类别标签进行解析,将其分解为单独的属性值。例如,“红色立方体”被分解成color=red、shape=cube。
此外,通过边界框的面积(即$(x_2-x_1)\times (y_2-y_1)$),本文进一步推断物体的大小属性,设定阈值以区分“大”和“小”物体。
\subsection{空间位置提取}
在完成目标检测后,需要确定每个物体的空间位置,以便为空间关系推理提供依据。
GLIP提供的边界框信息为位置提取提供了基础。本文使用以下方法计算物体的空间位置:
\begin{enumerate}[label=(\arabic*),itemsep=0pt,parsep=0pt]
    \item 中心点坐标:对于每个物体,计算其边界框的中心点坐标($x_c, y_c$),其中:
    $$x_c = \frac{x_1 + x_2}{2}, y_c \frac{y_1+y_2}{2}$$
    该坐标表示物体在二维图像中的位置。
    \item 深度信息:通过图像的深度信息进一步推断物体的三维位置($x$,$y$,$z$)。由于
本文生成的数据集时使用的为Blender渲染引擎,故可直接从渲染引擎中获取深度信息。
\end{enumerate}

最终,每个物体的位置信息被表示为ASP事实,例如:position(obj1, x1c, y1c, z1) 表示物体1的中心点位置。

\subsection{空间关系提取}
空间关系是复杂空间推理的核心,例如“左边”、“前面”、“遮挡”等关系。本文通过以下步骤从图像中提取这些关系:
\begin{enumerate}[label=(\arabic*),itemsep=0pt,parsep=0pt]
    \item 二维空间关系:取二维平面中,物体的中心点坐标。对于两物体之间的距离,通过欧几里得距离公式进行计算,用于判断“靠近”等关系。
对于“左右”关系,若物体A的x\_c坐标小于物体B的x\_c坐标,则认为A在B的左边,记作left(objA, objB)。
同理,对于“上下”关系,若物体A的y\_c坐标小于物体B的y\_c坐标,则认为A在B的上方,记作above(objA, objB)。
    \item 三位空间关系:对三维空间中的物体A和物体B,若A的z值小于B的z值,则认为A在B的前面,记作in\_front\_of(objA, objB)。
    \item 遮挡关系:遮挡关系的判定需要边界框信息以及深度信息。
若物体A的边界框与物体B的边界框重叠,并且A的z值小于B的z值,则A遮挡B,记作
occludes(objA, objB)。
\end{enumerate}

最终,所有提取的空间关系都被转化为ASP事实,例如:
left(obj1, obj2) 表示物体1在物体2的左边,in\_front\_of(obj1, obj2) 表示物体1在物体2的前面;
occludes(obj1, obj2) 表示物体1遮挡了物体2。
\subsection{场景图生成}
为了将提取的物体属性、位置和空间关系整合为一个统一的结构化表示,本文生成场景图。场景图是一种图结构,其中:
\begin{enumerate}
    \item 节点表示物体,带有属性标签(如color=red,shape=cube)。
    \item 边表示物体之间的空间关系(如left、in\_front\_of)。
\end{enumerate}

场景图的生成过程中,首先为每个检测到的物体创建一个节点,并标注其属性和位置。
随后,根据提取的空间关系,在节点之间添加有向边,例如从obj1到obj2添加边left\_of。
场景图随后被转化为ASP事实,以供后续环节使用。
\subsection{实现细节与实例}
为展示视觉场景理解模块的实际效果,本文提供一个具体示例。假设输入图像包含以下场景:
\begin{enumerate}
\item 一个红色大立方体位于图像左侧;
\item 一个蓝色小球体位于图像右侧,且在红色立方体前面。
\end{enumerate}

通过GLIP,检测结果如下:
\begin{enumerate}
\item 物体1:类别=“红色立方体”,边界框=(50, 100, 150, 200);
\item 物体2:类别=“蓝色球体”,边界框=(250, 50, 350, 150)。
\end{enumerate}

中心点计算:
\begin{enumerate}
    \item 物体1:(x\_c, y\_c) = (100, 150);
    \item 物体2:(x\_c, y\_c) = (300, 100)。
\end{enumerate}

假设深度信息显示物体2的z值=75,小于物体1的z值=50,则空间关系提取如下:
\begin{enumerate}
\item left\_of(obj1, obj2)(因为100 < 300);
\item above(obj2, obj1)(因为100 < 150);
\item in\_front\_of(obj2, obj1)(因为75 < 50)。
\end{enumerate}

最终生成的ASP事实为:
\begin{lstlisting}
color(obj1, red).
shape(obj1, cube).
size(obj1, large).
position(obj1, 100, 150, 50).

color(obj2, blue).
shape(obj2, sphere).
size(obj2, small).
position(obj2, 300, 100, 75).

left_of(obj1, obj2).
above(obj2, obj1).
in_front_of(obj2, obj1).
\end{lstlisting}
\subsection{技术挑战与解决方案}
在实现过程中,遇到了以下几点技术上的难题,本文提出了相应的解决方案:
\begin{enumerate}[label=(\arabic*),itemsep=0pt,parsep=0pt]
\item GLIP的准确性:GLIP可能对某些复杂场景(如物体密集或遮挡严重)产生误检。为解决此问题,本文在GLIP的基础上引入后处理步骤,通过非极大值抑制(NMS)去除重复检测,并结合深度信息过滤误检。
\item 空间关系的鲁棒性:二维空间关系的提取可能因视角变化而失效。为此,本文优先利用深度信息,并在缺乏深度信息时使用几何约束(如边界框重叠面积)提高鲁棒性。
\item 计算效率:GLIP的推理速度可能限制流水线的实时性。本文通过批处理图像和优化提示设计,显著减少了推理时间。
\end{enumerate}
\subsection{小结}
通过GLIP进行目标检测,并结合空间关系提取和场景图生成,成功将输入图像转化为结构化的ASP事实。这一模块为后续的语义解析和神经符号推理
提供了可靠的基础数据。下一节将介绍如何将自然语言问题解析为ASP查询,从而与视觉场景理解的输出无缝衔接。
\section{语义解析}
语义解析的主要任务是,通过LLM,使用上下文学习的方法,将自然语言问题转为用ASP进行表示,以便与视觉场景理解提取的场景事实结合进行逻辑推理。

直观上,LLM可能很难直接解决复杂的推理问题。然而,LLM已经在理解文本输入并将其转化为形式化程序方面取得了巨大成功,例如程序代码\cite{gao2023pal}和数学方程\cite{he2023solving}。
接下来,本文将介绍通过微调LLM,根据自然语言问题,生成正确的ASP程序。

\subsection{ASP模板设计}
根据第三章构造的数据集,本文设计了一组ASP模板,以覆盖数据集中可能出现的问题类型。以下将针对各问题类型介绍相应的ASP模板设计。
\subsubsection{基础存在性问题}
基础存在性问题是最简单的问题类型,其形式为“是否存在一个满足条件的物体”。本文设计了以下ASP模板:
\begin{lstlisting}
模板:是否存在一个[颜色][材质][形状]的物体?
提示:是否存在红色金属立方体?
编码:exists :- object(ID, red, metal, cube, _, _, _).
\end{lstlisting}
\subsubsection{三维空间关系问题}
\begin{lstlisting}
模板:物体A([属性])是否在物体B([属性])的[方位]方,且两者在Z轴上[关系]?
提示:红色球是否在蓝色立方体的左上方?
编码:left_above(O1, O2) :- object(O1, red, _, ball, X1, Y1, Z1), object(O2, blue, _, cube, X2, Y2, Z2), X1 < X2, Z1 > Z2 + 10.
exists :- left_above(O1, O2).
\end{lstlisting}
\subsubsection{多跳推理问题}
多跳推理的跳数的取值范围为2-5,主要考察模型的推理能力。本文对此设计了以下ASP模板:
\begin{lstlisting}
模板:若[物体A属性]在[物体B属性]的[方位1],且[物体B属性]在[物体C属性]的[方位2],那么[物体A属性]相对于[物体C属性]的位置是什么?
提示:若红色球在蓝色立方体左边,且蓝色立方体在绿色圆柱体前面,那么红色球相对于绿色圆柱体的位置?
编码:transitive_left_front(O1, O3) :- left(O1, O2), front(O2, O3).
final_relation(O1, O3) :- transitive_left_front(O1, O3), object(O1, red, _, ball, _, _, _), object(O3, green, _, cylinder, _, _, _).
\end{lstlisting}
\subsubsection{多参考系问题}
\begin{lstlisting}
模板:以[物体属性]为参照物,[目标物体属性]位于其哪个方向?
提示:以蓝色立方体为参照物,红色球是否在其右后方?
编码:local_right_behind(Target, Ref) :- object(Ref, blue, _, cube, Xr, Yr, Zr), object(Target, red, _, ball, Xt, Yt, Zt), Xt > Xr, Zt < Zr.
exists :- local_right_behind(Target, Ref).
\end{lstlisting}
\subsubsection{动态反事实问题}
\begin{lstlisting}
模板:如果移除[物体属性],那么[某条件]是否成立?
提示:如果移除所有红色物体,是否还存在比蓝色立方体大的球?
编码:hypothetical_world(ID) :- object(ID, _, _, _, _, _, _), not (object(ID, red, _, _, _, _, _), removed(ID)).
hypothetical_condition :- hypothetical_world(ID1), object(ID1, _, _, ball, _, _, S1), object(ID2, blue, _, cube, _, _, S2), S1 > S2.
\end{lstlisting}
\subsubsection{对抗性样本问题}
\begin{lstlisting}
模板:图中是否有[数量]个[属性]物体?注意[干扰条件描述]。
提示:是否有3个红色球?注意反光物体可能是玻璃材质而非球体。
编码:valid_ball(ID) :- object(ID, red, glass, ball, _, _, _), not material(ID, metallic). 
count(N) :- N = #count{ ID : valid_ball(ID) }.
\end{lstlisting}

\subsection{ASP查询生成}
为了对语义解析模块的专用LLM进行微调,以更好地生成ASP查询,本文首先根据上述设计的ASP模板,生成训练数据,再对该LLM进行训练。生成的数据集共包括
10万条训练数据,各类型占比见表。数据生成完成后,采用Clingo求解器。

目前,工业界和学术界已有的LLM有很多,如GPT、Copilot、Gemini等。为了选择最适合的LLM,本文首先对主流LLM在生成ASP查询这一方面进行评估。
LLM的参数大小、架构及训练数据源见表\ref{tab:llm-comparison}。从表\ref{tab:llm-comparison}中,不难看出Gemma 2B是在这一组LLM中最小的模型。

\begin{table}[ht]
    \centering
    \begin{tabular}{lccc}
        \toprule
        \textbf{模型} & \textbf{参数量} & \textbf{架构} & \textbf{训练数据源} \\
        \midrule
        ChatGPT 3.5    & 175B      & 编码器-解码器架构            & 网络数据         \\
        Copilot        & 1.5T      & 编码器-解码器架构            & Github仓库    \\
        Gemini         & 1T        & 混合专家模型            & 文档,书籍,代码      \\
        Gemma          & 2B--7B    & 纯解码器         & 文档,数学,代码      \\
        LLaMa2         & 7B--13B   & 纯解码器         & 网络数据         \\
        LLaMa3         & 7B--13B   & 纯解码器 + 分组查询注意力机制   & 网络数据         \\
        Mistral        & 7B--141B  & 纯解码器 + 分组查询注意力机制   & 网络数据         \\
        \bottomrule
    \end{tabular}
    \caption{LLM对比详细信息,其中参数量以十亿(B)或者万亿(T)为单位。}
    \label{tab:llm-comparison}
\end{table}

最终获得的数据集包括10万条训练数据,将数据集按照8:2的比例划分成训练集和验证集,并保持每个问题类别的数据分布比例保持一致。

为了能够进行高效微调,最终选择参数量最小的Gemma 2B作为语义解析模块的LLM。Gemma 2B模型采用了分组查询注意力(GQA)机制,将查询头划分为2组共享键值投影,相比传统多头注意力(MHA)减少25\%内存占用。
同时引入局部滑动窗口注意力(窗口大小4096)与全局注意力交替层,平衡长程依赖建模与计算效率。

为了验证LLM生成ASP查询的正确性,此处通过Python API调用ASP求解器Clingo。Clingo,从而定义一个函数$f(P) = AS(P)$,其中$AS(P)$表示程序P的所有回答集(可能为空)。
对于LLM根据提示$x$生成的ASP程序$y ~ P_L(|x)$,以及与$x$对应的能够真实表示问题的ASP程序$y^*$,按照以下步骤来进行验证:
\begin{enumerate}
\item 事实集合构建:构造一组事实,用集合$F_{y^*}$表示,其代表了问题$x$。
\item 程序合成。将$F_{y^*}$与$y$合并,得到新的ASP程序$P = y \cup F_{y^*}$。同理,将$y^*$与$F_{y^*}$合并,得到新的ASP程序$P^* = y^* \cup F_{y^*}$。
\item 语法命中:调用Clingo计算$f(P)$,若未发生解析错误,则判定为语法命中。
\item 语义验证:进一步计算$f(P)$与$f(P^*)$,分别得到$AS(P)$与$AS(P^*)$。若$AS(P)$与$AS(P^*)$完全匹配,则判定为语义命中。
\end{enumerate}

此后,基于DSPy框架发起对LLM的调用,将自然语言问题转化为ASP查询。此处使用DSPy的Template组件定义提示模板,指导LLM如何生成指定格式的ASP程序。其中,也使用了Example组件,
用于构建优化过程中使用的示例数据,封装输入和输出的对应关系,以帮助LLM进行学习。

\subsection{实验与结果分析}
为检测语义解析模块的性能,本文使用上节构造的数据集进行实验。考核指标选取上一节定义的语法命中率和语义命中率。
将未经过训练的LLM与本文经微调后的Gemma 2B模型进行对比。

实验结果如表及图所示。语法正确并不能保证语义正确。例如,Gemma 7B实现了45\%的语法正确率,但是
语义正确率明显偏低。根据实验结果分析,没有任何模型能够做到全方面正确,轻量级模型的综合表现最差。
此外,GPT-4.0 turbo、Copilot实现了100\%的语法正确率。同时也注意到,虽然Gemini和LLaMa3 70B
的参数规模和前述模型相当,但是准确率却明显较低。本文的经微调后的Gemma 2B模型在所有模型中的语义准确率最高。
根据实验结果,也能够看出,当生成的ASP程序在语法上正确时,其语义正确的可能性也会更大。
所有模型在每种类型的问题上的语法正确率和语义正确率的对比见表\ref{tab:semantics_comparison}。

\begin{table}[h]
    \centering
    \renewcommand{\arraystretch}{1.2}
    \setlength{\tabcolsep}{5pt}
    \begin{tabular}{lcccccccccccc}
        \toprule
        \multirow{2}{*}{模型} & \multicolumn{2}{c}{基础存在性} & \multicolumn{2}{c}{三维空间关系} & \multicolumn{2}{c}{多跳推理} & \multicolumn{2}{c}{多参考系} & \multicolumn{2}{c}{动态反事实} & \multicolumn{2}{c}{对抗性样本} \\
        \cmidrule(lr){2-3} \cmidrule(lr){4-5} \cmidrule(lr){6-7} \cmidrule(lr){8-9} \cmidrule(lr){10-11} \cmidrule(lr){12-13}
        & \textit{语法} & \textit{语义} & \textit{语法} & \textit{语义} & \textit{语法} & \textit{语义} & \textit{语法} & \textit{语义} & \textit{语法} & \textit{语义} & \textit{语法} & \textit{语义} \\
        \midrule
        GPT-4.0 turbo & 1 & 1 & 1 & 1 & 1 & 1 & 1 & 1 & 1 & 1 & 1 & 1 \\
        Copilot & 1 & 1 & 1 & 1 & 1 & 1 & 1 & 1 & 1 & 1 & 1 & 1 \\
        Gemini & 1 & 1 & 1 & 1 & 1 & 1 & 1 & 1 & 1 & 1 & 1 & 1 \\
        Gemma 2B & 1 & 1 & 1 & 1 & 1 & 1 & 1 & 1 & 1 & 1 & 1 & 1 \\
        LLama3 turbo & 1 & 1 & 1 & 1 & 1 & 1 & 1 & 1 & 1 & 1 & 1 & 1 \\
        \midrule
        \textbf{微调模型} & 1 & 1 & 1 & 1 & 1 & 1 & 1 & 1 & 1 & 1 & 1 & 1 \\
        \bottomrule
    \end{tabular}
    \caption{不同模型在各问题类型上的语法正确率和语义正确率的对比}
    \label{tab:semantics_comparison}
\end{table}

通过对比综合表现最佳的几个模型所生成的ASP程序,发现所有上述模型均未对多跳推理问题进行正确编码。


\section{迭代反馈与规则修正}
迭代反馈模块是整个管道的最核心部分。具体而言,Clingo求解器执行输入的ASP程序,并输出信息。如果
在执行过程中出现错误,Clingo将会把这些错误信息输出。而LLM则对这些错误信息进行分析,并对ASP程序进行优化。
优化后的ASP程序再次输入Clingo求解器,如此循环至多3次,最终获得优化修正后的ASP程序,以供进行正式推理。
不同的错误信息对应的问题不同,所需要的对ASP程序进行修正的方式也不同。指导LLM对不同问题进行修正的提示,
也会因为修正方式的不同而不同。针对不同的错误信息,本文设计了不同的提示模板,以指导LLM进行修正。

根据LLM的反馈并对提示模板进行修改,再重新提示LLM,以最终得到较优的提示模板。该过程中,
需要多次和LLM进行交互,流程比较复杂。本文使用Dspy来对该过程进行管理。Dspy的模块化特性增强了模块
之间的记忆保留能力,方便进行参数自适应调整和优化。此外,Dspy支持自动对LLM提示词和参数进行优化,
极大降低了人工修改提示模板的工作量。为了方便调试,所有模块的输出日志均记录错误信息与状态提示。

DSPy提供的三种优化器及人工提示工程的对比结果见表\ref{tab:optimizer_comparison}。本文最终选择了BootstrapFewShow优化器,主要理由是:人工标注ASP事实的成本较高,平均每个样本的标注时间为3.5分钟。而BootstrapFewShot专门为少样本场景设计,仅需15-20个标注样本就可启动优化。
基于本文有限的标注样本,选择了BootstrapFewShot优化器,以提高优化效率。
此外,BootstrapFewShot优化器能够进行多阶段分层次优化。具体而言,其将优化过程拆分为教室优化
和学生训练两个阶段,前者生成高质量的示例,后者完成知识蒸馏,以确保优化的效果。BootstrapFewShot优化器
也支持限制最大错误次数和迭代轮数,避免无限循环和资源浪费。

\begin{table}[htbp]
\centering
\caption{Dspy 框架支持的优化器比较}
\label{tab:optimizer_comparison}
\begin{tabular}{|>{\raggedright}p{4cm}|>{\raggedright}p{8cm}|>{\raggedright\arraybackslash}p{4cm}|}
\hline
\textbf{优化器名称} & \textbf{主要功能} & \textbf{适用场景} \\
\hline
BootstrapFewShot & 通过在提示中自动生成并包含优化示例来扩展签名。 & 少样本学习 \\
\hline
BootstrapFewShot-
WithRandomSearch & 在 BootstrapFewShot 的基础上,对生成的示例进行多次随机搜索,选择优化后的最佳程序。 & 少样本学习,需要更高精度 \\
\hline
MIPRO & 在每个步骤中生成指令和少量示例,使用贝叶斯优化来有效地搜索模块中的生成指令和示例空间。 & 需要复杂指令和示例优化的场景 \\
\hline
BootstrapFinetune & 将基于提示的 DSPy 程序提炼为较小语言模型的权重更新,微调底层大型语言模型以提高效率。 & 需要微调模型以提高效率的场景 \\
\hline
\end{tabular}
\end{table}

\section{ASP推理}
ASP推理阶段中,ASP求解器接收经过优化后的ASP查询语句、视觉场景理解模块提取的ASP事实以及已有常识的ASP表示,进行逻辑推理,最终获得答案。
本文使用Clingo求解器来进行求解,其工作过程分为基础化(grounding)阶段和求解(solving)阶段。

基础化阶段将ASP程序中的变量替换为常量,生成一个新的ASP程序。Clingo
通过使用内置的基础化器分析程序,生成所有可能的规则。例如,对规则
$a(X) :- b(X), not c(X).$,其将被展开为所有可能的值为X的实例。

求解阶段中,Clingo采用类似SAT求解器的冲突驱动答案集求解(CDNL)方法。CDNL方法通过迭代地添加约束,直到找到一个满足所有约束的解,或者证明无解。
具体而言,Clingo基于如下步骤进行求解:(1)初始化。从空分配开始(没有原子被标记为真);
(2)选择与分配。选择一个未分配的原子,猜测其真值为真或假;
(3)传播。根据当前分配和规则,推导其他原子的真值。例如,若规则$a :- b, not c.$满足,b为真且c不在答案集中,则a必须为真;
(4)冲突检测。若分配导致规则冲突(如与已有的约束条件相抵触),记录冲突原因;
(5)回溯与学习。若发生冲突,回溯到之前的选择点,学习冲突子句以避免未来类似错误;
(6)验证。当所有原子分配完成后,检查是否为答案集,即确保它是程序的模型,且最小化(无子集也能满足规则)。

相比其它的ASP求解器,Clingo在以下几个方面进行了优化:(1)从冲突中学习
新的规则,为将来进一步的搜索提供指导;(2)使用多线程实现并行化,同时搜索
多个路径,加速求解过程;(3)使用启发式方法决定分配顺序,例如优先选择高影响力的原子;
(4)预测可能发生的冲突,选择可能导致冲突的原子优先分配,减少搜索空间。
\section{实验与结果分析}
\subsection{对比试验}
为全面评估神经符号框架的有效性以及泛化能力,本文选取了目前主流的三种LLM:DeepSeek R1、
Llama3和GPT-4 turbo,在其上应用神经符号框架,进行对比。以上既包含DeepSeek这类轻量级专用模型,
也涵盖GPT-4 turbo这类通用型先进系统,而Llama3性能和效率之间取得了平衡,属于居中水平的模型。通过
在多个基座上进行实验,证明神经符号框架对不同LLM的泛化能力。

比较基准一选取直接提示VLM的方式,即直接将自然语言问题和图像输入到VLM中,并不给予任何的额外提示。
直接提示VLM的方式虽然简单,却是评估模型的关键基准,因为直接提示方法能够反映模型在没有任何
外部推理辅助机制的情况下,自身对空间问题的处理能力。

比较基准二采用“事实+规则”的提示方法。该方法的核心思路是:指示LLM使用预先定义的谓词,将输入的自然语言问题转换为结构化事实,然后LLM应用相关逻辑
规则,通过自然语言推理答案。
“事实+规则”的提示方法满足了将原始自然语言问题转为结构化符号表示,以配合视觉场景理解所得的ASP事实以及原有常识,共同进行推理的核心需求。同时,
该方法使用具有精确参数结构的谓词对LLM进行提示,使得LLM可以创建一致的中间态的知识表示,为问题解答提供便利。综合来看,“事实+规则”的提示方法作为一种
简化流程,既保留了形式化推理的优势,同时能够避免在生成ASP程序时和LLM进行多次交互,进而降低对计算资源的要求,降低了成本,也减少了对外部求解器的依赖。

实验结果见表\ref{tab:overall_comparison},本文提出的神经符号框架在DeepSeek R1、LLama3和GPT-4.0 turbo上均超过了两种比较基准方法,证明大语言模型与逻辑推理结合对于解决
复杂空间推理问题的有效性。

\begin{table}[h]
    \centering
    \small  % 调整字体大小(可选:\footnotesize 更紧凑)
    \renewcommand{\arraystretch}{1.2}  % 增加行距
    \setlength{\tabcolsep}{5pt}  % 调整列间距
    \resizebox{\textwidth}{!}{  % 让表格自适应页面宽度
    \begin{tabular}{lcccccc}
        \toprule
        \textbf{模型及方法} & \textbf{基础存在性问题} & \textbf{三维空间关系问题} & \textbf{多跳推理问题} & \textbf{多参考系问题} & \textbf{对抗性样本问题} & \textbf{总体} \\
        \midrule
        \multicolumn{7}{c}{\textbf{DeepSeek R1}} \\  
        直接提问 & 1 & 1 & 1 & 1 & 1 & 1 \\
        事实+规则提示方法 & 1 & 1 & 1 & 1 & 1 & 1 \\
        神经符号框架方法 & 1 & 1 & 1 & 1 & 1 & 1 \\
        \midrule
        \multicolumn{7}{c}{\textbf{Llama3}} \\  
        直接提问 & 1 & 1 & 1 & 1 & 1 & 1 \\
        事实+规则提示方法 & 1 & 1 & 1 & 1 & 1 & 1 \\
        神经符号框架方法 & 1 & 1 & 1 & 1 & 1 & 1 \\
        \midrule
        \multicolumn{7}{c}{\textbf{GPT-4.0 turbo}} \\  
        直接提问 & 1 & 1 & 1 & 1 & 1 & 1 \\
        事实+规则提示方法 & 1 & 1 & 1 & 1 & 1 & 1 \\
        神经符号框架方法 & 1 & 1 & 1 & 1 & 1 & 1 \\
        \bottomrule
    \end{tabular}
    }
    \caption{不同模型及方法在各问题类型上的表现}
    \label{tab:overall_comparison}
\end{table}

\subsection{消融实验}
尽管LLM在语义解析任务中表现出色,但其生成的ASP程序的正确率仍有较大的提升空间。Feng\cite{feng2024language}等人的研究表明,将自然语言直接转为逻辑规则的
成功率一般较低。而神经符号框架通过在LLM和外部ASP求解器之间设立反馈循环机制,使LLM能够根据ASP求解器在试图求解ASP程序之前的语法检查结果,对生成的ASP程序
中的错误进行修正,极大提高了将自然语言问题转为ASP程序的准确率。

为进一步探讨神经符号框架的迭代反馈机制的作用,本文重点关注ASP求解器执行过程中常发生的三类主要错误:解析错误、实例化失败以及求解阶段失败。
另外,即便生成的ASP程序执行成功,也有很大可能由于自然语言问题与LLM生成的ASP程序之间不一致,而产生与真实答案偏离的结果。

在测试中,迭代反馈机制显著提升了所有模型的性能。如图\ref{fig:ablation}所示,经过两轮迭代反馈之后,DeepSeek R1的可执行率从51.3\%提升到了84.2\%,LLama3的可执行率从60.2\%提升到了89.7\%
,GPT-4.0 turbo的可执行率从64.6\%提升到了91.1\%。
在准确率方面,DeepSeek R1的准确率从57.8\%提升到了79.3\%,LLama3的准确率从61.1\%提升到了85.7\%,GPT-4.0 turbo的准确率从69.7\%提升到了93.4\%。
以上结果表明,LLM与ASP求解器之间的迭代反馈机制能有效解决自然语言到逻辑程序的转换过程中面临的问题。对准确率和可执行率的提升,主要集中在
第一轮迭代反馈中,此后继续迭代的效果呈现边际递减趋势。由于计算资源有限,本次消融实验仅进行了三轮。实验结果有力证明,迭代反馈机制在提升ASP程序的可执行率和
正确率方面具有明显效果,充分验证了神经符号方法的有效性。

\begin{figure}
    \centering
    \includegraphics[width=\textwidth]{ablation.png}
    \caption{LLM与ASP求解器之间的迭代反馈机制的效果对比}
    \label{fig:ablation}
\end{figure}

\section{本章小结}

\input{chapter/System.tex}

\input{chapter/Expectation.tex}

\backmatter

%打印参考文献表
\thesisbib
%附录
\appendix
\chapter{附录}
\section{POVQAD数据集示例}
\subsection{环境}
\label{appendix:environment}
生成的每个场景都需要满足其对应环境中的所有约束。以下是POVQAD中所有环境共享的通用规则,所有规则均以ASP表示:
\begin{lstlisting}
1. property(color, gray). property(color, red).
2. property(color, blue). property(color, green).
3. property(color, brown). property(color, purple).
4. property(color, cyan). property(color, yellow).
5. property(shape, cube). property(shape, cylinder).
6. property(shape, sphere). property(shape, cone).
7. property(size, small). property(size, medium).
8. property(size, large).
9. property(material, rubber).
property(material, metal).
10. region(0). region(1). region(2). region(3).
11. right_R(0, 0). right_R(0, 1). right_R(0, 2).
right_R(0, 3).
12. right_R(1, 1). right_R(1, 3).
13. right_R(2, 0). right_R(2, 1). right_R(2, 2).
right_R(2, 3).
14. right_R(3, 1). right_R(3, 3).
15. left_R(R1, R2) :- right_R(R2, R1).
16. front_R(0, 0). front_R(0, 1). front_R(0, 2).
front_R(0, 3).
17. front_R(1, 0). front_R(1, 1). front_R(1, 2).
front_R(1, 3).
18. front_R(2, 2). front_R(2, 3).
19. front_R(3, 2). front_R(3, 3).
20. behind_R(R1, R2) :- front_R(R2, R1).
21. sameProperty(X1, X2, P) :- has_property(X1,P,V),}
22. has_property(X2,P,V), X1!=X2.
23. same_color(X,Y):- sameProperty(X, Y, color).
24. same_size(X,Y):- sameProperty(X, Y, size).
25. same_shape(X,Y):- sameProperty(X, Y, shape).
26. same_material(X,Y):- sameProperty(X, Y, material).
27. 1{has_property(X, color, V) :
28. property(color, V)}1 :- obj(X).
29. 1{has_property(X, material, V) :
30. property(material, V)}1 :- obj(X).
31. 1{has_property(X, shape, V) :
32. property(shape, V)}1 :- obj(X).
33. 1{has_property(X, size, V) :
34. property(size, V)}1 :- obj(X).
35.1{at(X, R): region(R)}1 :- obj(X).
36.:- sameProperty(X1, X2, color),
37. sameProperty(X1, X2, material),
38. sameProperty(X1, X2, size)},
39. sameProperty(X1, X2, shape),
40. obj(X1), obj(X2), X1!=X2.
41.exceed_region_capacity(R) :42. #count{X: obj(X), at(X, R)} >= 4, region(R).
43:- exceed_region_capacity(_).
\end{lstlisting}
以上通用规则对应的自然语言含义如下:
\begin{lstlisting}
1-9. 对象必须具有四个属性维度:颜色、形状、尺寸、材质。
1-4. 颜色属性取值范围(8种):灰色、红色、蓝色、绿色、棕色、紫色、青色、黄色。
5-6. 形状属性取值范围(4种):立方体、圆柱体、球体、圆锥体。
7-8. 尺寸属性取值范围(3级):小、中、大。
9. 材质属性取值范围(2种):橡胶、金属。
10. 场景划分为四个空间区域,编号为0、1、2、3。
11. 当对象A位于区域0时,其右侧对象B的合法区域:0/1/2/3。
12. 当对象A位于区域1时,其右侧对象B的合法区域:1/3。
13. 当对象A位于区域2时,其右侧对象B的合法区域:0/1/2/3。
14. 当对象A位于区域3时,其右侧对象B的合法区域:1/3。
15. 方位对称性规则:若对象A在对象B右侧,则对象B必在对象A左侧。
16. 当对象A位于区域0时,其前方对象B的合法区域:0/1/2/3。
17. 当对象A位于区域1时,其前方对象B的合法区域:0/1/2/3。
18. 当对象A位于区域2时,其前方对象B的合法区域:2/3。
19. 当对象A位于区域3时,其前方对象B的合法区域:2/3。
20. 方位对称性规则:若对象A在对象B前方,则对象B必在对象A后方。
27-28. 颜色属性强制单值约束:每个对象必须且只能具有一个颜色值。
29-30. 材质属性强制单值约束:每个对象必须且只能具有一个材质值。
31-32. 形状属性强制单值约束:每个对象必须且只能具有一个形状值。
33-34. 尺寸属性强制单值约束:每个对象必须且只能具有一个尺寸值。
35. 空间位置强制单值约束:每个对象必须且只能被分配至一个区域。
36-40. 对象差异性原则:任意两个对象不得在所有四个属性(颜色/形状/尺寸/材质)上完全一致。
41-43. 区域容量限制:每个空间区域最多容纳3个对象。
\end{lstlisting}
以下用ASP表示的约束,用来表示图\ref{}所示场景所属的特定环境。
\begin{lstlisting}
44. obj(0..4).
45. :- obj(X), at(X, 0),
has_property(X, size, large).
46. :- obj(X), at(X, 0),
has_property(X, shape, cylinder).
47. :- obj(X), at(X, 0),
has_property(X, shape, cone).
48. :- obj(X), at(X, 1),
has_property(X, size, small).
49. :- obj(X), at(X, 1),
has_property(X, shape, cone).
50. :- obj(X), at(X, 1),
has_property(X, material, rubber).
51. :- obj(X), at(X, 1),
has_property(X, shape, cube).
52. :- obj(X), at(X, 2),
not has_property(X, size, medium).
53. :- obj(X), at(X, 2),
not has_property(X, material, metal).
54. :- obj(X), at(X, 2),
has_property(X, material, rubber).
55. :- obj(X), at(X, 2),
has_property(X, shape, sphere).
56. :- obj(X), at(X, 2),
has_property(X, shape, cube).
57. :- obj(X), at(X, 3),
has_property(X, size, small).
58 :- obj(X), at(X, 3),
not has_property(X, material, metal),
59. not has_property(X, color, blue).
60. :- #count{X1, X2: sameProperty(X1, X2, shape),
61. obj(X1), obj(X2), at(X1, 3), at(X2, 2),
62. has_property(X1, color, yellow),
63. has_property(X2, color, yellow)} >= 4.
64. :- #count{X1, X2: sameProperty(X1, X2, color),
65. obj(X1), obj(X2),
66. at(X1, 0), at(X2, 3)} >= 2.
\end{lstlisting}
以下是对前述的规则的逐行解释:
\begin{lstlisting}
44. 场景中共存在5个对象。
45. 区域0内禁止存在大尺寸对象。
46. 区域0内禁止存在圆柱形对象。
47. 区域0内禁止存在圆锥形对象。
48. 区域1内禁止存在小尺寸对象。
49. 区域1内禁止存在圆锥形对象。
50. 区域1内禁止存在橡胶材质对象。
51. 区域1内禁止存在立方体对象。
52. 区域2内所有对象必须为中尺寸。
53. 区域2内所有对象必须为金属材质。
54. 区域2内禁止存在橡胶材质对象。
55. 区域2内禁止存在球形对象。
56. 区域2内禁止存在立方体对象。
57. 区域3内禁止存在小尺寸对象。
58-59. 区域3内所有对象必须满足以下条件之一:金属材质或者蓝色外观。
60-63. 区域3与区域2内黄色对象组合规则:相同形状的黄色对象配对组数≤1。
64-66. 区域0与区域3联合约束: 具有相同颜色的对象配对组数 =0(严格禁止)。
\end{lstlisting}
%致谢
\chapter{致谢}
到这里,我的学生时代就要告一段落了,真的要从学生变成社会人了。虽然不想承认,但还是要接受这个现实。
心中感慨万千,在此写下这篇致谢,权且当作一个简短的回忆录。

第一,我想感谢我的导师。张老师是一位非常和蔼可亲的老师,他极少发脾气,平时见到我们同学们一直都是
笑呵呵的,并且给我们发了不少助研金,能让我在硕士阶段不仅不用向父母要钱,还能自己存下不少存款,让很多同龄
的硕士研究生都非常羡慕。其次,他在培养上认真负责,对同学们的开题和毕业论文都十分上心,能让我们顺利毕业。
再者,他放我们出去实习,这一点在学院的众多导师中,其实还是比较难得的。当下,找互联网企业的对口工作,
或者是央国企,都十分看重实习经历,想找一个好点的算法或者后端的工作,都得要最少最少一段,一般都要两段
实习经历。张老师能够让我们离校实习,对我们学生个人的求职,无疑是十分重要的一点。

第二,我想感谢东大。依稀记得还在南师大准备考研的时候,每天我都会播放东大校歌《临江仙》,力争让自己在
精神上“皈依”东大,支撑自己每天能学进去11个小时。没考进来之前,每天都是念着东大的好,考进来之后,开始
经常吐槽学校。但思前想后,还是要感谢东大。东大作为985,还是给了我一些普通211享受不到的平台和资源。
能让我简历至少通过一些单位的筛选,能有资格考选调生、公务员特招,能享受到更好的资源。如果我还在南师大,
我是不敢想象我有机会参加就业办的活动,公费去广州和青岛游玩的。

第三,我想感谢我的亲人。我的父母是传统的山东农村人,他们没什么文化,见识也不多,做不到像双公务员家庭那样,
给子女在经济、升学、求职上给予很多帮助,但是他们确实也是尽己所能,让我有机会考大学。难能可贵的是,他们
至少没有阻挠我去自主做一些重大决策,在我追求个人发展的路上,他们没有像一些农村父母那样去当“拦路虎”,我想这也是
比较难能可贵的一点。我也想感谢我的姐姐和姐夫,他们在经济上给予了我一些帮助,每次我到北京考试或者实习,他们都热情接待
我。他们两人从山东一路考到北京,双方父母都没啥本事,凭自己的努力和时代的东风,在北京打下了一片天地,
着实让我羡慕。

第四,我想感谢南京这座城市。我从小一直在村镇上,拿个快递都要骑电动车去2公里外的镇上,娱乐生活极其匮乏,也没什么玩得来的同龄人。
南京给了我大城市的感觉,玄武湖、紫金山、长江,让我体验到这座城市的魅力。
我还记得2018年8月30日那天的六点多钟,天色已经黑了,我坐着客车跨过长江二桥,看到夜幕下奔涌的长江的感受。
七年来,我走遍了南京城区,从北京西路到九乡河西路,从燕子矶到禄口,桥林、岔路口、麒麟门等等这些地名我都
如数家珍,南京的一草一木,山山水水,都融入我的血脉之中。我也学会了一些南京方言,宿管和食堂的阿姨一度都问我:“小伙子啊是南京人啊?”。
苏A、3201、025,这几个南京的同义词,一直都萦绕在我的脑海之中。
其实硬要说南京好,也没那么好,但是也没那么坏。我是个比较懒的人,一旦在一个城市呆久了,就懒得换地方,
本能地进入了舒适圈,就不想再出来。
虽然由于客观就业形势,我无法留在南京工作,没能通过名校优生和江苏省考进入公务员序列。
但我今后不论身处何地,我都会一如既往的关心南京,支持南京。

当然,最重要的,还是要感谢我自己。我出身农村,从小放养,没上过补习班,父母也不懂选大学选专业,本科误入统计专业,
后来下定决心从南师大统计学跨考到东大计算机专业,全身心复习10个月,每天学习11个小时,最终考了395分考到东大。
我自己没有什么天资,最多是勤奋一点,愿意去卷。经济下行,内卷加剧,不少
应届生都找不到工作,更遑论薪水较高的工作,我作为学历一般,本科非计算机专业,实习经历不够“卷”的学生,能够签一个互联网大厂的工作,
实属不易。并且在实习转正后,并没有放弃求职,而是继续参加四个省市的选调生考试、国考、江苏省考、广东省考,尽管只是进面,没有最终考上,
但至少还是积极主动去试过,试图去为自己争取机会。作为只能吃到时代黑利的自己,已经不错了。
没赶上好时候,就只能在思想上去“比烂”,当一当阿Q,让自己过得心里舒服点,不然自己更难受,不是吗?

祝福一路上帮助过我的亲人和朋友,祝福母校,祝福南京,祝福自己。
希望我的美股账户早日能到100万美金,早日考上公务员。

\end{document}
