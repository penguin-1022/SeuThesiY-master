\chapter{致谢}
到这里,我的学生时代就要告一段落了,真的要从学生变成社会人了。虽然不想承认,但还是要接受这个现实。
心中感慨万千,在此写下这篇致谢,权且当作一个简短的回忆录。

第一,我想感谢我的导师。张老师是一位非常和蔼可亲的老师,他极少发脾气,平时见到我们同学们一直都是
笑呵呵的,并且给我们发了不少助研金,能让我在硕士阶段不仅不用向父母要钱,还能自己存下不少存款,让很多同龄
的硕士研究生都非常羡慕。其次,他在培养上认真负责,对同学们的开题和毕业论文都十分上心,能让我们顺利毕业。
再者,他放我们出去实习,这一点在学院的众多导师中,其实还是比较难得的。当下,找互联网企业的对口工作,
或者是央国企,都十分看重实习经历,想找一个好点的算法或者后端的工作,都得要最少最少一段,一般都要两段
实习经历。张老师能够让我们离校实习,对我们学生个人的求职,无疑是十分重要的一点。

第二,我想感谢东大。依稀记得还在南师大准备考研的时候,每天我都会播放东大校歌《临江仙》,力争让自己在
精神上“皈依”东大,支撑自己每天能学进去11个小时。没考进来之前,每天都是念着东大的好,考进来之后,开始
经常吐槽学校。但思前想后,还是要感谢东大。东大作为985,还是给了我一些普通211享受不到的平台和资源。
能让我简历至少通过一些单位的筛选,能有资格考选调生、公务员特招,能享受到更好的资源。如果我还在南师大,
我是不敢想象我有机会参加就业办的活动,公费去广州和青岛游玩的。

第三,我想感谢我的亲人。我的父母是传统的山东农村人,他们没什么文化,见识也不多,做不到像双公务员家庭那样,
给子女在经济、升学、求职上给予很多帮助,但是他们确实也是尽己所能,让我有机会考大学。难能可贵的是,他们
至少没有阻挠我去自主做一些重大决策,在我追求个人发展的路上,他们没有像一些农村父母那样去当“拦路虎”,我想这也是
比较难能可贵的一点。我也想感谢我的姐姐和姐夫,他们在经济上给予了我一些帮助,每次我到北京考试或者实习,他们都热情接待
我。他们两人从山东一路考到北京,双方父母都没啥本事,凭自己的努力和时代的东风,在北京打下了一片天地,
着实让我羡慕。

第四,我想感谢南京这座城市。我从小一直在村镇上,拿个快递都要骑电动车去2公里外的镇上,娱乐生活极其匮乏,也没什么玩得来的同龄人。
南京给了我大城市的感觉,玄武湖、紫金山、长江,让我体验到这座城市的魅力。
我还记得2018年8月30日那天的六点多钟,天色已经黑了,我坐着客车跨过长江二桥,看到夜幕下奔涌的长江的感受。
七年来,我走遍了南京城区,从北京西路到九乡河西路,从燕子矶到禄口,桥林、岔路口、麒麟门等等这些地名我都
如数家珍,南京的一草一木,山山水水,都融入我的血脉之中。我也学会了一些南京方言,宿管和食堂的阿姨一度都问我:“小伙子啊是南京人啊?”。
苏A、3201、025,这几个南京的同义词,一直都萦绕在我的脑海之中。
其实硬要说南京好,也没那么好,但是也没那么坏。我是个比较懒的人,一旦在一个城市呆久了,就懒得换地方,
本能地进入了舒适圈,就不想再出来。
虽然由于客观就业形势,我无法留在南京工作,没能通过名校优生和江苏省考进入公务员序列。
但我今后不论身处何地,我都会一如既往的关心南京,支持南京。

当然,最重要的,还是要感谢我自己。我出身农村,从小放养,没上过补习班,父母也不懂选大学选专业,本科误入统计专业,
后来下定决心从南师大统计学跨考到东大计算机专业,全身心复习10个月,每天学习11个小时,最终考了395分考到东大。
我自己没有什么天资,最多是勤奋一点,愿意去卷。经济下行,内卷加剧,不少
应届生都找不到工作,更遑论薪水较高的工作,我作为学历一般,本科非计算机专业,实习经历不够“卷”的学生,能够签一个互联网大厂的工作,
实属不易。并且在实习转正后,并没有放弃求职,而是继续参加四个省市的选调生考试、国考、江苏省考、广东省考,尽管只是进面,没有最终考上,
但至少还是积极主动去试过,试图去为自己争取机会。作为只能吃到时代黑利的自己,已经不错了。
没赶上好时候,就只能在思想上去“比烂”,当一当阿Q,让自己过得心里舒服点,不然自己更难受,不是吗?

祝福一路上帮助过我的亲人和朋友,祝福母校,祝福南京,祝福自己。
希望我的美股账户早日能到100万美金,早日考上公务员。