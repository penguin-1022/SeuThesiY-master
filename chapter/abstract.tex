%===============================================================================
\categorynumber{000} % 分类采用《中国图书资料分类法》
\UDC{000}            %《国际十进分类法UDC》的类号
\secretlevel{公开}    %学位论文密级分为"公开"、"内部"、"秘密"和"机密"四种
\studentid{222171}   %学号要完整,前面的零不能省略。
\title{积木世界VQA中的}{积木世界VQA中的}{空间推理问答技术研究与实现}{空间推理问答技术研究与实现}{Research and Implementation of Spatial Reasoning Questioning Answering Techniques }{in the Block World VQA}
\author{贾梁}{Jia Liang}
\advisor{张志政}{}{Zhang Zhizheng}{}
% \coadvisor{张志政}{副教授}{Zhang Zhizheng}{Associate Prof.} % 没有% 可以不填
\degreetype{工程硕士}{Master of Engineering} % 详细学位名称
\thesisform{应用研究} % 包括应用研究、调研报告、规划、产品开发、案例分析、项目管理、文学艺术作品、其它。非专业型硕士可忽略
\major{电子信息}
\submajor{计算机技术}
\defenddate{2025年5月30日}
\authorizedate{2025年6月20日}
\committeechair{翟玉庆}
\reviewer{倪庆剑}{张祥}
\department{东南大学计算机科学与工程学院}{School of Computer Science and Engineering}
\makebigcover
\makecover
\begin{abstract}{多模态,回答集编程,视觉问答,空间推理,神经符号方法}
积木世界是人工智能研究、教学、实验、评估的重要场景,能够模拟实际应用中物体间的空间关系。
积木世界VQA要求视觉语言模型(Visual Language Model, VLM)具备空间推理能力,但已有研究表明,当面对部分可见场景时,
其回答准确率显著下降。目前,通过结合神经网络和符号推理形成的神经符号模型,利用深度学习将视觉信息转化为逻辑符号,
再利用符号推理进行问题求解问题,比单纯依赖深度学习的视觉语言模型在空间推理任务中有更优异的表现。
由于回答集程序(Answer Set Program, ASP)是一种具备非单调推理和高效推理机的符号推理方法,
神经符号模型中采用ASP构建的神经符号VQA框架被寄予厚望。然而,现有框架设计中ASP规则的扩展仍依赖人工,
难以有效应对部分可见场景中基于空间推理的VQA任务。
    
针对上述问题,本文从构建积木世界部分可见场景中基于空间推理的VQA数据集、
设计规则自动补充的神经符号VQA方法、设计实现VQA课堂演示原型系统三个方面开展工作,具体如下:

\begin{enumerate}[itemsep=0pt]
\item 积木世界部分可见场景空间推理VQA数据集(Partial Observation VQA Dataset, POVQA\-D)构建。
CLEVR是空间推理的经典数据集,然而CLEVR中问题涉及的物体属性和物体间空间关系在图像中均完全可见,
通过引入部分可见性、遮挡机制与复杂空间关系模板,构建了覆盖部分场景可见问题的VQA数据集POVQAD。
该数据集将问题推理所需的平均步骤数从3提升到5,并新增了超 15\% 的空间封闭性与假设类问题,
且超过40\%的图像中包含不可见目标,比原数据集更能有效考察模型在部分可见场景中回答空间推理问题的能力。
\item 规则自动补充的神经符号VQA框架(Rule Complement Neuro-Symbolic Pipeline, RCNSP)设计。
RCNSP借鉴现有神经符号VQA框架的架构,新增规则蒸馏模块,
实现在ASP求解器进行推理前调用微调后的大语言模型自动对规则进行补充。
实验表明,在DeepSeek、Llama3、ChatGPT-4o三种大语言模型上,使用RCNSP比直接向VLM提问的准确率平均提升16.5\%,
比现有的神经符号VQA框架的准确率平均提升8.9\%,表明通过规则蒸馏能有效提升神经符号方法在部分可见场景中基于空间推理VQA中的效能。
\item 设计实现了一个VQA课堂演示原型系统。在RCNSP框架和POVQAD数据集基础上,设计实现了VQA原型系统,
该系统支持在针对用户的VQA任务进行空间推理时展示推理的中间步骤和逻辑链条,能够为开展相关研究实验和人工智能课堂教学提供辅助。
初步测试表明该系统在CPU为Intel Core i9-11900K,内存128G,显卡为3张RTX 3090并联的硬件环境下,并发量为38,
90\%响应时间为6.9秒,能够满足课堂教学场景下的用户需要。
\end{enumerate}

\end{abstract}

\begin{englishabstract}{Multi-modal, Answer Set Programming, Visual Question Answering, Spatial Reasoning, Neuro-symbolic Method}
The block world is an important scenario for artificial intelligence research, education, experimentation, and evaluation, as it can simulate the spatial relationships between objects in real-world applications. Block world VQA requires Visual Language Models (VLMs) to possess spatial reasoning abilities; however, previous studies have shown that when faced with partially observable scenes, the answer accuracy significantly decreases. Currently, neuro-symbolic models that combine neural networks with symbolic reasoning—by using deep learning to transform visual information into logical symbols and then employing symbolic reasoning for problem solving—have demonstrated superior performance in spatial reasoning tasks compared to VLMs that rely solely on deep learning. Given that Answer Set Programming (ASP) is a symbolic reasoning method characterized by non-monotonic inference and efficient solvers, ASP-based neuro-symbolic VQA frameworks are highly anticipated. Nevertheless, the current framework designs still rely on manually expanded ASP rules, making it challenging to effectively address spatial reasoning VQA tasks in partially observable scenes.

To tackle the above issues, this work is conducted from three aspects: constructing a VQA dataset for spatial reasoning in partially observable block worlds, designing a neuro-symbolic VQA approach with automatic rule supplementation, and developing a prototype VQA classroom demonstration system. The specific contributions are as follows:
\begin{enumerate}[itemsep=0pt, parsep=0pt]
\item Construction of Partial Observation VQA Dataset (POVQAD).
CLEVR is a classical dataset for spatial reasoning; however, in CLEVR, 
all object attributes and spatial relationships are completely visible in the images. 
By introducing partial observability, occlusion mechanisms, and complex spatial relation templates, 
the VQA dataset POVQAD is constructed to cover partially observable scenarios. 
This dataset increases the average number of reasoning steps required per question from 3 to 5, 
adds more than 15\% of spatial closure and hypothetical reasoning questions, 
and includes over 40\% of images with invisible targets, 
thereby more effectively evaluating a model’s ability to answer spatial reasoning questions in partially observable scenes.
\item Design of the Rule Complement Neuro-Symbolic Pipeline (RCNSP).
The RCNSP framework builds on the architecture of existing neuro-symbolic VQA frameworks by introducing 
a rule distillation module. This module automatically supplements the rules using a fine-tuned large language 
model before the ASP solver performs reasoning. Experiments show that, across three large language models—DeepSeek, 
Llama3, and ChatGPT-4o—RCNSP achieves an average accuracy improvement 
of 16.5\% over directly querying VLMs and an 8.9\% increase over existing neuro-symbolic VQA frameworks. 
These results indicate that rule distillation can effectively enhance the performance of neuro-symbolic methods 
in spatial reasoning VQA tasks under partial observability.
\item Implementation of a VQA Classroom Demonstration Prototype System. 
Based on the RCNSP framework and the POVQAD dataset, a VQA prototype system is developed. 
This system supports the display of intermediate reasoning steps and logical chains when performing spatial reasoning
 tasks in response to user queries, thereby providing assistance for related research experiments and artificial 
 intelligence classroom instruction. Preliminary tests indicate that, under a hardware configuration of 
 an Intel Core i9-11900K CPU, 128 GB memory, and three RTX 3090 GPUs running in parallel, 
 with a concurrency level of 38 and a 90th percentile response time of 6.9 seconds, 
 the system meets the requirements of classroom teaching scenarios.
\end{enumerate}
\end{englishabstract}

\setnomname{术语与符号约定}
\tableofcontents
\listofothers
%===============================================================================
