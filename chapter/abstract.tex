%===============================================================================
\categorynumber{000} % 分类采用《中国图书资料分类法》
\UDC{000}            %《国际十进分类法UDC》的类号
\secretlevel{公开}    %学位论文密级分为"公开"、"内部"、"秘密"和"机密"四种
\studentid{222171}   %学号要完整,前面的零不能省略。
\title{面向空间推理领域的}{面向空间推理领域的}{神经符号框架的设计与实现}{神经符号框架的设计与实现}{Research and implementation of visual question answering technology }{based on answer set programming}
\author{贾梁}{Jia Liang}
\advisor{张志政}{副教授}{Zhang Zhizheng}{Associate Prof.}
% \coadvisor{张志政}{副教授}{Zhang Zhizheng}{Associate Prof.} % 没有% 可以不填
\degreetype{工程硕士}{Master of Engineering} % 详细学位名称
\thesisform{应用研究} % 包括应用研究、调研报告、规划、产品开发、案例分析、项目管理、文学艺术作品、其它。非专业型硕士可忽略
\major{电子信息}
\submajor{计算机技术}
\defenddate{2025年5月30日}
\authorizedate{2025年6月20日}
\committeechair{翟玉庆}
\reviewer{倪庆剑}{张祥}
\department{东南大学计算机科学与工程学院}{School of Computer Science and Engineering}
\makebigcover
\makecover
\begin{abstract}{多模态,回答集编程,视觉问答,空间推理}
空间推理能力是智能机器人等具身人工智能的重要基础。CLEVR数据集作为空间推理领域的经典数据集,能够较好地模拟夹取、移动物体等情境,从而对智能体的空间推理能力进行有效评估。然而,CLEVR数据集生成的图像均基于完整场景,回答问题所需的信息直接可见,因而无法充分考察模型在处理部分可见场景问题上的能力。近年来,视觉语言模型(Visual Language Model, VLM)在空间推理任务上取得了一定成果,但已有研究表明,当面对部分可见场景时,其回答准确率显著下降,这一不足限制了其在机器人领域的进一步应用。与此同时,回答集编程(Answer Set Programming, ASP)作为一种符号型知识表示与推理方法,已在公理化及缺省的空间关系表示与推理方面表现出较强能力。通过利用深度学习技术对感知信息进行处理,生成输入图像与问题的逻辑符号表示,并结合ASP进行推理与解答的方法,相较于单纯依赖深度学习的视觉语言模型,在空间推理任务中取得了更为优异的表现。然而,现有神经符号框架在ASP规则的自动拓展方面存在不足,需要依赖人工干预进行规则扩充,从而增加了开发人员的工作量。
    
为了解决上述问题,本文构建了一个视觉问答数据集,并提出了一种面向空间推理领域的神经符号框架。本文主要工作如下:
\begin{enumerate}[itemsep=0pt]
\item 数据集构建:在原有CLEVR数据集的基础上,本文新增部分可见场景的问题,从而构建了一个能够更全面考察模型空间推理能力的视觉问答数据集。
\item 神经符号框架设计:基于DSPy开发平台,本文提出的框架融合了DSPy在自动化提示生成和优化过程中的优势,其整体架构包括视觉场景理解、语义解析、规则蒸馏与迭代反馈等模块。相较于现有框架,本文新增的规则蒸馏模块在ASP求解器进行推理前调用大语言模型,通过利用ASP知识库中的先验知识对语义解析与视觉场景理解生成的ASP规则进行补充,并在经过一致性检验后将新获得的知识保存至知识库,从而实现了ASP规则的自动拓展能力。
\end{enumerate}

通过对比实验,本文将所提出的神经符号方法与直接提示视觉语言模型的方法进行了比较。实验结果显示,在三种大语言模型上,神经符号方法在回答部分可见场景问题时的准确率平均提升了12.3\%,充分验证了该方法在提升大语言模型空间推理能力及跨模型迁移性方面的有效性。此外,消融实验结果表明,规则蒸馏模块对整体框架的问答准确率具有显著促进作用。最后,基于所设计的神经符号框架,本文实现了一个自动规划的课程教学演示问答原型系统,为教师开展机器人空间推理演示教学提供了有效支持。
\end{abstract}

\begin{englishabstract}{Multi-modal, Answer Set Programming, Visual Question Answering, Spatial Reasoning}
Currently, visual language models have achieved promising results in tasks such as visual question answering and image captioning; however, they often perform poorly in spatial reasoning—particularly in scenarios where scenes are only partially visible—resulting in inadequate understanding and inference of spatial relationships. Answer Set Programming (ASP) is a formal method for knowledge representation and reasoning, and several studies have demonstrated its strong capability in comprehending spatial relationships and conducting spatial reasoning. In the neural-symbolic system for visual question answering, deep learning techniques are used for perception to generate logical symbolic representations of input images and questions; ASP is then employed for reasoning to produce the corresponding answers. Compared with standalone visual language models, this neural-symbolic system exhibits superior performance in spatial reasoning. Nevertheless, existing visual question answering datasets that assess spatial reasoning ability suffer from issues such as overly simplistic question designs, limited diversity in object types and attributes, and a lack of varied linguistic expression in questions. Furthermore, current neural-symbolic systems generally exhibit poor generalization, with the process of writing and maintaining prompt templates being overly complex, resulting in relatively high development difficulty. They also require manually designed additional rules for new scenarios, which limits their scalability. To address these issues, this thesis constructs a visual question answering dataset and proposes a neural-symbolic framework specifically oriented toward spatial reasoning. The main contributions of this work are as follows:
\begin{enumerate}[itemsep=0pt]
\item A visual question answering dataset is constructed based on the CLEVR dataset. This dataset enhances the difficulty by considering factors such as the number of reasoning steps required, the diversity of object types and attributes, and the variability and ambiguity of question language. This allows for a more rigorous evaluation of a model’s capability in tackling complex spatial reasoning tasks.
\item A neural-symbolic framework tailored for spatial reasoning is designed, achieving system integration between an ASP solver and a large language model. Developed on the DSPy platform, the framework leverages DSPy’s notable advantages in automated prompt generation and optimization. Architecturally, it comprises modules for visual scene understanding, semantic parsing, rule distillation, and iterative feedback, thereby enhancing the visual question answering system’s ability to address spatial reasoning problems.
\item ased on the proposed neural-symbolic framework, a neural-symbolic system for visual question answering is designed and implemented.
\end{enumerate}

Through these contributions, the proposed neural-symbolic framework demonstrates strong performance and adaptability in enhancing the spatial reasoning capabilities of visual language models, particularly in handling complex and partially observable scenarios.
\end{englishabstract}

\setnomname{术语与符号约定}
\tableofcontents
\listofothers
%===============================================================================
