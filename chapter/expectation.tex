\chapter{结论与展望}
\section{本文工作总结}
本文的主要工作如下:
\begin{enumerate}[itemsep=0pt,parsep=0pt]
\item 构建旨在考察模型复杂空间推理能力的数据集。基于CLEVR数据集,本文构造了SRASP数据集,
并通过统计分析、双盲审核等机制,保证了数据集的质量和难度。
\item 设计了一种面向空间推理领域的神经符号框架。该框架包含视觉场景理解、语义解析、迭代反馈与规则修正、
规则蒸馏、ASP推理模块,并采用对比实验进行验证,证明该框架在不同的LLM上均可提升系统对空间推理问题的
解答能力,具有较强的泛化能力。
\item 设计实现了一个视觉问答系统。该原型系统以本文设计的面向空间推理领域的神经符号框架为核心,采用微服务设计理念
对各模块进行拆分,且采用了Redis缓存等方案对系统进行优化,以尽可能缩短响应时间,提升用户体验。
最终通过压力测试等技术手段,验证了该原型系统能够承受预想的用户体量和并发请求量,符合设计预期。
\end{enumerate}
\section{未来工作展望}
本文设计的面向空间推理的神经符号框架在视觉问答系统中,对增强其空间推理能力起到了积极作用,但仍有许多方面有待完善。未来的工作可以从以下几个方面展开。

迭代反馈机制的改进。目前,迭代反馈机制采用的是方案是,最多进行三轮迭代。如果在三轮迭代中,
某一轮生成的 ASP 程序能够顺利被 Clingo 执行(即不再出现语法、基础化或推理错误),
则认为程序已经达到预期的正确性,此时可以提前停止迭代。这一方案在一定程度上实现了自适应性,实现了
效率和效果之间的平衡,然而仍有进步的空间。可以考虑的改进方向有以下几点:
\begin{enumerate}[itemsep=0pt,parsep=0pt]
    \item 利用元学习或者贝叶斯优化的方法,自动搜索最优的迭代次数和反馈策略。
通过在不同任务或数据集上的交叉验证,可以自适应地调整迭代策略,使其在不同场景下达到最佳表现。
    \item 采用强化学习的方法,训练一个Agent,使其会根据当前生成 ASP 程序的状态和错误信息动态决定是否继续迭代。
类似于LLM-ARC中自动反馈调整的思路,Agent可以通过奖励信号来学习何时停止迭代,以达到最佳的平衡效果\cite{kalyanpur2024llmarcenhancingllmsautomated}。
\end{enumerate}

对数据集的改进。本文构造的SRASP数据集中,为了简化对相关问题的研究,采用的是Blender引擎
渲染后的静态的3D几何图像。然而目前应用领域对多模态模型的要求越来越高,静态图像为主的VQA数据集显然并不能
充分考察模型对动态空间变化的推理能力。此外,本文构造的SRASP数据集,图像的内容是相对较为简单的几何图形,
通过脚本进行生成而得,无法模拟真实物理世界中的物体运动、视角变化等动态场景。尤其是当前智能机器人领域发展
加快,对智能体的要求越来越高,智能体需要实时理解自身运动与物体位置的关系(如“转弯后如何避开障碍物”),而静态问答无法满足需求。
可以考虑的改进方向大致有以下几点:
\begin{enumerate}[itemsep=0pt,parsep=0pt]
    \item 生成动态空间问题。通过Unity或者Unreal等物理引擎的精确控制和程序化生成策略,在合成场景中模拟动态交互,自动生成涉及运动、视角变化和因果推理的动态空间问题。
    \item 增加人类视角问题。本文的SRASP数据集中的问题,均为相机视角下的问题。可以在生成问题时,
增加更多VQA系统在图像中的人类视角的问题,以考察VQA系统在视角适应上的能力。
\end{enumerate}