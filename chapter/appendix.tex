\chapter{附录}
\section{POVQAD数据集示例}
\subsection{约束模板}
\label{appendix:constraints}
\begin{lstlisting}
% 区域 R′ 中的每个对象 X 必须具有属性 P1′ 且取值 V1′。
:- object(X), at(X, R'), not has_property(X, P1', V1'). 
% 区域 R′ 中的每个对象 X 必须具有属性 P2′ 且取值 V2′。
:- object(X), at(X, R'), not has_property(X, P2', V2').
% 区域 R′ 中的任何对象 X 都不能具有属性 P1′ 且取值 V1′。
:- object(X), at(X, R'), has_property(X, P1', V1'). 
% 区域 R′ 中的任何对象 X 都不能具有属性 P2′ 且取值 V2′。
:- object(X), at(X, R'), has_property(X, P2', V2').
% 区域 R′ 中的每个对象 X 必须在属性 P1′ 上是 V1′,或在属性 P2′ 上是 V2′(否则不允许存在这样的 X)。
:- object(X), at(X, R'), not has_property(X, P1', V1'), not has_property(X, P2', V2').
% 在区域 R1′ 中,具有属性 P1′ 且取值 V1′ 的对象数量必。须恰好是 N′(否则该场景不合法)
:- #count{X: has_property(X, P1', V1'), object(X), at(X, R1')}!=N'.
% 在区域 R1′ 和区域 R2′ 之间,至少要有 N′ 对对象 (X1,X2) 满足它们在属性 P1′ 上取值相同,否则不允许。
:- #count{X1, X2: sameProperty(X1, X2, P1'), object(X1), object(X2), at(X1, R1'), at(X2, R2')}<N'.
% 在区域 R1′ 和区域 R2′ 之间,至少要有 N′ 对对象 (X1,X2),它们不仅在属性 P1′ 上取值相同,而且在属性 P2′ 上都取值 V2′。
:- #count{X1, X2: sameProperty(X1, X2, P1'), object(X1), object(X2), at(X1, R1'), at(X2, R2'), has_property(X1, P2', V2'), has_property(X2, P2', V2')}<N'.
% 在区域 R1′ 和区域 R2′ 之间,比 N′ 对对象 (X1,X2) 具有相同属性 P1′ 的情况是不允许的(即可有的相同对数最多为 N′-1)。
:- #count{X1, X2: sameProperty(X1, X2, P1'), object(X1), object(X2), at(X1, R1'), at(X2, R2')}>=N'.
% 在区域 R1′ 和区域 R2′ 之间,具有属性 P1′ 相同并且在属性 P2′ 上都取值 V2′ 的对象对数不能达到或超过 N′。
:- #count{X1, X2: sameProperty(X1, X2, P1'), object(X1), object(X2), at(X1, R1'), at(X2, R2'), has_property(X1, P2', V2'), has_property(X2, P2', V2')}>=N'.
% 遮挡关系的物体对数不少于N。
:- #count{X,Y : occludes(X,Y), object(X), object(Y)} < N.
% 每个被遮挡物体的遮挡率不超过R'。
:- occlusion_rate(T, R0), R0 > R_prime.
% 区域R中的物体最多有N个。
:- #count{X: object(X), at(X,R)} > N.
% 场景中至少有K对物体呈现“左-右”相邻的空间关系。
:- #count{X,Y: left(X,Y), object(X), object(Y)} < K.
% 目标物体的左侧有且只有L个物体。
:- #count{X: left(X,T), object(X)} != L.
% 在同一排(前/后关系)中,不同颜色的对象对数不少于 P。
:- #count{X,Y:
     same_row(X,Y),
     object(X),object(Y),
     hasProperty(X,color,C1),hasProperty(Y,color,C2),C1!=C2
   } < P.
% 
\end{lstlisting}
\subsection{全局约束}
\label{appendix:environment}
生成的每个场景都需要满足其对应环境中的所有约束。以下是POVQAD中所有环境共享的通用全局约束,所有约束均以ASP表示:
\begin{lstlisting}
1. property(color, gray). property(color, red).
2. property(color, blue). property(color, green).
3. property(color, brown). property(color, purple).
4. property(color, cyan). property(color, yellow).
5. property(shape, cube). property(shape, cylinder).
6. property(shape, sphere). property(shape, cone).
7. property(size, small). property(size, medium).
8. property(size, large).
9. property(material, rubber). property(material, metal).
10. region(0). region(1). region(2). region(3).
11. right(0, 0). right(0, 1). right(0, 3).
12. right(1, 1). 
13. right(2, 1). right(2, 2). right(2, 3).
14. right(3, 3).
15. left(R1, R2) :- right(R2, R1).
16. above(0, 0).
17. above(1, 1).
18. above(2, 0). above(2, 1). above(2, 2).
19. above(3, 0). above(3, 1). above(3, 3).
20. below(R1, R2) :- above(R2, R1).
21. sameProperty(X1, X2, P) :- has_property(X1,P,V),}
22. has_property(X2,P,V), X1!=X2.
23. same_color(X,Y):- sameProperty(X, Y, color).
24. same_size(X,Y):- sameProperty(X, Y, size).
25. same_shape(X,Y):- sameProperty(X, Y, shape).
26. same_material(X,Y):- sameProperty(X, Y, material).
27. 1{has_property(X, color, V) :
28. property(color, V)}1 :- obj(X).
29. 1{has_property(X, material, V) :
30. property(material, V)}1 :- obj(X).
31. 1{has_property(X, shape, V) :
32. property(shape, V)}1 :- obj(X).
33. 1{has_property(X, size, V) :
34. property(size, V)}1 :- obj(X).
35. 1{at(X, R): region(R)}1 :- obj(X).
36. :- sameProperty(X1, X2, color),
37. sameProperty(X1, X2, material),
38. sameProperty(X1, X2, size)},
39. sameProperty(X1, X2, shape),
40. obj(X1), obj(X2), X1!=X2.
41. :- left(X,Y), front(X,Y), below(X,Y). :- right(X,Y), behind(X,Y), below(X,Y).
\end{lstlisting}
以上通用规则对应的自然语言含义如下:
\begin{lstlisting}
1-9. 对象必须具有四个属性维度:颜色、形状、尺寸、材质。
1-4. 颜色属性取值范围(8种):灰色、红色、蓝色、绿色、棕色、紫色、青色、黄色。
5-6. 形状属性取值范围(4种):立方体、圆柱体、球体、圆锥体。
7-8. 尺寸属性取值范围(3级):小、中、大。
9. 材质属性取值范围(2种):橡胶、金属。
10. 场景划分为四个空间区域,编号为0、1、2、3。
11. 当对象A位于区域0时,其右侧对象B的合法区域:0/1/3。
12. 当对象A位于区域1时,其右侧对象B的合法区域:1。
13. 当对象A位于区域2时,其右侧对象B的合法区域:1/2/3。
14. 当对象A位于区域3时,其右侧对象B的合法区域:3。
15. 方位对称性规则:若对象A在对象B右侧,则对象B必在对象A左侧。
16. 当对象A位于区域0时,其上方对象B的合法区域:0。
17. 当对象A位于区域1时,其上方对象B的合法区域:1。
18. 当对象A位于区域2时,其上方对象B的合法区域:0/1/2。
19. 当对象A位于区域3时,其上方对象B的合法区域:0/1/3。
20. 方位对称性规则:若对象A在对象B上方,则对象B必在对象A下方。
27-28. 颜色属性强制单值约束:每个对象必须且只能具有一个颜色值。
29-30. 材质属性强制单值约束:每个对象必须且只能具有一个材质值。
31-32. 形状属性强制单值约束:每个对象必须且只能具有一个形状值。
33-34. 尺寸属性强制单值约束:每个对象必须且只能具有一个尺寸值。
35. 空间位置强制单值约束:每个对象必须且只能被分配至一个区域。
36-40. 对象差异性原则:任意两个对象不得在所有四个属性(颜色/形状/尺寸/材质)上完全一致。
41. 不存在一对对象既是左/右、前/后,又是上下相反(下/上)的。
\end{lstlisting}
以下用ASP表示的约束来表示特定环境。
\begin{lstlisting}
44. obj(0..4).
45. :- obj(X), at(X, 0),
    has_property(X, size, large).
46. :- obj(X), at(X, 0),
    has_property(X, shape, cylinder).
47. :- obj(X), at(X, 0),
    has_property(X, shape, cone).
48. :- obj(X), at(X, 1),
    has_property(X, size, small).
49. :- obj(X), at(X, 1),
    has_property(X, shape, cone).
50. :- obj(X), at(X, 1),
    has_property(X, material, rubber).
51. :- obj(X), at(X, 1),
    has_property(X, shape, cube).
52. :- obj(X), at(X, 2),
    not has_property(X, size, medium).
53. :- obj(X), at(X, 2),
    not has_property(X, material, metal).
54. :- obj(X), at(X, 2),
    has_property(X, material, rubber).
55. :- obj(X), at(X, 2),
    has_property(X, shape, sphere).
56. :- obj(X), at(X, 2),
    has_property(X, shape, cube).
57. :- obj(X), at(X, 3),
    has_property(X, size, small).
58 :- obj(X), at(X, 3),
    not has_property(X, material, metal),
59. not has_property(X, color, blue).
60. :- #count{X1, X2: sameProperty(X1, X2, shape),
61. obj(X1), obj(X2), at(X1, 3), at(X2, 2),
62. has_property(X1, color, yellow),
63. has_property(X2, color, yellow)} >= 4.
64. :- #count{X1, X2: sameProperty(X1, X2, color),
65. obj(X1), obj(X2),
66. {at(X1, 0), at(X2, 3)} >= 2.
\end{lstlisting}
以下是对前述的规则的逐行解释:
\begin{lstlisting}
44. 场景中共存在5个对象。
45. 区域0内禁止存在大尺寸对象。
46. 区域0内禁止存在圆柱形对象。
47. 区域0内禁止存在圆锥形对象。
48. 区域1内禁止存在小尺寸对象。
49. 区域1内禁止存在圆锥形对象。
50. 区域1内禁止存在橡胶材质对象。
51. 区域1内禁止存在立方体对象。
52. 区域2内所有对象必须为中尺寸。
53. 区域2内所有对象必须为金属材质。
54. 区域2内禁止存在橡胶材质对象。
55. 区域2内禁止存在球形对象。
56. 区域2内禁止存在立方体对象。
57. 区域3内禁止存在小尺寸对象。
58-59. 区域3内所有对象必须满足以下条件之一:金属材质或者蓝色外观。
60-63. 区域3与区域2内黄色对象组合规则:相同形状的黄色对象配对组数≤1。
64-66. 区域0与区域3联合约束: 具有相同颜色的对象配对组数 =0(严格禁止)。
\end{lstlisting}
\subsection{问题模板}
占位符的含义为:
<Z>、<C>、<M>、<S>:目标对象的大小、颜色、材质、形状;<Z1>,<C1>,…<S1>, <R1>,<R2>:参照对象及其空间关系
\subsubsection{位置类问题}
\begin{lstlisting}
Which region contains the <Z> <C> <M> <S> that is <R1> the <Z1> <C1> <M1> <S1> and <R2> the <Z2> <C2> <M2> <S2>?

In which region is the <Z> <C> <M> <S> located that is both <R1> the <Z1> <C1> <M1> <S1> and <R2> the <Z2> <C2> <M2> <S2>?

What region is the <Z> <C> <M> <S>—which lies <R1> the <Z1> <C1> <M1> <S1>—in?

Which of regions 0–3 does the <Z> <C> <M> <S> occupy, given that it is <R1> the <Z1> <C1> <M1> <S1> and between the <Z2> <C2> <M2> <S2> and the <Z3> <C3> <M3> <S3>?

The <Z> <C> <M> <S> that is both <R1> the <Z1> <C1> <M1> <S1> and <R2> the <Z2> <C2> <M2> <S2> is in which region?
\end{lstlisting}
\subsubsection{位置关系类问题}
What is the spatial relation <R> between the <Z> <C> <M> <S> and the <Z1> <C1> <M1> <S1>?

Which of the following describes how the <Z> <C> <M> <S> is positioned relative to the <Z1> <C1> <M1> <S1>: <R1> / <R2> / <R3>?

What is the relation between the <Z> <C> <M> <S> that is <R2> the <Z1> <C1> <M1> <S1> and the <Z2> <C2> <M2> <S2>?

Between <R1> and <R2>, which one holds for the <Z> <C> <M> <S> relative to the <Z1> <C1> <M1> <S1>?

What relation <R> holds between the <Z> <C> <M> <S> and each of <Z1> <C1> <M1> <S1> and <Z2> <C2> <M2> <S2>?
\subsubsection{与位置相关的计数类问题}
How many <S> are <R> the <Z> <C> <M> <S>?

How many <Z> <C> <M> <S> are both <R1> the <Z1> <C1> <M1> <S1> and <R2> the <Z2> <C2> <M2> <S2>?

What is the number of <S> that are <R> any of the <Z1> <C1> <M1> <S1>, <Z2> <C2> <M2> <S2>, and <Z3> <C3> <M3> <S3>?

How many objects are between the <Z1> <C1> <M1> <S1> and the <Z2> <C2> <M2> <S2>?

How many <S> that have <C> color are <R> the <Z> <C> <M> <S>?
\section{语义解析提示词}
\label{appendix:semantics-parsing-prompts}
\subsection{所处位置问题}
\begin{lstlisting}
任务描述:你是一个AI助手,负责将英文问题转换成ASP逻辑语言。
问题:What is the color of the object that is above the large blue metal cube?
ASP程序:query(Q) :-
    has_property(X, color, Q),above(X, Y),
    has_property(Y, size, large),has_property(Y, color, blue),
    has_property(Y, material, metal),has_property(Y, shape, cube),
    X != Y.

任务描述:你是一个 AI 助手,负责将英文问题转换成 ASP 逻辑语言。
问题:What is the shape of the object to the left of the large red rubber sphere?
ASP程序:query(Q) :-
    has_property(X, shape, Q),left(X, Y),
    has_property(Y, size, large),has_property(Y, color, red),
    has_property(Y, material, rubber),has_property(Y, shape, sphere),
    X != Y.

任务描述:你是一个 AI 助手,负责将英文问题转换成 ASP 逻辑语言。
问题:What is the material of the object in front of the small green cylinder?
ASP程序:query(Q) :-
    has_property(X, material, Q),front(X, Y),
    has_property(Y, size, small),has_property(Y, color, green),
    has_property(Y, shape, cylinder),X != Y.
\end{lstlisting}
\subsection{位置关系问题}
\begin{lstlisting}
任务描述:你是一个 AI 助手,负责将英文问题转换成 ASP 逻辑语言。
问题:Is the small purple cube to the right of the large yellow sphere?
ASP程序:query(right(X, Y)) :-
    has_property(X, size, small),has_property(X, color, purple),
    has_property(X, shape, cube),has_property(Y, size, large),
    has_property(Y, color, yellow),has_property(Y, shape, sphere),X != Y.

任务描述:你是一个 AI 助手,负责将英文问题转换成 ASP 逻辑语言。 
问题:Is the red cylinder above the blue cube?
ASP程序:query(above(X, Y)) :-
    has_property(X, color, red),has_property(X, shape, cylinder),
    has_property(Y, color, blue),has_property(Y, shape, cube),X != Y.


\end{lstlisting}
\subsection{与位置相关的计数问题}
\begin{lstlisting}
任务描述:你是一个 AI 助手,负责将英文问题转换成 ASP 逻辑语言。
问题:How many cylinders are both above the green cube and behind the yellow sphere?
ASP程序:query(count(X)) :-
    has_property(X, shape, cylinder),has_property(Y, color, green),
    has_property(Y, shape, cube),has_property(Z, color, yellow),
    has_property(Z, shape, sphere),above(X, Y),
    behind(X, Z),X != Y, X != Z.

任务描述:你是一个 AI 助手,负责将英文问题转换成 ASP 逻辑语言。
问题:How many rubber objects are behind the small blue sphere?
ASP程序:query(count(X)) :-
    has_property(X, material, rubber),behind(X, Y),
    has_property(Y, size, small),has_property(Y, color, blue),
    has_property(Y, shape, sphere),X != Y.

任务描述:你是一个 AI 助手,负责将英文问题转换成 ASP 逻辑语言。
问题:How many small cubes are to the right of the large green cylinder?
ASP程序:query(count(X)) :-
    has_property(X, size, small),has_property(X, shape, cube),
    right(X, Y),has_property(Y, size, large),
    has_property(Y, color, green),has_property(Y, shape, cylinder),
    X != Y.

任务描述:你是一个 AI 助手,负责将英文问题转换成 ASP 逻辑语言。
问题:How many blue spheres are both in front of the red cube and to the left of the yellow cylinder?
ASP程序:query(count(X)) :-
    has_property(X, color, blue),has_property(X, shape, sphere),
    has_property(Y, color, red),has_property(Y, shape, cube),
    has_property(Z, color, yellow),has_property(Z, shape, cylinder),
    front(X, Y),left(X, Z),X != Y, X != Z.
\end{lstlisting}
\section{ASP规则修正提示词}
\label{appendix:rule-fix}
\subsection{语法错误}
这类错误通常发生在程序违反了ASP的语法规则时。错误信息会指出文件、行号、列号以及意料之外的符号 。例如,遗漏句点、括号不匹配或使用了Clingo版本不支持的语法都可能导致语法错误 。
以下是该错误的一个提示词模板:
\begin{lstlisting}
任务描述:请根据提供的 ASP 程序和 Clingo 错误信息,对程序进行修正。

错误类型:语法错误 (Syntax Error) - 规则或事实缺少句号。

错误程序:
node(a).
node(b) % <- 错误:这里缺少句号
edge(a, b).

path(X, Y) :- edge(X, Y).
path(X, Z) :- edge(X, Y), path(Y, Z).

Clingo错误信息:
<stdin>:3:1: error: syntax error, unexpected IDENTIFIER, expecting '.' or ':
  node(b)

修正后程序:
node(a).
node(b). % <- 修正:添加句号
edge(a, b).

path(X, Y) :- edge(X, Y).
path(X, Z) :- edge(X, Y), path(Y, Z).
\end{lstlisting}
\subsection{谓词未定义}
这通常是一个警告,表示某个谓词在规则体中被使用,但从未在任何规则的头部定义过 。
程序执行不会因此停止,但可能暗示着逻辑错误 。以下是该错误的一个提示词模板:
\begin{lstlisting}
任务描述:请根据提供的 ASP 程序和 Clingo 错误信息,对程序进行修正。

错误类型:未定义的谓词(Undefined Predicate)

错误程序:
p(a) :- q(a).

Clingo错误信息:
error: in rule at line 1: undefined predicate: q/1

修正后程序:
q(a).
p(a) :- q(a).
\end{lstlisting}
\subsection{循环负依赖}
循环负依赖发生在程序的依赖图中存在负循环,即一个谓词通过一系列规则负依赖于自身。
负循环可能导致程序没有解答集。​此外在某些情况下,可能导致多个解答集的产生,增加了解释的复杂性。以下是该错误的一个提示词模板:
\begin{lstlisting}
任务描述:请根据提供的 ASP 程序和 Clingo 错误信息,对程序进行修正。

错误类型:​递归中的负循环(Negative Cycle in Recursion)​

错误程序:
p(X) :- not q(X).
q(X) :- not p(X).

Clingo错误信息:
error: in rule at line 2: cyclic dependency: p/1 -> q/1 -> p/1

修正后程序:
p(X) :- r(X), not q(X).
q(X) :- s(X), not p(X).
\end{lstlisting}
\subsection{重复的规则或者事实}
这类错误往往是由于人为输入错误或多个ASP程序进行合并时,没有去重导致。以下是该错误的一个提示词模板:
\begin{lstlisting}
任务描述:请根据提供的 ASP 程序和 Clingo 错误信息,对程序进行修正。

错误类型:​重复的规则或事实(Duplicate Rules or Facts)​

错误程序:
p(a).
p(a).

Clingo错误信息:
warning: fact at line 2 is a duplicate of fact at line 1

修正后程序:
p(a).
\end{lstlisting}
\subsection{约束条件恒为假}
在ASP中,约束条件的作用是排除使Body为真的解答集。
当约束条件的Body部分恒为真时,该约束条件始终排除所有可能的解答集。出现这一错误时,由于所有可能的解答集都被排除,程序将没有解答集,即程序不一致。
\begin{lstlisting}
任务描述:请根据提供的 ASP 程序和 Clingo 错误信息,对程序进行修正。

错误类型:​约束条件总为假(Constraint Always False)

错误程序:
:- p(X), not p(X).

Clingo错误信息:
warning: constraint at line 1 is always false

修正后程序:
:- p(X), not q(X).
\end{lstlisting}
\subsection{不安全变量}
当一个变量出现在规则的头部,但没有在规则体的任何肯定字面量中出现时,就会发生这种错误 。这可能会导致产生无限的稳定模型 。Clingo会报告出错的规则以及不安全的变量名 。以下是该错误的一个提示词模板:
\begin{lstlisting}
任务描述:请根据提供的 ASP 程序和 Clingo 错误信息,对程序进行修正。

错误类型:基础化错误 (Grounding Error) - 规则中存在不安全变量。

错误程序:
reachable(X) :- edge(Y, X). % <- 错误:变量 Y 是不安全的,因为它没有在规则头或任何正文字面量中安全出现

Clingo错误信息:
<stdin>:5:20-21: error: variable Y is unsafe
  reachable(X) :- edge(Y, X).

修正后程序:
reachable(X) :- edge(Y, X), node(Y). % <- 修正:添加 node(Y) 来约束 Y 的范围
\end{lstlisting}
\subsection{不一致的事实和规则}
如果程序中定义的事实和规则之间存在矛盾,会导致出现逻辑错误。以下是该错误的一个提示词模板:
\begin{lstlisting}
任务描述:请根据提供的 ASP 程序和 Clingo 错误信息,对程序进行修正。

错误类型:逻辑错误 (Logical Error) - 程序包含直接冲突的规则或事实,导致没有稳定模型。

错误程序:
light_on. % 灯是开着的

:- light_on. % 约束:灯不能是开着的

Clingo 错误信息:
Reading from <stdin>
Solving...
UNSATISFIABLE

Models       : 0
Calls        : 1
Time         : 0.000s (Solving: 0.00s 1st Model: 0.00s Unsat: 0.00s)
CPU Time     : 0.000s

修正后程序 (根据意图选择修正): (修正方法取决于真实意图。这里假设约束是错误的)
light_on. % 灯是开着的

% :- light_on. % <- 修正:注释掉或删除冲突的约束
(或者,如果事实是错误的)
% light_on. % <- 修正:注释掉或删除冲突的事实

:- light_on. % 约束:灯不能是开着的
\end{lstlisting}
\section{规则蒸馏预提示}
\label{appendix:preprompt}
\subsection{任务介绍}
\label{appendix:task-introduction}
我们目前在研究视觉问答的相关问题,需要你完成一系列任务。
任务包括接收一张图像和与该图像相关的问题作为输入,并产生正确的答案作为输出。

我们已经将图像和问题都预处理成了正确的答案集编程(Answer Set Programming,ASP)表示形式。
场景/问题对的ASP事实作为ASP程序的实例,我们称之为理论(Theory)。
这是一个规则集合,用于处理输入实例并计算出正确的答案。

你的任务是帮助我们随着新的问题实例的出现,扩展这个理论,添加新的规则。
在接下来的部分,我们会先给出一些定义,然后提出具体的任务。
\subsection{语言语法}
\label{appendix:asp-grammar}
答案集编程(Answer Set Programming,ASP)是一种面向解决困难搜索问题的声明式编程范式。
其语法和用法可以总结如下:

规则(Rules):ASP程序的基本构建模块。一个规则包含头部(head)和体部(body),其书写形式为:头部 :- 体部。
这意味着如果体部为真,那么头部也为真。
体部中的谓词之间用逗号分隔,而不是像Prolog中那样用分号。
例如:flies(tweety) :- bird(tweety), not penguin(tweety). (如果tweety是鸟并且不是企鹅,那么tweety会飞。)

原子(Atoms):这是基本命题,可以是任何以小写字母开头的字符和数字组成的字符串。

字面量(Literals):一个原子或其否定。ASP中使用的否定是失败即否定(negation as failure),用not表示。
例如,not a 表示无法证明a为真。

事实(Facts):这是没有体部的规则,声明一些无条件为真的事情。
例如:bird(tweety). (tweety是鸟。)

约束(Constraints):这是没有头部的规则,用于排除某些答案。
例如::- not fly(tweety). (任何tweety不飞的答案集都是不可接受的。)

选择规则(Choice Rules):这些规则允许生成多个答案,表达原子可以被自由选择是否包含在答案集中。
例如:{fish(tweety);bird(tweety)} :- penguin(tweety). (意味着企鹅tweety可能是鱼,可能是鸟,也可能都不是。)

注释(Comments):在ASP中,注释以%开始,直到行尾。它们会被ASP求解器忽略。

在ASP中,不允许在规则的体部使用;来表示逻辑析取。
相反,我们需要通过单独的规则来表达析取。

在ASP中,当一个变量出现在规则的头部或体部的否定字面量中,但没有出现在体部的肯定字面量中时,该变量被认为是**不安全(unsafe)**的。
不安全变量的示例:

% 示例1:否定谓词中存在未绑定的Y,常规谓词中存在S和T
p1(X) :- not q(X,Y), r(S), s(T).

% 示例2:否定谓词中存在未绑定的Y,常规谓词中存在S,否定谓词中存在T
p2(X, B) :- not q(Y), r(S), not s(T).
\subsection{场景与问题解释}
\label{appendix:scene-question-explanation}
请看以下用于表示图像中物体的ASP形式。

它由可能多个形如 'obj(ID,X,Y,M,C,F,S)' 的谓词组成,
其中 'ID' 是定义物体的唯一标识符,'X, Y' 是介于 0 和 10 之间的坐标,
'M' 是材质(material),'C' 是颜色(color),'F' 是形状(form),'S' 是尺寸(size)。
以下是这种编码的一个示例片段:

obj(0,324,201,rubber,purple,sphere,large). (ID为0的物体,坐标为(324, 201),材质是橡胶,颜色是紫色,形状是球体,尺寸是大的。)
obj(1,282,166,rubber,purple,cylinder,small). (ID为1的物体,坐标为(282, 166),材质是橡胶,颜色是紫色,形状是圆柱体,尺寸是小的。)
obj(2,216,94,metal,blue,sphere,large).   (ID为2的物体,坐标为(216, 94),材质是金属,颜色是蓝色,形状是球体,尺寸是大的。)
obj(3,127,115,metal,green,cube,large).   (ID为3的物体,坐标为(127, 115),材质是金属,颜色是绿色,形状是立方体,尺寸是大的。)
\subsection{答案格式}
\begin{lstlisting}
理论计算出的答案将总是采用 'ans(X)' 形式的谓词,其中 'X' 是预期的答案。
'X' 可以是一个布尔值(是或否),它可以是一个属性、一个关系,或某个属性的值。

这些例子包括:
ans(yes)
ans(red)
ans(man)
ans(left)
ans(pose)
\end{lstlisting}
\subsection{初始ASP知识库}
\label{appendix:initial-theory}
初始ASP知识库分为三个版本,不同版本的知识库可以根据任务需求选择,以优化推理性能。例如简单任务可以使用简化ASP知识库,
而复杂任务则使用完整ASP知识库。
\subsubsection{完整ASP知识库}
\begin{lstlisting}
scene(X) :- scene(X,_).

% 辅助谓词
object(ID) :- obj(ID,_,_,_,_,_,_).
position(ID,X,Y) :- obj(ID,X,Y,_,_,_,_).
has_size(ID,SIZE) :- obj(ID,_,_,_,_,_,SIZE).
has_color(ID,COLOR) :- obj(ID,_,_,_,COLOR,_,_).
has_material(ID,MATERIAL):- obj(ID,_,_,MATERIAL,_,_,_).
has_shape(ID,SHAPE) :- obj(ID,_,_,_,_,SHAPE,_).

left(ID,ID') :- position(ID,X,Y), position(ID',X',Y'), state(T',ID'), ID!=ID', X<X'.
right(ID,ID') :- position(ID,X,Y), position(ID',X',Y'), state(T',ID'), ID!=ID', X>=X'.
front(ID,ID') :- position(ID,X,Y), position(ID',X',Y'), state(T',ID'), ID!=ID', Y>Y'.
behind(ID,ID') :- position(ID,X,Y), position(ID',X',Y'), state(T',ID'), ID!=ID', Y<=Y'.

% 唯一性规则/约束
state(T+1,ID) :- unique(T), state(T,ID).
:- unique(T), state(T,ID), state(T,ID'), ID!=ID'.

% 空间关系规则
state(T+1,ID) :- relate_left(T), state(T,ID'), left(ID,ID').
state(T+1,ID) :- relate_right(T), state(T,ID'), right(ID,ID').
state(T+1,ID) :- relate_front(T), state(T,ID'), front(ID,ID') .
state(T+1,ID) :- relate_behind(T), state(T,ID'), behind_of(ID,ID').

% 计数规则
int(T+1,V) :- count(T), #count{ ID : state(T,ID) } = V.

% 存在性规则
bool(T+1,yes) :- exist(T), state(T,ID).
bool(T+1,no) :- exist(T), not bool(T+1,yes).

% 过滤规则
state(T+1,ID) :- filter_large(T), state(T,ID), has_size(ID,large).
state(T+1,ID) :- filter_small(T), state(T,ID), has_size(ID,small).
state(T+1,ID) :- filter_gray(T), state(T,ID), has_color(ID,gray).
state(T+1,ID) :- filter_red(T), state(T,ID), has_color(ID,red).
state(T+1,ID) :- filter_blue(T), state(T,ID), has_color(ID,blue).
state(T+1,ID) :- filter_green(T), state(T,ID), has_color(ID,green).
state(T+1,ID) :- filter_brown(T), state(T,ID), has_color(ID,brown).
state(T+1,ID) :- filter_purple(T), state(T,ID), has_color(ID,purple).
state(T+1,ID) :- filter_cyan(T), state(T,ID), has_color(ID,cyan).
state(T+1,ID) :- filter_yellow(T), state(T,ID), has_color(ID,yellow).
state(T+1,ID) :- filter_metal(T), state(T,ID), has_material(ID,metal).
state(T+1,ID) :- filter_rubber(T), state(T,ID), has_material(ID,rubber).
state(T+1,ID) :- filter_sphere(T), state(T,ID), has_shape(ID,sphere).
state(T+1,ID) :- filter_cylinder(T), state(T,ID), has_shape(ID,cylinder).
state(T+1,ID) :- filter_cube(T), state(T,ID), has_shape(ID,cube).

% 查询函数
size(T+1,SIZE) :- query_size(T), state(T,ID), has_size(ID,SIZE).
color(T+1,COLOR) :- query_color(T), state(T,ID), has_color(ID,COLOR).
material(T+1,MATERIAL) :- query_material(T), state(T,ID), has_material(ID,MATERIAL).
shape(T+1,SHAPE) :- query_shape(T), state(T,ID), has_shape(ID,SHAPE).

% 逻辑运算符
state(T+1,ID) :- and(T,T'), state(T,ID), state(T',ID).

state(T+1,ID) :- or(T,T'), state(T,ID).
state(T+1,ID') :- or(T,T'), state(T',ID').

bool(T+1, yes) :- boolean_negation(T), bool(T, no).
bool(T+1, no) :- boolean_negation(T), not bool(T+1, yes).

% 相同属性关系
state(T+1,ID') :- same_size(T), state(T,ID), has_size(ID,SIZE), has_size(ID',SIZE), ID!=ID'.
state(T+1,ID') :- same_color(T), state(T,ID), has_color(ID,COLOR), has_color(ID',COLOR), ID!=ID'.
state(T+1,ID') :- same_material(T), state(T,ID), has_material(ID,MATERIAL), has_material(ID',MATERIAL), ID!=ID'.
state(T+1,ID') :- same_shape(T), state(T,ID), has_shape(ID,SHAPE), has_shape(ID',SHAPE), ID!=ID'.

% 整数比较
bool(T+1,yes) :- equal_integer(T,T'), int(T,V), int(T',V'), V=V'.
bool(T+1,no) :- equal_integer(T,T'), not bool(T+1,yes).

bool(T+1,yes) :- less_than(T,T'), int(T,V), int(T',V'), V<V'.
bool(T+1,no) :- less_than(T,T'), not bool(T+1,yes).

bool(T+1,yes) :- greater_than(T,T'), int(T,V), int(T',V'), V>V'.
bool(T+1,no) :- greater_than(T,T'), not bool(T+1,yes).

% 属性比较
bool(T+1,yes) :- equal_size(T,T'), size(T,V), size(T',V'), V=V'.
bool(T+1,no) :- equal_size(T,T'), not bool(T+1,yes).

bool(T+1,yes) :- equal_color(T,T'), color(T,V), color(T',V'), V=V'.
bool(T+1,no) :- equal_color(T,T'), not bool(T+1,yes).

bool(T+1,yes) :- equal_material(T,T'), material(T,V), material(T',V'), V=V'.
bool(T+1,no) :- equal_material(T,T'), not bool(T+1,yes).

bool(T+1,yes) :- equal_shape(T,T'), shape(T,V), shape(T',V'), V=V'.
bool(T+1,no) :- equal_shape(T,T'), not bool(T+1,yes).

% 状态规则
state(0,ID) :- object(ID).
state(T+1,ID) :- scene(T), object(ID).

% 得出答案(T必须等于最后一个时间点)
ans(V) :- end(T), size(T,V).
ans(V) :- end(T), color(T,V).
ans(V) :- end(T), material(T,V).
ans(V) :- end(T), shape(T,V).
ans(V) :- end(T), bool(T,V).
ans(V) :- end(T), int(T,V).

:- not ans(_).

#show ans/1.
\end{lstlisting}
\subsubsection{简化ASP知识库}
\begin{lstlisting}
% 状态规则
state(0,ID) :- object(ID).
state(T+1,ID) :- scene(T), object(ID).

scene(X) :- scene(X,Y).

object(ID) :- obj(ID,_,_,_,_,_,_).
position(ID,X,Y) :- obj(ID,X,Y,_,_,_,_).
has_size(ID,SIZE) :- obj(ID,_,_,_,_,_,SIZE).
has_color(ID,COLOR) :- obj(ID,_,_,_,COLOR,_,_).
has_material(ID,MATERIAL):- obj(ID,_,_,MATERIAL,_,_,_).
has_shape(ID,SHAPE) :- obj(ID,_,_,_,_,SHAPE,_).

left(ID,ID') :- position(ID,X,Y), position(ID',X',Y'), state(T',ID'), ID!=ID', X<X'.
right(ID,ID') :- position(ID,X,Y), position(ID',X',Y'), state(T',ID'), ID!=ID', X>=X'.
front(ID,ID') :- position(ID,X,Y), position(ID',X',Y'), state(T',ID'), ID!=ID', Y>Y'.
behind_of(ID,ID') :- position(ID,X,Y), position(ID',X',Y'), state(T',ID'), ID!=ID', Y<=Y'.

% 得出答案(T必须等于最后一个时间点)
ans(V) :- end(T), size(T,V).
ans(V) :- end(T), color(T,V).
ans(V) :- end(T), material(T,V).
ans(V) :- end(T), shape(T,V).
ans(V) :- end(T), bool(T,V).
ans(V) :- end(T), int(T,V).

:- not ans(_).

#show ans/1.
\end{lstlisting}
\subsubsection{适中ASP知识库}
\begin{lstlisting}
% 空间关系规则
state(T+1,ID) :- relate_left(T), state(T,ID'), left(ID,ID').
% 过滤规则
state(T+1,ID) :- filter_large(T), state(T,ID), has_size(ID,large).

% 查询函数
size(T+1,SIZE) :- query_size(T), state(T,ID), has_size(ID,SIZE).

% 相同属性规则
state(T+1,ID') :- same_size(T), state(T,ID), has_size(ID,SIZE), has_size(ID',SIZE), ID!=ID'.

% % 整数比较
bool(T+1,yes) :- equal_integer(T,T'), int(T,V), int(T',V'), V=V'.
bool(T+1,no) :- equal_integer(T,T'), not bool(T+1,yes).

% 属性比较
bool(T+1,yes) :- equal_size(T,T'), size(T,V), size(T',V'), V=V'.
bool(T+1,no) :- equal_size(T,T'), not bool(T+1,yes).

% 状态规则
state(0,ID) :- object(ID).
state(T+1,ID) :- scene(T), object(ID).

scene(X) :- scene(X,Y).

object(ID) :- obj(ID,_,_,_,_,_,_).
position(ID,X,Y) :- obj(ID,X,Y,_,_,_,_).
has_size(ID,SIZE) :- obj(ID,_,_,_,_,_,SIZE).
has_color(ID,COLOR) :- obj(ID,_,_,_,COLOR,_,_).
has_material(ID,MATERIAL):- obj(ID,_,_,MATERIAL,_,_,_).
has_shape(ID,SHAPE) :- obj(ID,_,_,_,_,SHAPE,_).

left(ID,ID') :- position(ID,X,Y), position(ID',X',Y'), state(T',ID'), ID!=ID', X<X'.
right(ID,ID') :- position(ID,X,Y), position(ID',X',Y'), state(T',ID'), ID!=ID', X>=X'.
front(ID,ID') :- position(ID,X,Y), position(ID',X',Y'), state(T',ID'), ID!=ID', Y>Y'.
behind_of(ID,ID') :- position(ID,X,Y), position(ID',X',Y'), state(T',ID'), ID!=ID', Y<=Y'.

% 得出答案(T必须等于最后一个时间点)
ans(V) :- end(T), size(T,V).
ans(V) :- end(T), color(T,V).
ans(V) :- end(T), material(T,V).
ans(V) :- end(T), shape(T,V).
ans(V) :- end(T), bool(T,V).
ans(V) :- end(T), int(T,V).

:- not ans(_).

#show ans/1.
\end{lstlisting}
\subsection{任务说明}
\label{appendix:task-explanation}
以下为RCNSP运行前构建ASP知识库的规则蒸馏的预提示环节所用的任务说明提示词:
\begin{lstlisting}
你的任务是保持ASP理论的更新,添加能够处理问题的规则。
我们提供了一个可以处理一些实例的初始理论。
提示输入将包含一个或多个ASP表示的问题。
如果存在多个问题,它们将用#符号分隔。

你的任务是接收问题,并向理论添加规则,使其能够处理这些问题。

这非常重要!
如果你违反以下任何规则,输出将是错误的!

你只能向理论添加规则。
你不能从理论中删除任何规则。
不要向理论添加事实。
不要向理论添加注释。
如果你添加一条规则,它必须只有一个谓词作为头部,并且该谓词应该只包含变量,除非绝对必要。
添加的规则必须位于注释 '% Added rules to handle new instances' 和 '% End of Theory' 之间。
不要删除这些注释。
只输出完整的ASP理论,包含原始规则和新规则。
如果你输出任何自然语言,答案都是不正确的!
不要编写任何非显式的ASP规则。
不要将提示添加到理论中。

添加新规则时,请记住:
在ASP中,不允许在规则的体部使用;来表示逻辑析取。
相反,我们需要通过单独的规则来表达析取。

在ASP中,当一个变量出现在规则的头部或体部的否定字面量中,但没有出现在体部的肯定字面量中时,该变量被认为是**不安全(unsafe)**的。
不安全变量的示例:
不安全:a(X) :- not b(X). 这里,X是不安全的,因为它没有在体部的肯定字面量中定义。
不安全:a(Y) :- b(X). 这里,Y是不安全的,因为它没有在体部的肯定字面量中定义。
不安全:a(TO, no) :- not b(TO, TI), c(TI, ID).
安全:a(X) :- b(X), not c(X). 在这种情况下,X是安全的,因为它出现在肯定字面量 b(X) 中。
\end{lstlisting}
以下为处理VQA请求时补充ASP知识库的规则蒸馏的预提示环节所用的任务说明提示词:
\begin{lstlisting}
你的任务是保持ASP理论的更新,添加能够处理问题的规则。
我们提供了一个可以处理一些实例的初始理论。
提示输入将包含一个或多个ASP表示的问题。
如果只有一个问题,它将附带其相关的ASP场景编码。
如果存在多个问题,它们将用#符号分隔。

你的任务是接收问题,并向理论添加规则,使其能够处理这些问题。

这非常重要!
如果你违反以下任何规则,输出将是错误的!

只输出新的ASP规则。
不要添加输出ASP事实。
新规则应尽可能通用,即包含较少的常量和较多的变量。
新规则应尽可能通用。
返回新规则时不要使用代码块格式,只需纯文本。
不要添加任何不安全的规则。
如果你不确定,可以添加多个规则,但这些规则不应相互矛盾。
不要添加注释也不要输出解释。
不要输出任何自然语言。
只以纯文本形式输出新的规则。
不要输出任何非显式的ASP规则。
如果你输出任何自然语言,答案都是错误的!
\end{lstlisting}
以下为规则生成环节所用的提示词:
\begin{lstlisting}
你的任务是使用新的规则扩展我们提供的初始ASP理论,这些规则能够计算出正确的答案。
提示输入将包含一个或多个ASP表示的问题。
如果存在多个问题,它们将用#符号分隔。

你的任务是接收ASP问题,并返回能够正确处理这些问题的ASP规则。
识别理论中缺失的谓词并实现它。

这非常重要!
如果你违反以下任何规则,输出将是错误的!

只输出新的ASP规则。
不要添加输出ASP事实。
新规则应尽可能通用,即包含较少的常量和较多的变量。
新规则应尽可能通用。
返回新规则时不要使用代码块格式,只需纯文本。
不要添加任何不安全的规则。
请记住:

在ASP中,不允许在规则的体部使用;来表示逻辑析取。
相反,我们需要通过单独的规则来表达析取。

在ASP中,当一个变量出现在规则的头部或体部的否定字面量中,但没有出现在体部的肯定字面量中时,该变量被认为是**不安全(unsafe)**的。
不安全变量的示例:
不安全:a(X) :- not b(X). 这里,X是不安全的,因为它没有在体部的肯定字面量中定义。
不安全:a(Y) :- b(X). 这里,Y是不安全的,因为它没有在体部的肯定字面量中定义。
不安全:a(TO, no) :- not b(TO, TI), c(TI, ID).
安全:a(X) :- b(X), not c(X). 在这种情况下,X是安全的,因为它出现在肯定字面量 b(X) 中。
\end{lstlisting}
\section{求解结果翻译}
\label{appendix:result-translate}
\begin{lstlisting}
以下是ASP谓词到自然语言表述的同义词字典,你需要学习这个字典中的对应含义,以能够准确地将ASP求解器输出的结果翻译为自然语言。

color(o1, red)	“红色的”
shape(o1, cube)	“立方体”
size(o1, large)	“大的”
material(o1, metal)	“金属的”
object(o1)	“一个物体(o1)”
visible(o1)	“在可见区域中”
not visible(o1)	“在不可见区域中”
left(o1, o2)	“o1 在 o2 的左侧”
right(o1, o2)	“o1 在 o2 的右侧”
front(o1, o2)	“o1 在 o2 的前面”
behind(o1, o2)	“o1 在 o2 的后面”
inside(o1, o2) “o1 在 o2 的内部”
above(o1, o2) “o1 在 o2 的上方”
below(o1, o2) “o1 在 o2 的下方”
outside(o1, o2) “o1 在 o2 的外部”
partially_occludes(o1, o2) “o1 将 o2 部分遮挡”
fully_occludes(o1, o2) “o1 将 o2 全部遮挡”
same_row(o1, o2)	“o1 和 o2 在同一排”
answer(yes)	“是”
answer(no)	“否”
answer(o1)	“答案是 o1”
cannot_answer(q1)	“由于信息不全,无法回答问题 q1”
on_table(o1)	“o1 放置在桌面上”
supported(o1, o2)	“o1 放在 o2 上”
\end{lstlisting}