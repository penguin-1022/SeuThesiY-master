\chapter{附录}
\section{SRASP数据集示例例}
\subsection{环境}
\label{appendix:environment}
生成的每个场景都需要满足其对应环境中的所有约束。以下是SRASP中所有环境共享的通用规则,所有规则均以ASP表示:
\begin{lstlisting}
1. property(color, gray). property(color, red).
2. property(color, blue). property(color, green).
3. property(color, brown). property(color, purple).
4. property(color, cyan). property(color, yellow).
5. property(shape, cube). property(shape, cylinder).
6. property(shape, sphere). property(shape, cone).
7. property(size, small). property(size, medium).
8. property(size, large).
9. property(material, rubber).
property(material, metal).
10. region(0). region(1). region(2). region(3).
11. right_R(0, 0). right_R(0, 1). right_R(0, 2).
right_R(0, 3).
12. right_R(1, 1). right_R(1, 3).
13. right_R(2, 0). right_R(2, 1). right_R(2, 2).
right_R(2, 3).
14. right_R(3, 1). right_R(3, 3).
15. left_R(R1, R2) :- right_R(R2, R1).
16. front_R(0, 0). front_R(0, 1). front_R(0, 2).
front_R(0, 3).
17. front_R(1, 0). front_R(1, 1). front_R(1, 2).
front_R(1, 3).
18. front_R(2, 2). front_R(2, 3).
19. front_R(3, 2). front_R(3, 3).
20. behind_R(R1, R2) :- front_R(R2, R1).
21. sameProperty(X1, X2, P) :- hasProperty(X1,P,V),}
22. hasProperty(X2,P,V), X1!=X2.
23. same_color(X,Y):- sameProperty(X, Y, color).
24. same_size(X,Y):- sameProperty(X, Y, size).
25. same_shape(X,Y):- sameProperty(X, Y, shape).
26. same_material(X,Y):- sameProperty(X, Y, material).
27. 1{hasProperty(X, color, V) :
28. property(color, V)}1 :- object(X).
29. 1{hasProperty(X, material, V) :
30. property(material, V)}1 :- object(X).
31. 1{hasProperty(X, shape, V) :
32. property(shape, V)}1 :- object(X).
33. 1{hasProperty(X, size, V) :
34. property(size, V)}1 :- object(X).
35.1{at(X, R): region(R)}1 :- object(X).
36.:- sameProperty(X1, X2, color),
37. sameProperty(X1, X2, material),
38. sameProperty(X1, X2, size)},
39. sameProperty(X1, X2, shape),
40. object(X1), object(X2), X1!=X2.
41.exceed_region_capacity(R) :42. #count{X: object(X), at(X, R)} >= 4, region(R).
43:- exceed_region_capacity(_).
\end{lstlisting}
以上通用规则对应的自然语言含义如下:
\begin{lstlisting}
1-9. 对象必须具有四个属性维度:颜色、形状、尺寸、材质。
1-4. 颜色属性取值范围(8种):灰色、红色、蓝色、绿色、棕色、紫色、青色、黄色。
5-6. 形状属性取值范围(4种):立方体、圆柱体、球体、圆锥体。
7-8. 尺寸属性取值范围(3级):小、中、大。
9. 材质属性取值范围(2种):橡胶、金属。
10. 场景划分为四个空间区域,编号为0、1、2、3。
11. 当对象A位于区域0时,其右侧对象B的合法区域:0/1/2/3。
12. 当对象A位于区域1时,其右侧对象B的合法区域:1/3。
13. 当对象A位于区域2时,其右侧对象B的合法区域:0/1/2/3。
14. 当对象A位于区域3时,其右侧对象B的合法区域:1/3。
15. 方位对称性规则:若对象A在对象B右侧,则对象B必在对象A左侧。
16. 当对象A位于区域0时,其前方对象B的合法区域:0/1/2/3。
17. 当对象A位于区域1时,其前方对象B的合法区域:0/1/2/3。
18. 当对象A位于区域2时,其前方对象B的合法区域:2/3。
19. 当对象A位于区域3时,其前方对象B的合法区域:2/3。
20. 方位对称性规则:若对象A在对象B前方,则对象B必在对象A后方。
27-28. 颜色属性强制单值约束:每个对象必须且只能具有一个颜色值。
29-30. 材质属性强制单值约束:每个对象必须且只能具有一个材质值。
31-32. 形状属性强制单值约束:每个对象必须且只能具有一个形状值。
33-34. 尺寸属性强制单值约束:每个对象必须且只能具有一个尺寸值。
35. 空间位置强制单值约束:每个对象必须且只能被分配至一个区域。
36-40. 对象差异性原则:任意两个对象不得在所有四个属性(颜色/形状/尺寸/材质)上完全一致。
41-43. 区域容量限制:每个空间区域最多容纳3个对象。
\end{lstlisting}
以下用ASP表示的约束,用来表示图\ref{}所示场景所属的特定环境。
\begin{lstlisting}
44. object(0..4).
45. :- object(X), at(X, 0),
hasProperty(X, size, large).
46. :- object(X), at(X, 0),
hasProperty(X, shape, cylinder).
47. :- object(X), at(X, 0),
hasProperty(X, shape, cone).
48. :- object(X), at(X, 1),
hasProperty(X, size, small).
49. :- object(X), at(X, 1),
hasProperty(X, shape, cone).
50. :- object(X), at(X, 1),
hasProperty(X, material, rubber).
51. :- object(X), at(X, 1),
hasProperty(X, shape, cube).
52. :- object(X), at(X, 2),
not hasProperty(X, size, medium).
53. :- object(X), at(X, 2),
not hasProperty(X, material, metal).
54. :- object(X), at(X, 2),
hasProperty(X, material, rubber).
55. :- object(X), at(X, 2),
hasProperty(X, shape, sphere).
56. :- object(X), at(X, 2),
hasProperty(X, shape, cube).
57. :- object(X), at(X, 3),
hasProperty(X, size, small).
58 :- object(X), at(X, 3),
not hasProperty(X, material, metal),
59. not hasProperty(X, color, blue).
60. :- #count{X1, X2: sameProperty(X1, X2, shape),
61. object(X1), object(X2), at(X1, 3), at(X2, 2),
62. hasProperty(X1, color, yellow),
63. hasProperty(X2, color, yellow)} >= 4.
64. :- #count{X1, X2: sameProperty(X1, X2, color),
65. object(X1), object(X2),
66. at(X1, 0), at(X2, 3)} >= 2.
\end{lstlisting}
以下是对前述的规则的逐行解释:
\begin{lstlisting}
44. 场景中共存在5个对象。
45. 区域0内禁止存在大尺寸对象。
46. 区域0内禁止存在圆柱形对象。
47. 区域0内禁止存在圆锥形对象。
48. 区域1内禁止存在小尺寸对象。
49. 区域1内禁止存在圆锥形对象。
50. 区域1内禁止存在橡胶材质对象。
51. 区域1内禁止存在立方体对象。
52. 区域2内所有对象必须为中尺寸。
53. 区域2内所有对象必须为金属材质。
54. 区域2内禁止存在橡胶材质对象。
55. 区域2内禁止存在球形对象。
56. 区域2内禁止存在立方体对象。
57. 区域3内禁止存在小尺寸对象。
58-59. 区域3内所有对象必须满足以下条件之一:金属材质或者蓝色外观。
60-63. 区域3与区域2内黄色对象组合规则:相同形状的黄色对象配对组数≤1。
64-66. 区域0与区域3联合约束: 具有相同颜色的对象配对组数 =0(严格禁止)。
\end{lstlisting}

\subsection{问题}

\subsection{回答集}

\subsection{推理步骤}

\section{数据集统计}
\subsection{问题模板分布}